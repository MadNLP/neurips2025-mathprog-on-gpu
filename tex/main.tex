\documentclass{article}

% if you need to pass options to natbib, use, e.g.:
%     \PassOptionsToPackage{numbers, compress}{natbib}
% before loading neurips_2025

% The authors should use one of these tracks.
% Before accepting by the NeurIPS conference, select one of the options below.
% 0. "default" for submission
\usepackage[preprint]{neurips_2025}
% the "default" option is equal to the "main" option, which is used for the Main Track with double-blind reviewing.
% 1. "main" option is used for the Main Track
%  \usepackage[main]{neurips_2025}
% 2. "position" option is used for the Position Paper Track
%  \usepackage[position]{neurips_2025}
% 3. "dandb" option is used for the Datasets & Benchmarks Track
 % \usepackage[dandb]{neurips_2025}
% 4. "creativeai" option is used for the Creative AI Track
%  \usepackage[creativeai]{neurips_2025}
% 5. "sglblindworkshop" option is used for the Workshop with single-blind reviewing
 % \usepackage[sglblindworkshop]{neurips_2025}
% 6. "dblblindworkshop" option is used for the Workshop with double-blind reviewing
%  \usepackage[dblblindworkshop]{neurips_2025}

% After being accepted, the authors should add "final" behind the track to compile a camera-ready version.
% 1. Main Track
 % \usepackage[main, final]{neurips_2025}
% 2. Position Paper Track
%  \usepackage[position, final]{neurips_2025}
% 3. Datasets & Benchmarks Track
 % \usepackage[dandb, final]{neurips_2025}
% 4. Creative AI Track
%  \usepackage[creativeai, final]{neurips_2025}
% 5. Workshop with single-blind reviewing
%  \usepackage[sglblindworkshop, final]{neurips_2025}
% 6. Workshop with double-blind reviewing
%  \usepackage[dblblindworkshop, final]{neurips_2025}
% Note. For the workshop paper template, both \title{} and \workshoptitle{} are required, with the former indicating the paper title shown in the title and the latter indicating the workshop title displayed in the footnote.
% For workshops (5., 6.), the authors should add the name of the workshop, "\workshoptitle" command is used to set the workshop title.
% \workshoptitle{WORKSHOP TITLE}

% "preprint" option is used for arXiv or other preprint submissions
 % \usepackage[preprint]{neurips_2025}

% to avoid loading the natbib package, add option nonatbib:
% \usepackage[nonatbib]{neurips_2025}


\usepackage[utf8]{inputenc} % allow utf-8 input
\usepackage[T1]{fontenc}    % use 8-bit T1 fonts
\usepackage{hyperref}       % hyperlinks
\usepackage{url}            % simple URL typesetting
\usepackage{booktabs}       % professional-quality tables
\usepackage{amsfonts}       % blackboard math symbols
\usepackage{nicefrac}       % compact symbols for 1/2, etc.
\usepackage{microtype}      % microtypography
\usepackage{xcolor}         % colors
\usepackage{glossaries}
\usepackage{enumitem}
\usepackage{listings}
\lstset{basicstyle=\ttfamily\small}
\usepackage{multirow}
\usepackage{cleveref}

\crefname{equation}{}{}


\newacronym{pde}{PDE}{partial differential equation}
\newacronym{ad}{AD}{automated differentiation}
\newacronym{alcf}{ALCF}{Argonne Leadership Computing Facility}
\newacronym{blas}{BLAS}{basic linear algebra subprograms}
\newacronym{iap}{IAP}{independent activities period}
\newacronym{der}{DER}{distributed energy resource}
\newacronym{derms}{DERMS}{distributed energy resource management system}
\newacronym{minlp}{MINLP}{mixed-integer nonlinear programming}
\newacronym{nlp}{NLP}{nonlinear programming}
\newacronym{kkt}{KKT}{Karush-Kuhn-Tucker}
\newacronym{sqp}{SQP}{sequential quadratic programming}
\newacronym{ipm}{IPM}{interior-point method}
\newacronym{cpu}{CPU}{central processing units}
\newacronym{gpu}{GPU}{graphics processing units}
\newacronym{mpc}{MPC}{model predictive control}
\newacronym{ac}{AC}{alternating current}
\newacronym{dc}{DC}{direct current}
\newacronym{opf}{OPF}{optimal power flow}
\newacronym{hpc}{HPC}{high-performance computing}
\newacronym{pcg}{PCG}{projected conjugate gradient}
\newacronym{alm}{ALM}{augmented Lagrangian method}
\newacronym{dac}{DAC}{direct air capture}
\newacronym{pem}{PEM}{proton exchange membrane}
\newacronym{tea}{TEA}{technoeconomic analysis}
\newacronym{lca}{LCA}{life cycle assessment}
\newacronym{ft}{FT}{Fischer-Tropsch}
\newacronym{bess}{BESS}{battery energy storage system}
\newacronym{cqa}{CQA}{critical quality attribute}
\newacronym{lnp}{LNP}{lipid nanoparticle}
\newacronym{iso}{ISO}{independent system operator}
\newacronym{simd}{SIMD}{single instruction, multiple data}
\newacronym{mimd}{MIMD}{multiple instruction, multiple data}
\newacronym{orcd}{ORCD}{Office of Research Computing and Data}
\newacronym{mitei}{MITei}{MIT Energy Initiative}
\newacronym{hipsat}{HIP-SAT}{High School Introduction to the Physical Sciences and Advanced Technologies}
\newacronym{mpi}{MPI}{message passing interface}
\newacronym{goc}{GOC}{grid optimization competition}
\newacronym{sc}{SC}{security-constrained}
\newacronym{mp}{MP}{multi-period}
\newacronym{admm}{ADMM}{alternating direction method of multipliers}
\newacronym{gmres}{GMRES}{generalized mean residual}
\newacronym{pdhg}{PDHG}{primal-dual hybrid gradient}
\newacronym{lp}{LP}{linear programming}
\newacronym{ams}{AMS}{algebraic modeling system}
\newacronym{ncl}{NCL}{nonlinear constrained Lagrangian}
\newacronym{bcl}{BCL}{bound constrained Lagrangian}
\newacronym{sqd}{SQD}{symmetric quasi-definite}
\newacronym{spd}{SPD}{symmetric positive definite}


% Note. For the workshop paper template, both \title{} and \workshoptitle{} are required, with the former indicating the paper title shown in the title and the latter indicating the workshop title displayed in the footnote.
\title{On the GPU Implementation of Second-Order Linear and Nonlinear Programming Solvers}



% The \author macro works with any number of authors. There are two commands
% used to separate the names and addresses of multiple authors: \And and \AND.
%
% Using \And between authors leaves it to LaTeX to determine where to break the
% lines. Using \AND forces a line break at that point. So, if LaTeX puts 3 of 4
% authors names on the first line, and the last on the second line, try using
% \AND instead of \And before the third author name.


\author{%
  Alexis Montoison\\
  Mathematics and Computer Science Division\\
  Argonne National Laboratory\\
  Lemont, IL 60439\\
  \texttt{amontoison@anl.gov}\\
  \And
  Fran\c{c}ois Pacaud\\
  Centre Automatique et Systèmes\\
  Mines Paris-PSL\\
  Paris, 75006 \\
  \texttt{francois.pacaud@minesparis.psl.eu}\\
  \And
  Sungho Shin\\
  Department of Chemical Engineering\\
  Massachusetts Institute of Technology\\
  Cambridge, MA 02139\\
  \texttt{sushin@mit.edu}\\
  \And
  Mihai Anitescu\\
  Mathematics and Computer Science Division\\
  Argonne National Laboratory\\
  Lemont, IL 60439\\
  \texttt{anitescu@mcs.anl.gov}\\
  % examples of more authors
  % Coauthor \\
  % Affiliation \\
  % Address \\
  % \texttt{email} \\
  % \AND
  % Coauthor \\
  % Affiliation \\
  % Address \\
  % \texttt{email} \\
  % \And
  % Coauthor \\
  % Affiliation \\
  % Address \\
  % \texttt{email} \\
  % \And
  % Coauthor \\
  % Affiliation \\
  % Address \\
  % \texttt{email} \\
}


\begin{document}


\maketitle


\begin{abstract}
In recent years, GPU-accelerated optimization solvers based on second-order methods (e.g., interior-point) have gained momentum with the advent of mature and efficient GPU-accelerated direct linear solvers, such as cuDSS.
This paper provides an overview of the state of the art in GPU-based second-order solvers, focusing on \emph{pivoting-free interior-point methods} for linear and nonlinear programming.
We begin by highlighting the capabilities and limitations of GPU-accelerated direct linear solvers currently available. 
Next, we discuss different formulations of \gls*{kkt} system for second-order methods and evaluate their suitability for GPU implementation.
We also discuss strategies for computing sparse Jacobian and Hessian matrices on GPUs.
Finally, we present numerical results showcasing scalability of the GPU solvers (often exceeding 10x speed-ups over the best CPU alternatives) and examine current limitations of existing solvers.
\end{abstract}

\section{Introduction}\label{eqn:intro}

This paper focuses on the implementation of solvers for problems of the following sort:
\begin{align}\label{eqn:opt}
  \min_{x } \; f(x) \quad \text{s.t.} \quad g(x) \geq 0,
\end{align}
where \(x \in \mathbb{R}^n\) is the decision variable, and \(f: \mathbb{R}^n \to \mathbb{R}\) and \(g: \mathbb{R}^n \to \mathbb{R}^m\) are smooth objective and constraint functions.
% For simplicity, we do not explicitly consider equality constraints.
We shall discuss both \gls*{lp} ($f$ and $g$ are affine) and \gls*{nlp} ($f$ and $g$ are nonlinear) and will place emphasis on the algorithms targeting large and sparse instances.
% Existing solvers like Ipopt, Knitro, OSQP, HiGHS, and Gurobi are designed to exploit sparsity and handle potentially large problem sizes.

Despite advances in general-purpose GPU computing, state-of-the-art mathematical programming solvers have not widely adopted these techniques.
GPUs excel in repetitive computations on large data sets, such as dense matrix multiplication in AI model training.
However, many mathematical programming problems in classical application areas are sparse, lack a uniform memory layout, and therefore do not benefit from the same kind of parallelism as in dense linear algebra.
As a result, integrating GPUs into mathematical programming solvers poses greater challenges and often necessitates a substantial modifications to the overall algorithm.

First-order algorithms have emerged as a suitable algorithm class for GPU implementation.
As these algorithms rely typically on sparse matrix-vector multiplication (SpMV) and simple vector operations, implementing GPU acceleration is less likely obstructed by the unavailability of low-level kernels.
Recent successful implementations include cuPDLP \cite{}, cuOSQP \cite{}, and cuOPT \cite{}. However, the linear convergence rate of first-order methods restricts their effectiveness in applications requiring fast convergence, prompting the exploration of second-order solvers.

The development of second-order solvers heavily depends on \emph{direct linear solvers}. Specifically, when using \gls*{ipm}, each barrier iteration necessitates solving a linear system, known as the \gls*{kkt} system. These \gls*{kkt} systems become highly ill-conditioned near the solution, which makes the use of iterative linear solvers, such as preconditioned Krylov methods, ineffective in most cases. The issue is exacerbated in nonconvex settings, where KKT systems lack \gls*{sqd} structure. For years, the advancement of GPU-accelerated second-order solvers has been limited by the absence of robust and efficient direct linear solvers for GPUs.

Recently, NVIDIA released cuDSS, a direct sparse solver library for GPUs, which provides Cholesky and LDL$^\top$ factorization capabilities, which are essential for many second-order solvers. While it currently lacks the LBL$^\top$ factorization capabilities commonly used for \gls*{nlp} solvers, its LDL$^\top$ and Cholesky functionalities are sufficient for implementing modified versions of the \gls*{ipm}. Consequently, cuDSS has spurred advances in GPU-accelerated second-order solvers, with several, such as MadNLP and Clarabel, integrating it and achieving significant speedups on large-scale instances.

This paper aims provide an overview of the current state of the art in the implementation of GPU-accelerated second-order optimization solvers, while focusing on the following aspects: (i) We focus on the \gls*{ipm} as the primary mechanism to handle inequality constraints, as the alternative paradigm, active-set methods, are widely accepted to be less scalable \cite{}. (ii) We focus on solving KKT systems, which present more challenges. Other components like barrier parameter tuning and line search can be straightforwardly ported on GPUs using low-level kernels through \texttt{map} or \texttt{reduce} operations. (iii) We focus on NVIDIA GPUs and their software stack, because as of now, NVIDIA and CUDA provides the most mature direct spase solver capabilities.


% In the remainder of the paper, we first review the latest GPU-accelerated direct linear solvers and their application in mathematical programming. We then explore different KKT system formulations for \gls*{ipm}s and assess their suitability for GPU implementation. We also discuss approaches for sparse automatic differentiation on GPUs, which is crucial for \gls*{nlp}. Finally, we present numerical results showcasing the scalability—often exceeding 10x speed-ups over leading CPU alternatives, followed by a discussion of current limitations and open questions.


\section{Direct Linear Solvers for Optimization}\label{eqn:linear}
This section provides an overview of the direct linear algebra methods frequently employed in second-order methods and discusses the rationale behind the development of pivoting-free \gls*{ipm}.

%% SS: Alexis, could you write  this section?
% This section summarizes the direct linear solver techniques commonly utilized within optimization solvers, with discussion on the emerging trends in GPU-accelerated direct linear solvers.
% Although GPU solvers has made significant progress in the recent years, their numerical stability, especially for handling indefinite linear systems is still rather limited.



\paragraph{LDL$^\top$ factorization}
LDL$^\top$ factorization, a signed variant of Choleskty decomposition, decomposes a matrix $A$ into $LDL^\top$, where $L$ is lower triangular and $D$ is diagonal.
LDL$^\top$ factorization can be utilized to solve $Ax = b$, where the solution is obtained by first solving the lower triangular system $Ly = b$, followed by diagonal scaling with $D^{-1}$ and the solution of upper triangular system $L^\top x = y$.
A nice property of LDL$^\top$ factorization is that, provided that the matrix $A$ is \gls*{sqd}, LDL$^\top$ factorization exists for any given permutation of the matrix (so-called strongly factorizable) \cite{}.
Many of the saddle point systems encountered in mathematical programming solvers are \gls*{sqd}, nearly \gls*{sqd} (can become \gls*{sqd} with infinitesimal regularization), or can be reduced into \gls*{spd} systems.
Thus, LDL$^\top$ factorization is often used within \gls*{ipm} solvers.

For sparse systems, if a fill-in reducing  permutation (reordering) $P$ is employed, resulting in the decomposition of $A = P L D L^\top P^\top$  form.
Although algorithms to compute fill-in-reducing reordering (e.g., minimum degree ordering or nested dissection) are serial in nature (e.g., cuDSS performs this operation on CPUs), the reordering can be computed only once and can be reused for multiple factorization calls, thus the overhead can be amortized over multiple calls.
% LDL$^\top$ factorization exists for more broad class of matrices, including many indefinite matrices.

% \paragraph{Ordering}




% Conversely, if the matrix is not \gls*{spd}, the Cholesky factorization will fail, and the algorithm will encounter a zero pivot.
% Thus, the failure of the Cholesky factorization can be used by the optimization sovler as a signal that the primal regularization needs to be increased to ensure the descent direction of the Newton's step.

\paragraph{Numerical Pivoting}

For general indefinite matrices without \gls*{sqd} structure, LDL$^\top$ factorization procedure is not guaranteed to be numerically stable, and dynamic numerical pivoting is commonly employed to enhance the numerical stability of the factorization process at the cost of leading additional fill-in (thus memory reallocation) and breaking memory ``design''.
Dynamic numerical pivoting procedures examine a limited set of candidate pivots, typically within a row and column, and selects the most suitable one according to a stability criterion, such as the largest absolute value.
Three widely used dynamic pivoting strategies are Bunch–Kaufman, rook, and delayed pivoting, which select $1 \times 1$ or $2 \times 2$ pivots, although other variants and hybrid approaches also exist \cite{}.
The variantof LDL$^\top$ with $2\times 2$ pivots is oftne called LBL$^\top$ factorization.
% When performing an LDL$^\top$ factorization, a suitable $1 \times 1$ pivot may not always exist; in such cases, a $2 \times 2$ block pivot is used instead.
% It should be noted that $2 \times 2$ pivoting alone does not guarantee the absence of zero pivots; it must be applied in a specific order, which is not easily ensured when processing the elimination tree in a block-wise fashion.
If none of these methods succeed, the pivot is perturbed by a small value (typically $\mathbf{u}\|A\|_1$ for some small $\mathbf{u} > 0$).
This procedure introduces numerical error, which must be corrected through iterative refinement.
% This procedure perturbs the matrix (introducing numerical error), so the error in the solution has to be reduced by using iterative refinement strategies.
% For dense matrices, Bunch-Kaufman pivoting for LBL$^\top$ (\emph{sytrf}) is often used and still exploit level-3 BLAS routines such as \emph{gemm} and \emph{trsm}.
% This procedure introduces numerical error, which must be corrected through iterative refinement.

\paragraph{State-of-the-art GPU solvers}
As described above, the numerical pivoting procedure is crucial to ensure numerical stability of direct linear solvers, but these strategies are  quite sophisticated and are non-trivial to implement on GPUs.
Moreover, since tree level parallelism must be employed to exploit GPU parallelism, the numerical pivoting should be applied in such a way that deos not break the parallelism at the elimination tree level, which further complicates the implementation.
Implementing numerical pivoting has been recognized as one of the most challenging components of the direct linear solver implementation on GPUs \cite{}.
The current version of cuDSS does have partial pivoting capabilities, but it does not support the LBL$^\top$ factorization used in CPU implementations.
% Since performing numerical pivoting on GPUs is known to be challenging and inefficient, it is always desired to transform a matrix in a Cholesky factorizable form to apply direct solution methods on GPUs.
% Therefore, to best utilize the benefits of the existing GPU direct solvers, it is crucial to devlop a \emph{pivoting-free} algorithm that does not rely heavily on numerical pivoting.
% In particular, if the linear systems solved within the optimization algortihm can be expressed in the form of \gls*{sqd} or at least near-\gls*{sqd} matrices, one can apply the solvers with LDL$^\top$ factorization methods (with a simpler version of partial pivoting) to solve the linear systems.

On the other hand, when the matrix is \gls*{sqd}, LDL$^\top$ factorization on GPU can achieve high degree of robustness even without numerical pivoting \cite{}.
Furthermore, these methods can effectively utilize parallelism; it is known that the factorization process is oftentimes highly parallelizable, while the triangular tends to be less so but still can benefit from GPU parallelism \cite{}.
% https://research.nvidia.com/sites/default/files/pubs/2011-06_Parallel-Solution-of/nvr-2011-001.pdf
Therefore, to fully exploit the benefits of existing GPU direct solvers, it is crucial to ensure that \emph{the \gls*{kkt} system can be solved without pivoting}, which motivates the development of \emph{pivoting-free interior-point methods}.

\section{Pivoting-free interior-point methods}\label{eqn:ipm}
We now explain how the \gls*{ipm} can be adapted to avoid numerical pivoting, thus enabling the use of the existing GPU direct solvers. We first provide a brief overview of the \gls*{ipm} and its KKT system formulation, followed by a discussion of the condensed KKT systems.

\paragraph{Interior-point methods and KKT systems}
The \gls*{ipm} is a class of optimization algorithms that are designed to solve inequality-constrained optimization problems like \cref{eqn:opt}. They transform \cref{eqn:opt} into a sequence of log-barrier subproblems:
\begin{align}\label{eqn:barrier}
  \min_{x,s } \; & f(x) - \mu \sum_{i=1}^m \log(s_i) \quad \text{s.t.} \; g(x) - s = 0 ,
\end{align}
where $s \in \mathbb{R}^m$ is the slack variable and $\mu > 0$ is the barrier parameter gradually sent to zero.

Problem \cref{eqn:barrier} is addressed by attempting to solve its KKT conditions:
\begin{align}\label{eqn:kkt}
  \nabla f(x) - \nabla g(x)^\top \lambda = 0, \quad
  \Lambda S e = \mu e,\quad
  g(x) - s =   0 ,
\end{align}
where $\lambda \in \mathbb{R}^m$ is the Lagrange multipliers, $S = \text{diag}(s)$,
 $\Lambda = \text{diag}(\lambda)$, and $e$ is the vector of ones.

\Gls*{ipm} attempts to iteratively solve \cref{eqn:kkt} using Newton's method.
At each iteration, we obtain the
search direction by solving the following (regularized) KKT system:
\begin{align}\label{eqn:kkt_system}
  \begin{bmatrix}
    \nabla^2_{x x} \mathcal{L}(x,\lambda) + \delta_p I&& \nabla g(x)^\top  \\
                                          & S^{-1}\Lambda & -I \\
    \nabla g(x) & -I &  - \delta_d I\\
  \end{bmatrix}
  \begin{bmatrix}
    \phantom{-}d_x\\
    \phantom{-}d_s \\
    -d_\lambda
  \end{bmatrix} =
  -\begin{bmatrix}
    \nabla f(x) - \nabla g(x)^\top \lambda\\
    \Lambda e - \mu S^{-1} e \\
    g(x) - s\\
  \end{bmatrix},
\end{align}
where $\mathcal{L}(x,\lambda,s)  := f(x) - \lambda^\top (g(x)-s)$ is the Lagrangian function and $\delta_p,\delta_d\geq 0$ are primal and dural regurlization parameters employed by the \gls*{ipm} solver to ensure the well-posedness of \cref{eqn:kkt_system} or the descent property of the Newton step.

% \paragraph{Eliminating slack variables}
% $S^{-1}\Lambda + \delta_p I$ is always invertible due to the nature of primal-dual \gls*{ipm}, as the slack $s$ and the dual variable $\lambda$ are always forced to be positive. The resulting system can be written as follows:
% \begin{align}\label{eqn:kkt_2}
%   \begin{bmatrix}
%     \nabla^2 \mathcal{L}(x,\lambda) + \delta_p I&  \nabla g(x)^\top \\
%     \nabla g(x) &  - \delta_d I - (S^{-1}\Lambda + \delta_p I)^{-1}\\
%   \end{bmatrix}
%   \begin{bmatrix}
%     *\\
%     *\\
%   \end{bmatrix} =
%   -\begin{bmatrix}
%     *\\
%     *\\
%   \end{bmatrix},
% \end{align}
% Since $S^{-1}\Lambda + \delta_p I$ is diagonal, the elimination procedure does not incur significant computational overhead and the number of non-zero entries in the resulting system is not increased. However, since $S^{-1}\Lambda$ can be arbitrarily ill-conditioned, the numerical stability of the resulting system can be compromised. This system is \gls*{sqd} if the original problem is strongly convex (thus LDL$^\top$ factorizable), but for general linear programs, the $(1,1)$-block is not guaranteed to be \gls*{spd}. For general \glspl*{nlp}, even the $(1,1)$-block can have negative eigenvalues, and thus, LBL$^\top$ method with partial pivoting needs to be employed to solve the in a numerically stable fashion.
\paragraph{Condensed KKT systems}
While \cref{eqn:kkt_system} is directly addressed by some solvers (e.g., Ipopt \cite{}), the block structure in \cref{eqn:augmentedKKT} admits several possibilities for further reduction. In the context of GPU, they offer advantages sometimes simply by making the system smaller and denser (providing more opportunities for exploiting parallelism) or sometimes by enabling numerically stable factorization without pivoting. Below, we outline such elimination strategies.

\begin{itemize}[leftmargin=*,itemsep=0pt,parsep=0pt,partopsep=0pt]
\item \textbf{Primal-dual condensed system.}
As the $(2,2)$-block in \cref{eqn:kkt_system} is always invertible due to the nature of the \gls*{ipm}, we can eliminate the slack variable $s$ to obtain the following block $2\times 2$ system:
\begin{equation}\label{eqn:augmentedKKT}
  \begin{bmatrix}
    \nabla^2_{x x} \mathcal{L}(x,\lambda) + \delta_p I & \nabla g(x)^\top \\
    \nabla g(x) &  - \delta_d I - \Lambda^{-1} S
  \end{bmatrix}
  \begin{bmatrix}
    \phantom{-}d_x\\
    - d_\lambda
  \end{bmatrix} =
  -\begin{bmatrix}
    \nabla f(x) - \nabla g(x)^\top \lambda\\
    g(x) - \mu \Lambda^{-1} e
  \end{bmatrix}.
\end{equation}
The slack step $d_s$ can be recovered once the primal and dual steps $d_x$ and $d_\lambda$ are computed. This elimination does not incur significant computational overhead, and the number of non-zero entries in the resulting system is not increased. However, since $\Lambda^{-1} S$ can be arbitrarily ill-conditioned, the numerical stability of the resulting system can be compromised.

\item \textbf{Primal condensed system.}
The block $\delta_d I + \Lambda^{-1}S$ within \eqref{eqn:augmentedKKT} is always invertible; their elimination gives rise to a \emph{primal condensed KKT system}:
\begin{align}\label{eqn:kkt_primal}
  % \left(\nabla^2 \mathcal{L}(x,\lambda) + \delta_p I + \nabla g(x)^\top \left(\delta_d I + (S^{-1}\Lambda + \delta_p I)^{-1}\right) \nabla g(x)\right)
  % \begin{bmatrix}
  %   *\\
  % \end{bmatrix} =
  % -\begin{bmatrix}
  %   *\\
  % \end{bmatrix}.
  \left(\nabla^2_{x x} \mathcal{L}(x, \lambda) + \delta_p I + \nabla g(x)^\top (\delta_d I + \Lambda^{-1} S)^{-1} \nabla g(x)  \right) d_x = - r_p \; ,
\end{align}
with $r_p$ an appropriate right-hand-side derives from \eqref{eqn:augmentedKKT}.
Compared to \cref{eqn:augmentedKKT}, the system size is further reduced.
However, since the Jacobian $\nabla g(x)$ can have dense rows, the condensed system can become arbitrarily dense, potentially causing serious computational overhead. F
urthermore, this elimination strategy suffers from the same potential ill-conditioning issue as in \cref{eqn:augmentedKKT}.
However, this condensation has the key advantage for \gls*{nlp}, as this system becomes \gls*{spd} under the application of primal-dual regularization $(\delta_p, \delta_p)$ chosen based on the standard inertia correction procedure \cite{}.
Thus, the system is always Cholesky factorizable with static pivoting, thus allowing the pivoting-free implementation of the \gls*{ipm} for \glspl*{nlp}.

\item \textbf{Dual condensed system.}
When the problem is strongly convex or when the regularization parameter $\delta_p$ is sufficiently large, the $\nabla^2 \mathcal{L}(x,\lambda) + \delta_p I$ block is invertible, and by eliminating the this block, we obtain the \emph{dual condensed KKT system}:
\begin{align}\label{eqn:kkt_dual}
  \left(\delta_d I + \Lambda^{-1}S + \nabla g(x)\left(\nabla_{x x}^2 \mathcal{L}(x,\lambda) + \delta_p I\right)^{-1} \nabla g(x)^\top\right)
  d_\lambda = - r_d \; ,
  % \begin{bmatrix}
  %   *\\
  % \end{bmatrix} =
  % -\begin{bmatrix}
  %   *\\
  % \end{bmatrix},
\end{align}
with $r_d$ an appropriate right-hand-side.
In the context of linear programming, this system is often called \emph{normal equations}.
Unless $\nabla^2 \mathcal{L}(x,\lambda) + \delta_p I$ is diagonal, this system can be arbitrarily dense, and thus, this elimination needs to be used with caution. Assuming that the primal Hessian is \gls*{spd}, this system is always \gls*{spd}, thus is Cholesky factorizable with static pivoting. 
\end{itemize}


\paragraph{Regularization of KKT systems}
We now explain how the $(\delta_p, \delta_d)$ parameters are chosen.
% As seen before, the algorithm tunes the regularization parameters $\delta_p$
% and $\delta_p$ to ensure the system \cref{eqn:augmentedKKT} is well-posed.
\begin{itemize}[leftmargin=*,itemsep=0pt,parsep=0pt,partopsep=0pt]
\item \textbf{Convex case}:
  For strongly convex programs, all four formulations \cref{eqn:kkt_system,eqn:augmentedKKT,eqn:kkt_dual,eqn:kkt_primal} are appropriate, so one may choose the best one based on the sparsity pattern and degree of ill-conditioning of the KKT system matrix.
  Even when the problem lacks strong convexity, the system \cref{eqn:augmentedKKT} can become SQD for infinitsimal $(\delta_x, \delta_d) > 0$, and thus, become strongly factorizable.
  This idea has lead to several robust \gls{ipm} implementations~\cite{}.
  % Reference Friedlander and Orban
  % This is the approach adopted in MadIPM.
  For linear program, the dual condensed KKT system~\cref{eqn:kkt_dual} is often used
  by default as the diagonal block $\nabla^2_{x x} \mathcal{L}(x, \lambda) + \delta_p I$ is easy
  to invert in that case. Customized implementation of the Cholesky factorization
  are used to handle dense columns in the Jacobian $\nabla g(x)$ \cite{}.
% For general quadratic programs, \cref{eqn:kkt_primal} is most suitable, though other formulations, such as \cref{eqn:kkt_system,eqn:augmentedKKT,eqn:kkt_dual} can also be used when proper regularization, pivot perturbation, and/or iterative refinement strategies are employed.
% For strongly convex programs, all four formulations \cref{eqn:kkt_system,eqn:augmentedKKT,eqn:kkt_dual,eqn:kkt_primal} are appropriate, so one may choose the best one based on the sparsity pattern and degree of ill-conditioning of the KKT system matrix.
\item \textbf{Nonconvex case}:
%   The case of \gls*{nlp} is different, as the solver has to accomodate
%   the local nonconvexities in the model.
  For nonconvex problems, primal-dual regularization provides a mechanism to ensure that the Newton step is a descent direction for a merit function. Traditional implementations of \gls{ipm} use so-called \emph{inertia correction} procedure, where the regularization parameters $(\delta_p, \delta_d)$ are increased until the number of positive, negative, and zero eigenvalues (so-called inertia, available as a byproduct of the LDL$^\top$ and LBL$^\top$ factorizations) of the KKT system matrix in \cref{eqn:kkt_system} is equal to $(n+m, m, 0)$.
  This condition is equivalent to the condition where the primal condensed system \cref{eqn:kkt_primal} is \gls*{spd}, thus confirming that the primal condensed system enables the pivoting-free implementation of \gls*{ipm} for \glspl*{nlp}.
% If the problem~\cref{eqn:opt} has additional equality constraints,
% the equality constraints are relaxed into doubly bounded inequalities with a small gap, called LiftedKKT system strategy.
\end{itemize}



\section{Algebraic Modeling Systems and Automatic Differentiation}\label{eqn:ad}
\Glspl*{nlp} solvers require external oracle to  evaluate $f$, $g$, and their first/second-order derivatives.
In most modern optimization software stacks, the derivative evaluation code (either compiled or interpreted) is generated in a fully automated fashion through the so-called algebraic modeling systems, typically equipped with \gls*{ad} capabilities, such as AMPL, GAMS, CasADi, JuMP, Pyomo, and others.
To enable efficient derivative evaluations and ensure a fully GPU-resident optimization workflow, it is crucial to develop algebraic modeling systems that can provide derivative evaluation code in the form of GPU kernels.

To achieve this, one can focus on the fact that many practical instances of the large-scale sparse mathematical programs exhibit highly repetitive structure.
For example, $f$ may be a sum of many terms (e.g., $f(x) = \sum_{p\in P} f^o(x; p)$) and $g$ may be a collection of many constraints generated from a common template (e.g., $g(x) = \left\{g^o(x; p)\right\}_{p\in P}$).
If such a structure exists, which is true for many classical optimization instances, the evaluation and differentiation of $f$ and $g$ becomes embarassingly parallel, and thus, makes it possible to construct them as GPU kernels.
The emerging algebraic modeling systems, such as ExaModels.jl and Gravity, are designed to capture such repeated structures, and provide the derivative evaluation kernels in an automated fashion.
For example, ExaModels.jl requires the user to specify the objective and constraint functions always in a form of iterator, such as
\begin{lstlisting}
  objective(c, 100 * (x[i-1]^2 - x[i])^2 + (x[i-1] - 1)^2 for i = 2:N)
\end{lstlisting}
which corresponds to the case of $f^o(x; p) = 100(x[p-1]^2 - x[p])^2 + (x[p-1]-1)^2$ and $p=\{2,\cdots,N\}$.
This syntax allows the user to inform the modeling system of repeated structured in the model, so that later the GPU kernel for derivative evaluation can be generated.
% As of now, to the best of our knowledge, the classical modeling systems, such as AMPL, GAMS, CasADi, and JuMP, do not store the data in such a structured format and are incompatible with GPUs.
% AM: They could exploit partial separability but it is another kind of parallelism.

\section{Numerical Results}\label{eqn:num}
This section provides performance benchmarks for the GPU solvers against the state-of-the-art CPU solvers.
% Publicly available benchmark instances, such as MIPLIB, pglib-opf, and COPS, are used to evaluate the performance of the solvers.
% The following versions of the software are used within the benchmark: Julia v1.11.6, MadNLP v0.8.8, ExaModels v0.9.0, CUDSS v0.6.0, HiGHS v1.7.0, and Gurobi v10.0.0.
The numerical results can be reproduced with the source code and Manifest file available at \url{https://github.com/MadNLP/neurips2025-mathprog-on-gpu}.
The benchmark was performed on a workstation with two Intel(R) Xeon(R) Platinum 6130 CPU @ 2.10GHz, two Quadro GV 100 GPUs, and 128 GB of RAM.
When reporting the solve time for multiple instances, we represent them by shifted geometric mean: $\left(\prod_{i=1}^n (t_i + \Delta)\right)^{1/n} - \Delta$, where $t_i$ is the solve time for the $i$-th instance and $\Delta$ is the shift factor.
We denote this metric with $\Delta = 10$ as SGM10.
If an instance is unsolved, its solving time is set to the corresponding time limit.


\subsection{Linear Programming}
We have performed the benchmark on the MIPLIB 2017 benchmark library \cite{}.
Two solvers, MadIPM running on GPU and Gurobi running on CPU, are compared.
MadIPM solves the primal-dual condensed KKT system \cref{eqn:augmentedKKT} using cuDSS as the linear solver, while Gurobi uses its internal barrier method implementation with a custom sparse symmetric indefinite direct solver.

\paragraph{MIPLIB}
% \begin{table}[t]
%   \centering\footnotesize
%   \caption{Solve time in seconds and SGM10 on instances of MIPLIB without presolve}
% \end{table}
\begin{center}
  \begin{tabular}{|c|c|cc|cc|cc|cc|}
    \hline
    \multirow{ 3}{*}{Tol} & \multirow{ 3}{*}{Solver} & \multicolumn{2}{c|}{\textbf{Small} (0)}& \multicolumn{2}{c|}{\textbf{Medium} (0)}& \multicolumn{2}{c|}{\textbf{Large} (4)}& \multicolumn{2}{c|}{\multirow{2}{*}{\textbf{Total} (4)}}\\
                          && \multicolumn{2}{c|}{(900 sec max)}& \multicolumn{2}{c|}{(900 sec max)}& \multicolumn{2}{c|}{(900 sec max)}&&\\
                          &&  Solved & Time &  Solved & Time &  Solved & Time &  Solved & Time \\
    \hline
    \multirow{2}{*}{$10^{-4}$} & MadNLP (gpu) & 0 & -9.0 & 0 & -9.0 & 2 & 1.2854801994313352 & 2 & 1.2854801994313352  \\
                          & Ipopt (cpu) & 0 & -9.0 & 0 & -9.0 & 2 & 13.338996752053031 & 2 & 13.338996752053031  \\

    \hline
    \multirow{2}{*}{$10^{-4}$} & MadNLP (gpu) & 0 & -9.0 & 0 & -9.0 & 2 & 1.8189292455720203 & 2 & 1.8189292455720203  \\
                          & Ipopt (cpu) & 0 & -9.0 & 0 & -9.0 & 2 & 15.189445328247583 & 2 & 15.189445328247583  \\

    \hline
  \end{tabular}
\end{center}

\subsection{Nonlinear programming}
In this benchmark, we compare the performance of MadNLP running on GPU and Ipopt running on CPU.
MadNLP solves the primal condensed KKT system \cref{eqn:kkt_primal}, while Ipopt solves the augmented KKT system \cref{eqn:augmentedKKT}.
The solver performance is benchmarked against the pglib-opf benchmark suite \cite{} and the COPS benchmark suite \cite{}. For both solvers, the \gls*{nlp} function callbacks are provided by ExaModels.jl (either running on GPU or CPU).
MadNLP is configured with cuDSS as the linear solver, while Ipopt is configured with either Ma27 (for pglib-opf) or Ma57 (for COPS).
OpenBLAS is used as BLAS and LAPACK backends inside Ipopt and HSL linear solvers.

\paragraph{AC Optimal Power Flow}
The results are summarized in the table below:
\begin{center}
  \footnotesize
  \begin{tabular}{|c|c|cc|cc|cc|cc|}
  \hline
  \multirow{ 3}{*}{Tol} & \multirow{ 3}{*}{Solver} & \multicolumn{2}{c|}{\textbf{Small} (31)}& \multicolumn{2}{c|}{\textbf{Medium} (24)}& \multicolumn{2}{c|}{\textbf{Large} (11)}& \multicolumn{2}{c|}{\multirow{2}{*}{\textbf{Total} (66)}}\\
                        && \multicolumn{2}{c|}{nnz$<2^{18}$}& \multicolumn{2}{c|}{$2^{18}\leq$nnz$<2^{20}$}& \multicolumn{2}{c|}{$2^{20}\leq$ nnz}&&\\
                        &&  Solved & Time &  Solved & Time &  Solved & Time &  Solved & Time \\
  \hline
    \multirow{2}{*}{$10^{-4}$} & MadNLP (gpu) & 31 & 0.4166 & 24 & 2.6380 & 11 & 3.7040 & 66 & 1.6979  \\
                        & Ipopt (cpu) & 31 & 0.3970 & 24 & 5.0697 & 11 & 38.5053 & 66 & 5.3817  \\

  \hline
    \multirow{2}{*}{$10^{-8}$} & MadNLP (gpu) & 30 & 2.5037 & 24 & 4.6016 & 10 & 12.8040 & 64 & 4.6228  \\
                        & Ipopt (cpu) & 31 & 0.5100 & 24 & 5.4292 & 11 & 37.7818 & 66 & 5.5541  \\

  \hline
\end{tabular}

\end{center}
The results indicate that the GPU solver can achieve more than 10x speed-up on avarage for the 11 largest instances (with more than $2^{20}$ non-zeros) when the problem is solved up to low precision.
The speed-up is rather small for medium-sized instances, and there is practically no advantage for small instances.
This is expected, as the GPU solver is designed to handle large-scale problems, and small-scale problems cannot fully utilize the available parallel cores.
In such cases, the overhead related to parallelism, such as task scheduling and thread launching, dominates the computation time, rather than actual performance gains.
For high precision, however, the GPU solver is not able to achieve the same level of performance as the CPU solver, which is expected, as the the condensed system utilized within the GPU solvers must handle worse conditioning in practice.
One can see that in total, the GPU solver converges in 2 less instances than the CPU solver.
However, for the converged instances, the speed-up is still substantial (3x speed-up in SGM 10 on large instances).
% \begin{table}[t]
%   \centering\footnotesize
%   \caption{Solve time in seconds and SGM10 on instances of pglib-opf}\label{tab:pglib-opf}
% \end{table}


\paragraph{COPS Benchmark}
The results are summarized in the table below:
\begin{center}
  \footnotesize
  \begin{tabular}{|c|c|cc|cc|cc|cc|}
  \hline
  \multirow{ 3}{*}{Tol} & \multirow{ 3}{*}{Solver} & \multicolumn{2}{c|}{\textbf{Small} (13)}& \multicolumn{2}{c|}{\textbf{Medium} (16)}& \multicolumn{2}{c|}{\textbf{Large} (16)}& \multicolumn{2}{c|}{\multirow{2}{*}{\textbf{Total} (45)}}\\
                        && \multicolumn{2}{c|}{nnz$<2^{18}$}& \multicolumn{2}{c|}{$2^{18}\leq$nnz$<2^{20}$}& \multicolumn{2}{c|}{$2^{20}\leq$ nnz}&&\\
                        &&  Solved & Time &  Solved & Time &  Solved & Time &  Solved & Time \\
  \hline
    \multirow{2}{*}{$10^{-4}$} & MadNLP (gpu) & 13 & 0.8665 & 15 & 4.8665 & 16 & 3.8194 & 44 & 3.2314  \\
                        & Ipopt (cpu) & 13 & 5.2315 & 15 & 15.9701 & 15 & 45.8411 & 43 & 19.2243  \\

  \hline
    \multirow{2}{*}{$10^{-8}$} & MadNLP (gpu) & 13 & 0.8575 & 16 & 1.5572 & 16 & 8.3549 & 45 & 3.3797  \\
                        & Ipopt (cpu) & 13 & 5.9413 & 15 & 17.6758 & 15 & 40.8639 & 43 & 19.2999  \\

  \hline
\end{tabular}
%%% Local Variables:
%%% mode: LaTeX
%%% TeX-master: "main"
%%% End:

\end{center}
The results are largely similar to the pglib-opf benchmark, but the speed-up is more pronounced in these instances, as COPS benchmark library is designed to be scalable, thus exhibiting highly regular structure, which provides more opportunities for exploiting parallelism.
Again, for large instances, one can achieve more than 10x speed-up on average, while for small- and medium-sized instances, the GPU solver is not able to outperform the CPU solver.
Speed-up is substantial also for high-precision solves, but the speed-up factor is smaller overall.

% \begin{table}[t]
%   \centering\footnotesize
%   \caption{Solve time in seconds and SGM10 on instances of COPS Benchmark}\label{tab:cops}
% \end{table}

%%%%%%%%%%%%%%%%%%%%%%%%%%%%%%%%%%%%%%%%%%%%%%%%%%%%%%%%%%%%

\section{Conclusions and Future Outlook}\label{eqn:conclusion}
We have presented a current landscape of GPU-accelerated second-order optimization solvers.
We also have demonstrated with numerical examples, that GPU acceleration can achieve more than an order of magnitude speed-up for large-scale instances.
However, some open questions and implementation challenges remain, which we summarize below.
\begin{itemize}[leftmargin=*,itemsep=0pt,parsep=0pt,partopsep=0pt]
\item \textbf{Numerical precisions of condensed KKT systems}: The KKT system often needs to be expressed in a condensed form, which may detrimentally affect the numerical stability of the system.
Further research is needed to understand the numerical properties of the condensed KKT systems and to develop strategies to mitigate the numerical issue for high-precision solves.
\item \textbf{Alternative Strategies}: While we have presented the plain \gls*{ipm}, the it can be combined with other optimization strategies, such as the augmented Lagrangian method. Augmented Lagrangian or penalty-based approahces, which naturally provides a mechanism to eliminate the indefiniteness of the KKT system, may be worth investigating in the context of GPU-accelerated solvers.
% \item \textbf{Pivoting-Free Factorization Routines}: % Indefinite matrices can be numerically stably LDL$^\top$ factorized if a proper ordering is employed.
% It would be interesting to investigate whether there exists a fill-reducing ordering for KKT systems that also ensures the numerical stability of the factorization without numerical pivoting.
% AM: It is more or less a matching but it requires the nonzeros at the analysis phase. In our case we change the nnz of the KKT system at each IPM iteration.
\item \textbf{Batch mathematical programming solvers}: GPU-accelerated solvers can provide the capabilities to solve many small- to medium-sized problems in parallel, and thus, it is interesting to develop batch mathematical programming solvers.
Since the release 0.6 of CUDSS, both uniform and non-uniform batch linear systems are supported. 
% \item \textbf{Utilizing tensor cores and dense linear algebra routines}: Currently, in mathematical programming solvers, tensor cores are underutilized as most focus is on sparse linear algebra routines.
% However, it would be interesting to develop solvers that can utilize tensor cores, e.g., in supernodal or multifrontal context, to handle dense blocks within an overall sparse KKT system.
% AM: It is (probably) what they do in CUDSS (90% chance)
\item \textbf{Hardware portability and cross-platform implementations}:
  Currently, most existing optimization and linear solvers are limited to NVIDIA GPUs, but it is of interest to develop hardware-agnostic solvers that can run on a variety of GPU architectures, such as AMD and Intel GPUs.
  A key requirement for this will be the development of portable sparse LDL$^\top$ factorization routines.
% However, since \emph{sytrs} is still not implemented in rocSOLVER or oneAPI, the dense LBL$^\top$ factorization is not yet fully usable, so development of the sparse version may have to wait.
\end{itemize}
%% SS will write this part

\appendix

\section{Technical Appendices and Supplementary Material}
Technical appendices with additional results, figures, graphs and proofs may be submitted with the paper submission before the full submission deadline (see above), or as a separate PDF in the ZIP file below before the supplementary material deadline. There is no page limit for the technical appendices.

%%%%%%%%%%%%%%%%%%%%%%%%%%%%%%%%%%%%%%%%%%%%%%%%%%%%%%%%%%%%

\newpage
\section*{NeurIPS Paper Checklist}

%%% BEGIN INSTRUCTIONS %%%
The checklist is designed to encourage best practices for responsible machine learning research, addressing issues of reproducibility, transparency, research ethics, and societal impact. Do not remove the checklist: {\bf The papers not including the checklist will be desk rejected.} The checklist should follow the references and follow the (optional) supplemental material.  The checklist does NOT count towards the page
limit.

Please read the checklist guidelines carefully for information on how to answer these questions. For each question in the checklist:
\begin{itemize}
    \item You should answer \answerYes{}, \answerNo{}, or \answerNA{}.
    \item \answerNA{} means either that the question is Not Applicable for that particular paper or the relevant information is Not Available.
    \item Please provide a short (1–2 sentence) justification right after your answer (even for NA).
   % \item {\bf The papers not including the checklist will be desk rejected.}
\end{itemize}

{\bf The checklist answers are an integral part of your paper submission.} They are visible to the reviewers, area chairs, senior area chairs, and ethics reviewers. You will be asked to also include it (after eventual revisions) with the final version of your paper, and its final version will be published with the paper.

The reviewers of your paper will be asked to use the checklist as one of the factors in their evaluation. While "\answerYes{}" is generally preferable to "\answerNo{}", it is perfectly acceptable to answer "\answerNo{}" provided a proper justification is given (e.g., "error bars are not reported because it would be too computationally expensive" or "we were unable to find the license for the dataset we used"). In general, answering "\answerNo{}" or "\answerNA{}" is not grounds for rejection. While the questions are phrased in a binary way, we acknowledge that the true answer is often more nuanced, so please just use your best judgment and write a justification to elaborate. All supporting evidence can appear either in the main paper or the supplemental material, provided in appendix. If you answer \answerYes{} to a question, in the justification please point to the section(s) where related material for the question can be found.

IMPORTANT, please:
\begin{itemize}
    \item {\bf Delete this instruction block, but keep the section heading ``NeurIPS Paper Checklist"},
    \item  {\bf Keep the checklist subsection headings, questions/answers and guidelines below.}
    \item {\bf Do not modify the questions and only use the provided macros for your answers}.
\end{itemize}


%%% END INSTRUCTIONS %%%


\begin{enumerate}

\item {\bf Claims}
    \item[] Question: Do the main claims made in the abstract and introduction accurately reflect the paper's contributions and scope?
    \item[] Answer: \answerTODO{} % Replace by \answerYes{}, \answerNo{}, or \answerNA{}.
    \item[] Justification: \justificationTODO{}
    \item[] Guidelines:
    \begin{itemize}
        \item The answer NA means that the abstract and introduction do not include the claims made in the paper.
        \item The abstract and/or introduction should clearly state the claims made, including the contributions made in the paper and important assumptions and limitations. A No or NA answer to this question will not be perceived well by the reviewers.
        \item The claims made should match theoretical and experimental results, and reflect how much the results can be expected to generalize to other settings.
        \item It is fine to include aspirational goals as motivation as long as it is clear that these goals are not attained by the paper.
    \end{itemize}

\item {\bf Limitations}
    \item[] Question: Does the paper discuss the limitations of the work performed by the authors?
    \item[] Answer: \answerTODO{} % Replace by \answerYes{}, \answerNo{}, or \answerNA{}.
    \item[] Justification: \justificationTODO{}
    \item[] Guidelines:
    \begin{itemize}
        \item The answer NA means that the paper has no limitation while the answer No means that the paper has limitations, but those are not discussed in the paper.
        \item The authors are encouraged to create a separate "Limitations" section in their paper.
        \item The paper should point out any strong assumptions and how robust the results are to violations of these assumptions (e.g., independence assumptions, noiseless settings, model well-specification, asymptotic approximations only holding locally). The authors should reflect on how these assumptions might be violated in practice and what the implications would be.
        \item The authors should reflect on the scope of the claims made, e.g., if the approach was only tested on a few datasets or with a few runs. In general, empirical results often depend on implicit assumptions, which should be articulated.
        \item The authors should reflect on the factors that influence the performance of the approach. For example, a facial recognition algorithm may perform poorly when image resolution is low or images are taken in low lighting. Or a speech-to-text system might not be used reliably to provide closed captions for online lectures because it fails to handle technical jargon.
        \item The authors should discuss the computational efficiency of the proposed algorithms and how they scale with dataset size.
        \item If applicable, the authors should discuss possible limitations of their approach to address problems of privacy and fairness.
        \item While the authors might fear that complete honesty about limitations might be used by reviewers as grounds for rejection, a worse outcome might be that reviewers discover limitations that aren't acknowledged in the paper. The authors should use their best judgment and recognize that individual actions in favor of transparency play an important role in developing norms that preserve the integrity of the community. Reviewers will be specifically instructed to not penalize honesty concerning limitations.
    \end{itemize}

\item {\bf Theory assumptions and proofs}
    \item[] Question: For each theoretical result, does the paper provide the full set of assumptions and a complete (and correct) proof?
    \item[] Answer: \answerTODO{} % Replace by \answerYes{}, \answerNo{}, or \answerNA{}.
    \item[] Justification: \justificationTODO{}
    \item[] Guidelines:
    \begin{itemize}
        \item The answer NA means that the paper does not include theoretical results.
        \item All the theorems, formulas, and proofs in the paper should be numbered and cross-referenced.
        \item All assumptions should be clearly stated or referenced in the statement of any theorems.
        \item The proofs can either appear in the main paper or the supplemental material, but if they appear in the supplemental material, the authors are encouraged to provide a short proof sketch to provide intuition.
        \item Inversely, any informal proof provided in the core of the paper should be complemented by formal proofs provided in appendix or supplemental material.
        \item Theorems and Lemmas that the proof relies upon should be properly referenced.
    \end{itemize}

    \item {\bf Experimental result reproducibility}
    \item[] Question: Does the paper fully disclose all the information needed to reproduce the main experimental results of the paper to the extent that it affects the main claims and/or conclusions of the paper (regardless of whether the code and data are provided or not)?
    \item[] Answer: \answerTODO{} % Replace by \answerYes{}, \answerNo{}, or \answerNA{}.
    \item[] Justification: \justificationTODO{}
    \item[] Guidelines:
    \begin{itemize}
        \item The answer NA means that the paper does not include experiments.
        \item If the paper includes experiments, a No answer to this question will not be perceived well by the reviewers: Making the paper reproducible is important, regardless of whether the code and data are provided or not.
        \item If the contribution is a dataset and/or model, the authors should describe the steps taken to make their results reproducible or verifiable.
        \item Depending on the contribution, reproducibility can be accomplished in various ways. For example, if the contribution is a novel architecture, describing the architecture fully might suffice, or if the contribution is a specific model and empirical evaluation, it may be necessary to either make it possible for others to replicate the model with the same dataset, or provide access to the model. In general. releasing code and data is often one good way to accomplish this, but reproducibility can also be provided via detailed instructions for how to replicate the results, access to a hosted model (e.g., in the case of a large language model), releasing of a model checkpoint, or other means that are appropriate to the research performed.
        \item While NeurIPS does not require releasing code, the conference does require all submissions to provide some reasonable avenue for reproducibility, which may depend on the nature of the contribution. For example
        \begin{enumerate}
            \item If the contribution is primarily a new algorithm, the paper should make it clear how to reproduce that algorithm.
            \item If the contribution is primarily a new model architecture, the paper should describe the architecture clearly and fully.
            \item If the contribution is a new model (e.g., a large language model), then there should either be a way to access this model for reproducing the results or a way to reproduce the model (e.g., with an open-source dataset or instructions for how to construct the dataset).
            \item We recognize that reproducibility may be tricky in some cases, in which case authors are welcome to describe the particular way they provide for reproducibility. In the case of closed-source models, it may be that access to the model is limited in some way (e.g., to registered users), but it should be possible for other researchers to have some path to reproducing or verifying the results.
        \end{enumerate}
    \end{itemize}


\item {\bf Open access to data and code}
    \item[] Question: Does the paper provide open access to the data and code, with sufficient instructions to faithfully reproduce the main experimental results, as described in supplemental material?
    \item[] Answer: \answerTODO{} % Replace by \answerYes{}, \answerNo{}, or \answerNA{}.
    \item[] Justification: \justificationTODO{}
    \item[] Guidelines:
    \begin{itemize}
        \item The answer NA means that paper does not include experiments requiring code.
        \item Please see the NeurIPS code and data submission guidelines (\url{https://nips.cc/public/guides/CodeSubmissionPolicy}) for more details.
        \item While we encourage the release of code and data, we understand that this might not be possible, so “No” is an acceptable answer. Papers cannot be rejected simply for not including code, unless this is central to the contribution (e.g., for a new open-source benchmark).
        \item The instructions should contain the exact command and environment needed to run to reproduce the results. See the NeurIPS code and data submission guidelines (\url{https://nips.cc/public/guides/CodeSubmissionPolicy}) for more details.
        \item The authors should provide instructions on data access and preparation, including how to access the raw data, preprocessed data, intermediate data, and generated data, etc.
        \item The authors should provide scripts to reproduce all experimental results for the new proposed method and baselines. If only a subset of experiments are reproducible, they should state which ones are omitted from the script and why.
        \item At submission time, to preserve anonymity, the authors should release anonymized versions (if applicable).
        \item Providing as much information as possible in supplemental material (appended to the paper) is recommended, but including URLs to data and code is permitted.
    \end{itemize}


\item {\bf Experimental setting/details}
    \item[] Question: Does the paper specify all the training and test details (e.g., data splits, hyperparameters, how they were chosen, type of optimizer, etc.) necessary to understand the results?
    \item[] Answer: \answerTODO{} % Replace by \answerYes{}, \answerNo{}, or \answerNA{}.
    \item[] Justification: \justificationTODO{}
    \item[] Guidelines:
    \begin{itemize}
        \item The answer NA means that the paper does not include experiments.
        \item The experimental setting should be presented in the core of the paper to a level of detail that is necessary to appreciate the results and make sense of them.
        \item The full details can be provided either with the code, in appendix, or as supplemental material.
    \end{itemize}

\item {\bf Experiment statistical significance}
    \item[] Question: Does the paper report error bars suitably and correctly defined or other appropriate information about the statistical significance of the experiments?
    \item[] Answer: \answerTODO{} % Replace by \answerYes{}, \answerNo{}, or \answerNA{}.
    \item[] Justification: \justificationTODO{}
    \item[] Guidelines:
    \begin{itemize}
        \item The answer NA means that the paper does not include experiments.
        \item The authors should answer "Yes" if the results are accompanied by error bars, confidence intervals, or statistical significance tests, at least for the experiments that support the main claims of the paper.
        \item The factors of variability that the error bars are capturing should be clearly stated (for example, train/test split, initialization, random drawing of some parameter, or overall run with given experimental conditions).
        \item The method for calculating the error bars should be explained (closed form formula, call to a library function, bootstrap, etc.)
        \item The assumptions made should be given (e.g., Normally distributed errors).
        \item It should be clear whether the error bar is the standard deviation or the standard error of the mean.
        \item It is OK to report 1-sigma error bars, but one should state it. The authors should preferably report a 2-sigma error bar than state that they have a 96\% CI, if the hypothesis of Normality of errors is not verified.
        \item For asymmetric distributions, the authors should be careful not to show in tables or figures symmetric error bars that would yield results that are out of range (e.g. negative error rates).
        \item If error bars are reported in tables or plots, The authors should explain in the text how they were calculated and reference the corresponding figures or tables in the text.
    \end{itemize}

\item {\bf Experiments compute resources}
    \item[] Question: For each experiment, does the paper provide sufficient information on the computer resources (type of compute workers, memory, time of execution) needed to reproduce the experiments?
    \item[] Answer: \answerTODO{} % Replace by \answerYes{}, \answerNo{}, or \answerNA{}.
    \item[] Justification: \justificationTODO{}
    \item[] Guidelines:
    \begin{itemize}
        \item The answer NA means that the paper does not include experiments.
        \item The paper should indicate the type of compute workers CPU or GPU, internal cluster, or cloud provider, including relevant memory and storage.
        \item The paper should provide the amount of compute required for each of the individual experimental runs as well as estimate the total compute.
        \item The paper should disclose whether the full research project required more compute than the experiments reported in the paper (e.g., preliminary or failed experiments that didn't make it into the paper).
    \end{itemize}

\item {\bf Code of ethics}
    \item[] Question: Does the research conducted in the paper conform, in every respect, with the NeurIPS Code of Ethics \url{https://neurips.cc/public/EthicsGuidelines}?
    \item[] Answer: \answerTODO{} % Replace by \answerYes{}, \answerNo{}, or \answerNA{}.
    \item[] Justification: \justificationTODO{}
    \item[] Guidelines:
    \begin{itemize}
        \item The answer NA means that the authors have not reviewed the NeurIPS Code of Ethics.
        \item If the authors answer No, they should explain the special circumstances that require a deviation from the Code of Ethics.
        \item The authors should make sure to preserve anonymity (e.g., if there is a special consideration due to laws or regulations in their jurisdiction).
    \end{itemize}


\item {\bf Broader impacts}
    \item[] Question: Does the paper discuss both potential positive societal impacts and negative societal impacts of the work performed?
    \item[] Answer: \answerTODO{} % Replace by \answerYes{}, \answerNo{}, or \answerNA{}.
    \item[] Justification: \justificationTODO{}
    \item[] Guidelines:
    \begin{itemize}
        \item The answer NA means that there is no societal impact of the work performed.
        \item If the authors answer NA or No, they should explain why their work has no societal impact or why the paper does not address societal impact.
        \item Examples of negative societal impacts include potential malicious or unintended uses (e.g., disinformation, generating fake profiles, surveillance), fairness considerations (e.g., deployment of technologies that could make decisions that unfairly impact specific groups), privacy considerations, and security considerations.
        \item The conference expects that many papers will be foundational research and not tied to particular applications, let alone deployments. However, if there is a direct path to any negative applications, the authors should point it out. For example, it is legitimate to point out that an improvement in the quality of generative models could be used to generate deepfakes for disinformation. On the other hand, it is not needed to point out that a generic algorithm for optimizing neural networks could enable people to train models that generate Deepfakes faster.
        \item The authors should consider possible harms that could arise when the technology is being used as intended and functioning correctly, harms that could arise when the technology is being used as intended but gives incorrect results, and harms following from (intentional or unintentional) misuse of the technology.
        \item If there are negative societal impacts, the authors could also discuss possible mitigation strategies (e.g., gated release of models, providing defenses in addition to attacks, mechanisms for monitoring misuse, mechanisms to monitor how a system learns from feedback over time, improving the efficiency and accessibility of ML).
    \end{itemize}

\item {\bf Safeguards}
    \item[] Question: Does the paper describe safeguards that have been put in place for responsible release of data or models that have a high risk for misuse (e.g., pretrained language models, image generators, or scraped datasets)?
    \item[] Answer: \answerTODO{} % Replace by \answerYes{}, \answerNo{}, or \answerNA{}.
    \item[] Justification: \justificationTODO{}
    \item[] Guidelines:
    \begin{itemize}
        \item The answer NA means that the paper poses no such risks.
        \item Released models that have a high risk for misuse or dual-use should be released with necessary safeguards to allow for controlled use of the model, for example by requiring that users adhere to usage guidelines or restrictions to access the model or implementing safety filters.
        \item Datasets that have been scraped from the Internet could pose safety risks. The authors should describe how they avoided releasing unsafe images.
        \item We recognize that providing effective safeguards is challenging, and many papers do not require this, but we encourage authors to take this into account and make a best faith effort.
    \end{itemize}

\item {\bf Licenses for existing assets}
    \item[] Question: Are the creators or original owners of assets (e.g., code, data, models), used in the paper, properly credited and are the license and terms of use explicitly mentioned and properly respected?
    \item[] Answer: \answerTODO{} % Replace by \answerYes{}, \answerNo{}, or \answerNA{}.
    \item[] Justification: \justificationTODO{}
    \item[] Guidelines:
    \begin{itemize}
        \item The answer NA means that the paper does not use existing assets.
        \item The authors should cite the original paper that produced the code package or dataset.
        \item The authors should state which version of the asset is used and, if possible, include a URL.
        \item The name of the license (e.g., CC-BY 4.0) should be included for each asset.
        \item For scraped data from a particular source (e.g., website), the copyright and terms of service of that source should be provided.
        \item If assets are released, the license, copyright information, and terms of use in the package should be provided. For popular datasets, \url{paperswithcode.com/datasets} has curated licenses for some datasets. Their licensing guide can help determine the license of a dataset.
        \item For existing datasets that are re-packaged, both the original license and the license of the derived asset (if it has changed) should be provided.
        \item If this information is not available online, the authors are encouraged to reach out to the asset's creators.
    \end{itemize}

\item {\bf New assets}
    \item[] Question: Are new assets introduced in the paper well documented and is the documentation provided alongside the assets?
    \item[] Answer: \answerTODO{} % Replace by \answerYes{}, \answerNo{}, or \answerNA{}.
    \item[] Justification: \justificationTODO{}
    \item[] Guidelines:
    \begin{itemize}
        \item The answer NA means that the paper does not release new assets.
        \item Researchers should communicate the details of the dataset/code/model as part of their submissions via structured templates. This includes details about training, license, limitations, etc.
        \item The paper should discuss whether and how consent was obtained from people whose asset is used.
        \item At submission time, remember to anonymize your assets (if applicable). You can either create an anonymized URL or include an anonymized zip file.
    \end{itemize}

\item {\bf Crowdsourcing and research with human subjects}
    \item[] Question: For crowdsourcing experiments and research with human subjects, does the paper include the full text of instructions given to participants and screenshots, if applicable, as well as details about compensation (if any)?
    \item[] Answer: \answerTODO{} % Replace by \answerYes{}, \answerNo{}, or \answerNA{}.
    \item[] Justification: \justificationTODO{}
    \item[] Guidelines:
    \begin{itemize}
        \item The answer NA means that the paper does not involve crowdsourcing nor research with human subjects.
        \item Including this information in the supplemental material is fine, but if the main contribution of the paper involves human subjects, then as much detail as possible should be included in the main paper.
        \item According to the NeurIPS Code of Ethics, workers involved in data collection, curation, or other labor should be paid at least the minimum wage in the country of the data collector.
    \end{itemize}

\item {\bf Institutional review board (IRB) approvals or equivalent for research with human subjects}
    \item[] Question: Does the paper describe potential risks incurred by study participants, whether such risks were disclosed to the subjects, and whether Institutional Review Board (IRB) approvals (or an equivalent approval/review based on the requirements of your country or institution) were obtained?
    \item[] Answer: \answerTODO{} % Replace by \answerYes{}, \answerNo{}, or \answerNA{}.
    \item[] Justification: \justificationTODO{}
    \item[] Guidelines:
    \begin{itemize}
        \item The answer NA means that the paper does not involve crowdsourcing nor research with human subjects.
        \item Depending on the country in which research is conducted, IRB approval (or equivalent) may be required for any human subjects research. If you obtained IRB approval, you should clearly state this in the paper.
        \item We recognize that the procedures for this may vary significantly between institutions and locations, and we expect authors to adhere to the NeurIPS Code of Ethics and the guidelines for their institution.
        \item For initial submissions, do not include any information that would break anonymity (if applicable), such as the institution conducting the review.
    \end{itemize}

\item {\bf Declaration of LLM usage}
    \item[] Question: Does the paper describe the usage of LLMs if it is an important, original, or non-standard component of the core methods in this research? Note that if the LLM is used only for writing, editing, or formatting purposes and does not impact the core methodology, scientific rigorousness, or originality of the research, declaration is not required.
    %this research?
    \item[] Answer: \answerTODO{} % Replace by \answerYes{}, \answerNo{}, or \answerNA{}.
    \item[] Justification: \justificationTODO{}
    \item[] Guidelines:
    \begin{itemize}
        \item The answer NA means that the core method development in this research does not involve LLMs as any important, original, or non-standard components.
        \item Please refer to our LLM policy (\url{https://neurips.cc/Conferences/2025/LLM}) for what should or should not be described.
    \end{itemize}

\end{enumerate}


\end{document}

%%% Local Variables:
%%% mode: LaTeX
%%% TeX-master: t
%%% End:
