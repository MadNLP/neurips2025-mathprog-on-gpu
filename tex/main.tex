\documentclass{article}

% if you need to pass options to natbib, use, e.g.:
%     \PassOptionsToPackage{numbers, compress}{natbib}
% before loading neurips_2025

% The authors should use one of these tracks.
% Before accepting by the NeurIPS conference, select one of the options below.
% 0. "default" for submission
\usepackage[preprint]{neurips_2025}
% the "default" option is equal to the "main" option, which is used for the Main Track with double-blind reviewing.
% 1. "main" option is used for the Main Track
%  \usepackage[main]{neurips_2025}
% 2. "position" option is used for the Position Paper Track
%  \usepackage[position]{neurips_2025}
% 3. "dandb" option is used for the Datasets & Benchmarks Track
 % \usepackage[dandb]{neurips_2025}
% 4. "creativeai" option is used for the Creative AI Track
%  \usepackage[creativeai]{neurips_2025}
% 5. "sglblindworkshop" option is used for the Workshop with single-blind reviewing
 % \usepackage[sglblindworkshop]{neurips_2025}
% 6. "dblblindworkshop" option is used for the Workshop with double-blind reviewing
%  \usepackage[dblblindworkshop]{neurips_2025}

% After being accepted, the authors should add "final" behind the track to compile a camera-ready version.
% 1. Main Track
 % \usepackage[main, final]{neurips_2025}
% 2. Position Paper Track
%  \usepackage[position, final]{neurips_2025}
% 3. Datasets & Benchmarks Track
 % \usepackage[dandb, final]{neurips_2025}
% 4. Creative AI Track
%  \usepackage[creativeai, final]{neurips_2025}
% 5. Workshop with single-blind reviewing
%  \usepackage[sglblindworkshop, final]{neurips_2025}
% 6. Workshop with double-blind reviewing
%  \usepackage[dblblindworkshop, final]{neurips_2025}
% Note. For the workshop paper template, both \title{} and \workshoptitle{} are required, with the former indicating the paper title shown in the title and the latter indicating the workshop title displayed in the footnote.
% For workshops (5., 6.), the authors should add the name of the workshop, "\workshoptitle" command is used to set the workshop title.
% \workshoptitle{WORKSHOP TITLE}

% "preprint" option is used for arXiv or other preprint submissions
 % \usepackage[preprint]{neurips_2025}

% to avoid loading the natbib package, add option nonatbib:
% \usepackage[nonatbib]{neurips_2025}


\usepackage[utf8]{inputenc} % allow utf-8 input
\usepackage[T1]{fontenc}    % use 8-bit T1 fonts
\usepackage{hyperref}       % hyperlinks
\usepackage{url}            % simple URL typesetting
\usepackage{booktabs}       % professional-quality tables
\usepackage{amsfonts}       % blackboard math symbols
\usepackage{nicefrac}       % compact symbols for 1/2, etc.
\usepackage{microtype}      % microtypography
\usepackage{xcolor}         % colors
\usepackage{glossaries}
\usepackage{enumitem}
\usepackage{listings}
\lstset{basicstyle=\ttfamily\small}
\usepackage{multirow}
\usepackage{cleveref}

\newacronym{pde}{PDE}{partial differential equation}
\newacronym{ad}{AD}{automated differentiation}
\newacronym{alcf}{ALCF}{Argonne Leadership Computing Facility}
\newacronym{blas}{BLAS}{basic linear algebra subprograms}
\newacronym{iap}{IAP}{independent activities period}
\newacronym{der}{DER}{distributed energy resource}
\newacronym{derms}{DERMS}{distributed energy resource management system}
\newacronym{minlp}{MINLP}{mixed-integer nonlinear programming}
\newacronym{nlp}{NLP}{nonlinear programming}
\newacronym{kkt}{KKT}{Karush-Kuhn-Tucker}
\newacronym{sqp}{SQP}{sequential quadratic programming}
\newacronym{ipm}{IPM}{interior-point method}
\newacronym{cpu}{CPU}{central processing units}
\newacronym{gpu}{GPU}{graphics processing units}
\newacronym{mpc}{MPC}{model predictive control}
\newacronym{ac}{AC}{alternating current}
\newacronym{dc}{DC}{direct current}
\newacronym{opf}{OPF}{optimal power flow}
\newacronym{hpc}{HPC}{high-performance computing}
\newacronym{pcg}{PCG}{projected conjugate gradient}
\newacronym{alm}{ALM}{augmented Lagrangian method}
\newacronym{dac}{DAC}{direct air capture}
\newacronym{pem}{PEM}{proton exchange membrane}
\newacronym{tea}{TEA}{technoeconomic analysis}
\newacronym{lca}{LCA}{life cycle assessment}
\newacronym{ft}{FT}{Fischer-Tropsch}
\newacronym{bess}{BESS}{battery energy storage system}
\newacronym{cqa}{CQA}{critical quality attribute}
\newacronym{lnp}{LNP}{lipid nanoparticle}
\newacronym{iso}{ISO}{independent system operator}
\newacronym{simd}{SIMD}{single instruction, multiple data}
\newacronym{mimd}{MIMD}{multiple instruction, multiple data}
\newacronym{orcd}{ORCD}{Office of Research Computing and Data}
\newacronym{mitei}{MITei}{MIT Energy Initiative}
\newacronym{hipsat}{HIP-SAT}{High School Introduction to the Physical Sciences and Advanced Technologies}
\newacronym{mpi}{MPI}{message passing interface}
\newacronym{goc}{GOC}{grid optimization competition}
\newacronym{sc}{SC}{security-constrained}
\newacronym{mp}{MP}{multi-period}
\newacronym{admm}{ADMM}{alternating direction method of multipliers}
\newacronym{gmres}{GMRES}{generalized mean residual}
\newacronym{pdhg}{PDHG}{primal-dual hybrid gradient}
\newacronym{lp}{LP}{linear programming}
\newacronym{ams}{AMS}{algebraic modeling system}
\newacronym{ncl}{NCL}{nonlinear constrained Lagrangian}
\newacronym{bcl}{BCL}{bound constrained Lagrangian}


% Note. For the workshop paper template, both \title{} and \workshoptitle{} are required, with the former indicating the paper title shown in the title and the latter indicating the workshop title displayed in the footnote.
\title{On the GPU Implementation of Large-Scale Second-Order Optimization Solvers}


% The \author macro works with any number of authors. There are two commands
% used to separate the names and addresses of multiple authors: \And and \AND.
%
% Using \And between authors leaves it to LaTeX to determine where to break the
% lines. Using \AND forces a line break at that point. So, if LaTeX puts 3 of 4
% authors names on the first line, and the last on the second line, try using
% \AND instead of \And before the third author name.


\author{%
  Alexis Montoison\\
  Mathematics and Computer Science Division\\
  Argonne National Laboratory\\
  Lemont, IL 60439\\
  \texttt{montoison@anl.gov}\\
  \And
  Fran\c{c}ois Pacaud\\
  Centre Automatique et Systèmes\\
  Mines Paris-PSL\\
  Paris, 75006 \\
  \texttt{francois.pacaud@minesparis.psl.eu}\\
  \And
  Sungho Shin\\
  Department of Chemical Engineering\\
  Massachusetts Institute of Technology\\
  Cambridge, MA 02139\\
  \texttt{sushin@mit.edu}\\
  \And
  Mihai Anitescu\\
  Mathematics and Computer Science Division\\
  Argonne National Laboratory\\
  Lemont, IL 60439\\
  \texttt{anitescu@mcs.anl.gov}\\
  % examples of more authors
  % Coauthor \\
  % Affiliation \\
  % Address \\
  % \texttt{email} \\
  % \AND
  % Coauthor \\
  % Affiliation \\
  % Address \\
  % \texttt{email} \\
  % \And
  % Coauthor \\
  % Affiliation \\
  % Address \\
  % \texttt{email} \\
  % \And
  % Coauthor \\
  % Affiliation \\
  % Address \\
  % \texttt{email} \\
}


\begin{document}


\maketitle


\begin{abstract}
In recent years, GPU-accelerated second-order (e.g., interior-point) optimization solvers have gained momentum with the advent of mature and efficient GPU-accelerated direct linear solvers such as cuDSS, though several challenges remain. This paper summarizes the current state of the art and open research questions in the GPU implementation of large-scale second-order optimization solvers, focusing on linear and nonlinear programming (local solvers only). Specifically, we first summarize the current state-of-the-art in GPU-accelerated direct linear solvers, focusing on their use within mathematical programming solvers. Then, we discuss different aspects of various KKT system formulations that can be employed within interior-point methods and examine their suitability for GPU implementation. We also discuss strategies for implementing sparse automatic differentiation on GPUs, which is crucial for nonlinear programming. Finally, we present numerical results highlighting scalability---often achieving more than 10x speed-ups over the best existing CPU alternatives---and the current limitations of the solvers available within MadSuite, a GPU-accelerated mathematical programming software suite. The paper is concluded by discussing several open questions.
\end{abstract}


\section{Introduction}\label{eqn:intro}

%% SS: Sungho will write this section

This paper focuses on the continuous optimization problems of the following form:
\begin{align}\label{eqn:opt}
  \min_{x } \; & f(x) \quad \text{s.t.} \; g(x) \geq 0 \; ,
\end{align}
where $x \in \mathbb{R}^n$ is the optimization decision variable, $f: \mathbb{R}^n \to \mathbb{R}$ is the objective function, and $g: \mathbb{R}^n \to \mathbb{R}^m$ is the constraint function.
For the remainder of the paper, we assume that $f$ and $g$ are sufficiently smooth.
We discuss both the situation where $f$ and $g$ are affine, which corresponds to \gls{lp}, and the situation where $f$ and $g$ are nonlinear (and potentially nonconvex), which corresponds to \gls{nlp}. In the nonconvex setting, we only focus on local solution methods.
Further, we particularly focus on the implementations targeting large and \emph{sparse} problems, where the constraint Jacobian and Lagrangian Hessian are large (e.g., the number of row and columns exceeds one million) and sparse matrices (e.g., tens of non-zero entries per row).
These problems arise in various applications, including power systems \cite{}, chemical process optimization \cite{}, autonomous control systems \cite{}, and many others \cite{}; these applications often require solving large-scale optimization problems in real-time up to high accuracy to ensure constraint satisfaction and optimal performance.
Existing solvers include Ipopt, Knitro, HiGHS, Gurobi, etc, and they are all designed to exploit the inherent sparsity to cope with the large problem sizes.

While general-purpose GPU computing has made significant progress in recent years, the GPU computing techniques were not actively adopted by the state-of-the-art optimization solvers.
GPU computing is designed to be most powerful when performing highly repetitive operations on large data sets, due to the SIMD (single instruction, multiple data) architecture of GPUs.
Matrix multiplications in particular, is performed on a separate hardware unit, called Tensor cores in NVIDIA's architecture, which is further optimized for performing multiply-accumulate operations on large matrices.
When they are properly applied, GPUs can provide significant accelerations over the serial CPU implementations due to their massive parallelization capabilities.
For example, for large AI model training, the GPU architectures provide the main computational power for performing dense linear algebra operations at the low-level, thus enabling the training of AI models with massive number of parameters.
However, many of the problem formulations that have been targeted by the classical mathematical programming solvers are extremely sparse in nature, and they cannot adequately exploit the parallelization at the dense matrix operation level.
Furthermore, even if some part of the algorithm can benefit from GPU acceleration, the acceleration cannot be applied in an ad-hoc fashion because if the other part of the algorithm runs on the CPU, a lot of host-device data transfer is required, which can be a bottleneck for the overall performance.
As such, GPU utilization within the state-of-the-art mathematical programming solvers is (i) inherently more challenging, (ii) may not be even suitable for certain class of problems, and (iii) often requires careful redesign the algorithms to allow effective utilization of parallel computing resources.

First-order algorithms have emerged as a suitable algorithm class for GPU implementation, as they rely typically on sparse matrix-vector multiplication (SpMV) and vector operations, which can be efficiently parallelized on GPUs.
Notably, several successful implementations have been reported in the recent years, such as cuPDLP \cite{}, cuPDLP+ \cite{}, and cuOSQP \cite{}.
Although these approaches are highly promising when targeting low or mid-precision solutions, it is difficult to achieve high-precision solutions with these methods, due to the inherent linear convergence of the first-order methods.
Further, the performance advantage can be observed only when the problem size is large enough, which further limits the applicability of these methods.

The limitations of the first-order algorithms motivates the development of second-order solvers on GPUs.
As the second-order algorithms are predominantly used methods in the state-of-the-art solvers, when implemented appropriately, they can achieve similarly high precision solutions and be effective for a wide range of problem sizes.
However, implementing second-order solvers on GPUs is inherently more challenging than first-order methods, as they rely on more sophisticated low-level linear algebra subroutines.
In particular, at each iteration of the second-order methods it is required to solve a linear system, called \gls{kkt} system.
As the interior-point (barrier) method is typically employed to handle inequalities in a scalable fashion, these KKT systems are extremely ill-conditioned, as they approach to the solution, which hampers iterative linear algebra approaches even with preconditioning.
Furthermore, in nonconvex settings, the KKT systems do not exhibit any nice properties, such as positive quasi-definiteness, which makes ensuring the numerical stability of the linear solvers even more challenging.

Recently, NVIDIA has released cuDSS, a direct linear solver library designed to handle large sparse linear systems on GPUs, which was predated by cuSOLVERRF or the undocumented sparse Cholesky routines shipped with CUDA.
cuDSS provides mature implementation of direct solvers based on Cholesky, LDL$^\top$, and LU sparse factorization, which are the most commonly used sparse factorization methods in the state-of-the-art mathematical programming solvers, with unprecedented speed and numerical stability.
Though they do not provide the entire toolset required to exactly port the existing IPM solvers to GPUs, they currently provide the adequate toolset to implement a certain modified, parallel-friendly version of the optimization algorithms.
Several open-source solvers are already incorporating cuDSS, such as MadNLP \cite{}, a GPU-accelerated nonlinear programming solver.
When properly configured with GPU-accelerated automatic differentiation system, this solver is known to achieve remarkable performance on large-scale nonlinear programming problems, achieving more than 10x speed-up over the best existing CPU alternatives \cite{}.

In this context, it is worthwhile to summarize the current state of the art in GPU-accelerated second-order optimization solvers, focusing on the current capabilities of cuDSS and its use within mathematical programming solvers.
This paper aims to provide a comprehensive overview of the current state of the art in GPU-accelerated second-order optimization solvers, focusing on linear and nonlinear programming.


Below we summarize the key focus of the paper and rationale:
\begin{itemize}[leftmargin=*,itemsep=0pt,parsep=0pt,partopsep=0pt]
\item We base our discussion on the interior-point method as the measure to handle inequality constraints. An alternative approach, the active-set based methods (or simplex methods for \glspl*{lp}), are not considered in this paper, as they are not typically used in large-scale optimization problems.
\item We focus on the solution of the KKT systems within the interior-point method, which is the most non-trivial part of the GPU implementation. Although second-order based mathematical programming solvers rely on various low-level kernels and sophisticated strategies to ensure robust performance, most of the computations needed can be implemented with rather simple \texttt{map} or \texttt{reduce} operations, which can be straightforwardly executed on GPUs.
\item We focus on NVIDIA GPUs and their software stack, as they are the most widely used GPUs in the high-performance computing community. While we expect that the same capabilities in principle can be implemented on other GPU architectures, such as AMD or Intel GPUs, or in a cross-platform fashion, such as OpenCL or KernelAbstractions.jl, NVIDIA currently provides the most mature sparse direct sovler library, cuDSS.
\end{itemize}
In the remainder of the paper, we first summarize the current state-of-the-art in GPU-accelerated direct linear solvers, focusing on their use within mathematical programming solvers.
Then, we discuss different aspects of various KKT system formulations that can be employed within interior-point methods and examine their suitability for GPU implementation.
We also discuss strategies for implementing sparse automatic differentiation on GPUs, which is crucial for nonlinear programming.
Finally, we present numerical results highlighting scalability---often achieving more than 10x speed-ups over the best existing CPU alternatives---and the current limitations of the solvers available within MadSuite, a GPU-accelerated mathematical programming software suite.

\section{Direct Linear Solvers for Optimization Sovlers}\label{eqn:linear}
%% SS: Alexis, could you write  this section?
The most commonly used direct linear solvers in mathematical programming solvers are based on sparse matrix factorization methods.
This section summarizes the current state-of-the-art in the GPU implementations of direct solvers, particularly focusing on the current capabilities of cuDSS, NVIDIA's direct solver library.
% Although GPU solvers has made significant progress in the recent years, their numerical stability, especially for handling indefinite linear systems is still rather limited.



\subsection{Cholesky factorization}
Cholesky factorization is a matrix factorization method that decomposes a positive definite matrix $A$ into the product of a lower triangular matrix $L$ and its transpose, i.e., $A = LL^\top$. For sparse matrices, a proper ordering of the matrix is required to minimize fill-in and to ensure the parallelizability at the elimination tree level. By also incorporating the permutation, the factorization can be written as $A = P L L^\top P^\top$, where $P$ is a permutation matrix.
Within direct solvers, Cholesky factorization is utilized to solve linear systems of the form $Ax = b$, where the solution is obtained by first solving the lower triangular system $Ly = Pb$ and then the upper triangular system $L^\top x = y$.

Although algorithms to compute $P$ are serial in nature (e.g., minimum degree ordering or nested dissection) and unsuitable for the implementation on GPUs (e.g., cuDSS performs this operation on CPUs), the ordering can be computed only once and can be reused for multiple factorization calls, thus the overhead can be amortized over multiple calls.

A nice property of Cholesky factorization is that it can be computed in a numerically stable manner, provided that the matrix $A$ is positive definite.
Since performing numerical pivoting on GPUs is known to be challenging and inefficient, it is always desired to transform a matrix in a Cholesky factorizable form to apply direct solution methods on GPUs.
Cholesky factorization exists only if the original matrix $A$ is positive definite.

% Conversely, if the matrix is not positive definite, the Cholesky factorization will fail, and the algorithm will encounter a zero pivot.
% Thus, the failure of the Cholesky factorization can be used by the optimization sovler as a signal that the primal regularization needs to be increased to ensure the descent direction of the Newton's step.


\subsection{LDL$^\top$ factorization}
If Cholesky factorization is not an option, one may resort to LDL$^\top$ factorization, often referred to as modified or signed Cholesky factorization.
This method decomposes a matrix $A$ into the product of a lower triangular matrix $L$, a diagonal matrix $D$, and the transpose of $L$, i.e., $A = L D L^\top$ (or $A = P L D L^\top P^\top$ with a permutation matrix $P$).
The algorithm is almost the same as the Cholesky factorization, except that the diagonal matrix $D$ can contain both positive and negative entries, which allows the factorization to be applied to indefinite matrices.
LDL$^\top$ factorization exists for more broad class of matrices, including many indefinite matrices.
However, they are guaranteed to be stable with any given ordering (thus not requiring numerical pivoting) only when the matrix is symmetric quasi-definite.
Many of the saddle point systems encountered in mathematical programming solvers are positive quasi-definite or close to be positive quasi-definite (can be come quasi-definite with infinitesimal regularization), and thus, LDL$^\top$ factorization is often used to solve the KKT systems in the interior-point method.

However, for general indefinite matrices, LDL$^\top$ factorization procedure is not guaranteed to be numerically stable and they can encounter zero pivot.
Thus, various mitigation strategies are employed in the state-of-the-art direct linear solvers.
Below we summarize the existing strategies and their potential use within GPU solvers.
\paragraph{Partial Pivoting}
\paragraph{Pivot Perturbing}
\paragraph{Iterative Refinement}


\subsection{LBL$^\top$ factorization}

% This factorization method is guaranteed to be numerically stable when it is applied to symmetric quasi-definite matrices, and thus, is particularly suitable for factorizing the KKT system matrices encountered within the interior-point method for convex optimization problems.



% In practice, the KKT systems

% some basic description of LDLT factorization algorihthm, highlighting their usage within convex solvers.

% \subsection{Strategies to Mitigate Numerical Failures}



% discuss the strategies to mitigate the numerical failure; partial pivoting, LBL^T, pivot perturbing, iterative refinement
% \subsection{Strategies to Mitigate Numerical Failures}



\section{Interior-Point (Barrier) Methods and KKT Systems}\label{eqn:ipm}
The \gls*{ipm} is a class of optimization algorithms that are designed to solve inequality-constrained optimization problems, such as \cref{eqn:opt}, by transforming the problem into a sequence of unconstrained optimization problems using a barrier function.

In most of practical implementations, slack variables are introduced as follows:
\begin{align}\label{eqn:barrier}
  \min_{x,s } \; & f(x) - \mu \sum_{i=1}^m \log(s_i) \quad \text{s.t.} \; g(x) - s = 0 \;.
\end{align}
Here, $s \in \mathbb{R}^m$ is the slack variable, and $\mu > 0$ is the barrier parameter that is gradually reduced to zero as the algorithm progresses.
The KKT conditions for the problem \cref{eqn:barrier} are given by:
\begin{align}\label{eqn:kkt}
  \nabla f(x) - \nabla g(x)^\top \lambda = 0, \quad
  \Lambda S e = \mu e,\quad
  g(x) - s =   0 ,
\end{align}
where $\lambda \in \mathbb{R}^m$ is the Lagrange multiplier, $S = \text{diag}(s)$
and $\Lambda = \text{diag}(\lambda)$ are the diagonal matrices of slack variables and
multipliers respectively, and $e$ is the vector of ones.

\subsection{KKT systems}
\Gls*{ipm} solves \cref{eqn:kkt} for $(x, \lambda, s)$ iteratively using Newton method. At each iteration, we obtain the
search direction by solving the symmetric KKT system:
\begin{align}\label{eqn:kkt_system}
  \begin{bmatrix}
    \nabla^2_{x x} \mathcal{L}(x,\lambda) && \nabla g(x)^\top  \\
                                          & S^{-1}\Lambda & -I \\
    \nabla g(x) & -I & \\
  \end{bmatrix}
  \begin{bmatrix}
    \phantom{-}d_x\\
    \phantom{-}d_s \\
    -d_\lambda
  \end{bmatrix} =
  -\begin{bmatrix}
    \nabla f(x) - \nabla g(x)^\top \lambda\\
    \Lambda e - \mu S^{-1} e \\
    g(x) - s\\
  \end{bmatrix},
\end{align}
Here,
$\mathcal{L}(x,\lambda,s)  := f(x) - \lambda^\top g(x)$ is the Lagrangian function.
By removing $d_s = \Lambda^{-1} \left(\mu e - S d_\lambda - S\Lambda e\right)$, the system
\eqref{eqn:kkt_system} is equivalent to the following $2\times 2$ KKT system:
\begin{equation}\label{eqn:augmentedKKT}
  \begin{bmatrix}
    \nabla^2_{x x} \mathcal{L}(x,\lambda) + \delta_w I & \nabla g(x)^\top \\
    \nabla g(x) &  - \delta_c I - \Lambda^{-1} S
  \end{bmatrix}
  \begin{bmatrix}
    \phantom{-}d_x\\
    - d_\lambda
  \end{bmatrix} =
  -\begin{bmatrix}
    \nabla f(x) - \nabla g(x)^\top \lambda\\
    g(x) - \mu \Lambda^{-1} e
  \end{bmatrix},
\end{equation}
Note that we have introduced in \eqref{eqn:augmentedKKT} regularization parameters for the primal variables
$\delta_w\geq 0$ and for the dual variables $\delta_c\geq 0$. We will detail
in the following section how to tune the regularization parameters depending on
the class of problems we are solving.

Since the block structure in \cref{eqn:augmentedKKT} admits several possibilities for further reduction, the KKT system can be condensed in several different ways.

% \paragraph{Eliminating slack variables}
% $S^{-1}\Lambda + \delta_w I$ is always invertible due to the nature of primal-dual interior-point method, as the slack $s$ and the dual variable $\lambda$ are always forced to be positive. The resulting system can be written as follows:
% \begin{align}\label{eqn:kkt_2}
%   \begin{bmatrix}
%     \nabla^2 \mathcal{L}(x,\lambda) + \delta_w I&  \nabla g(x)^\top \\
%     \nabla g(x) &  - \delta_c I - (S^{-1}\Lambda + \delta_w I)^{-1}\\
%   \end{bmatrix}
%   \begin{bmatrix}
%     *\\
%     *\\
%   \end{bmatrix} =
%   -\begin{bmatrix}
%     *\\
%     *\\
%   \end{bmatrix},
% \end{align}
% Since $S^{-1}\Lambda + \delta_w I$ is diagonal, the elimination procedure does not incur significant computational overhead and the number of non-zero entries in the resulting system is not increased. However, since $S^{-1}\Lambda$ can be arbitrarily ill-conditioned, the numerical stability of the resulting system can be compromised. This system is positive quasi-definite if the original problem is strongly convex (thus LDL$^\top$ factorizable), but for general linear programs, the $(1,1)$-block is not guaranteed to be positive definite. For general \glspl*{nlp}, even the $(1,1)$-block can have negative eigenvalues, and thus, LBL$^\top$ method with partial pivoting needs to be employed to solve the in a numerically stable fashion.

\paragraph{Primal condensed system.}
The 2x2 block $ \delta_c I+ \Lambda^{-1}S $ is always invertible given the structure. The
\emph{primal condensed KKT system} can be written as follows:
\begin{align}\label{eqn:kkt_primal}
  % \left(\nabla^2 \mathcal{L}(x,\lambda) + \delta_w I + \nabla g(x)^\top \left(\delta_c I + (S^{-1}\Lambda + \delta_w I)^{-1}\right) \nabla g(x)\right)
  % \begin{bmatrix}
  %   *\\
  % \end{bmatrix} =
  % -\begin{bmatrix}
  %   *\\
  % \end{bmatrix}.
  \left(\nabla^2_{x x} \mathcal{L}(x, \lambda) + \delta_w I + \nabla g(x)^\top (\delta_c I + \Lambda^{-1} S)^{-1} \nabla g(x)  \right) d_x = - r_p \; ,
\end{align}
with $r_p$ an appropriate right-hand-side.
Compared to \eqref{eqn:augmentedKKT}, the system size is further reduced.
However, since the Jacobian $\nabla g(x)$ can have dense rows, the condensed system can become arbitrarily dense, which can significantly increase the computational cost of the linear solver. Furthermore, this elimination strategy suffers from the same potential ill-conditioning issue as in \eqref{eqn:augmentedKKT}.
However, this system is always positive definite for appropriate choice of the
regularization $(\delta_w, \delta_p)$.
Thus, the system can always be Cholesky factorized with static pivoting, even for nonlinear programs.

\paragraph{Dual condensed system.}
This is less common, but the primal 1x1 block $\nabla^2 \mathcal{L}(x,\lambda) + \delta_w I$ is invertible either if the problem is strongly convex or if the default regularization parameter $\delta_w$ is strictly positive.
By eliminating the first block of rows, we obtain the \emph{dual condensed KKT system}:
\begin{align}\label{eqn:kkt_dual}
  \left(\delta_c I + \Lambda^{-1}S + \nabla g(x)\left(\nabla_{x x}^2 \mathcal{L}(x,\lambda) + \delta_w I\right)^{-1} \nabla g(x)^\top\right)
  d_\lambda = - r_d \; ,
  % \begin{bmatrix}
  %   *\\
  % \end{bmatrix} =
  % -\begin{bmatrix}
  %   *\\
  % \end{bmatrix},
\end{align}
with $r_d$ an appropriate right-hand-side.
In the context of linear programming, this system is often called \emph{normal equations}.
Unless $\nabla^2 \mathcal{L}(x,\lambda) + \delta_w I$ is diagonal, this system can be arbitrarily dense, and thus, this elimination needs to be used with caution. Assuming that the primal Hessian is positive definite, this system is always positive definite, thus is Cholesky factorizable.


\subsection{Regularization parameters}
As seen before, the algorithm tunes the regularization parameters $\delta_w$
and $\delta_w$ to ensure the system \eqref{eqn:augmentedKKT} is well-posed.

\paragraph{Convex programming.}

If the problem~\eqref{eqn:opt} is convex, the system \eqref{eqn:augmentedKKT}
becomes SQD for all $(\delta_x, \delta_c) > 0$. By setting
$\delta_w = \delta_c = \varepsilon$, we obtain a strongly factorizable regularized KKT system,
easing the solution of \eqref{eqn:augmentedKKT}.
This idea has lead to several robust \gls{ipm} implementations~\cite{}.
This is the approach adopted in MadIPM.

For linear program, the dual condensed KKT system~\eqref{eqn:kkt_dual} is often used
by default as the diagonal block $\nabla^2_{x x} \mathcal{L}(x, \lambda) + \delta_w I$ is easy
to invert in that case. Customized implementation of the Cholesky factorization
are used to handle dense columns in the Jacobian $\nabla g(x)$ \cite{}.

For general quadratic programs, \cref{eqn:kkt_primal} is most suitable, though other formulations, such as \cref{eqn:kkt_system,eqn:augmentedKKT,eqn:kkt_dual} can also be used when proper regularization, pivot perturbation, and/or iterative refinement strategies are employed.

For strongly convex programs, all four formulations \cref{eqn:kkt_system,eqn:augmentedKKT,eqn:kkt_dual,eqn:kkt_primal} are appropriate, so one may choose the best one based on the sparsity pattern and degree of ill-conditioning of the KKT system matrix.


\paragraph{Nonlinear programming.}
The case of nonlinear programming is different, as the solver has to accomodate
the local nonconvexities in the model.

Traditional implementations of \gls{ipm} for nonlinear programming use an \emph{inertia correction} mechanism.
The regularization parameters $(\delta_w, \delta_c)$ are increased until the number of positive, negative, and zero eigenvalues (so-called inertia, available as a byproduct of the LDL$^\top$ and LBL$^\top$ factorizations) of the KKT system matrix in \eqref{eqn:kkt_system} is equal to $(n+m, m, 0)$.

For GPU solvers for \glspl*{nlp}, \cref{eqn:kkt_primal} is the most suitable formulation as the other formulations are bound to exhibit numerical instabilities.
Notably, this is approach used in MadNLP.jl.
If the problem~\eqref{eqn:opt} has additional equality constraints,
the equality constraints are relaxed into doubly bounded inequalities with a small gap, called LiftedKKT system strategy.


%% SS: Francois, could you lead writing this section? I can also take a pass.


\section{Algebraic Modeling Systems and Automatic Differentiation}\label{eqn:ad}
For nonlinear programming, the evaluation of the functions $f$ and $g$ and their derivatives is crucial for the performance of the optimization solvers.
In most modern optimization workflows, the derivative evaluation code (or interpreter) is generated in a fully automated fashion via the use of algebraic modeling systems designed for mathematical programming.
It is of interest to develop an algebraic modeling system that can generate the code for evaluating and differentiating the functions $f$ and $g$ on GPUs.
This is not only because they are major contributors to the overall computational cost, but also because if the derivatives have to be evaluated on CPUs, the host-device data transfer can become a bottleneck for the overall performance.

The parallelization of derivative evaluation requires exploiting the internal structure of the functions $f$ and $g$, as the function oracles by themselves do not permit performing parallelization.
In particular, functions $f$ and $g$ observed within many practical instances of the large-scale sparse mathematical programs exhibit highly repetitive structure that can be exploited by GPUs.
For example, like the following example:
\begin{align*}
  f(x) &= \sum_{p\in P} f^o(x; p), \quad g(x) = \left\{g^o(x; p)\right\}_{p\in P},
\end{align*}
$f$ may be a sum of many terms, each of which is a simple expressions, such as affine, polynomial, or simple nonlinear functions, and $g$ may be a collection of many constraints generated from a common template, such as power flow equations applied to many branches of a power network or material and energy balance equations applied to many streams and units in a chemical process. In such cases, the evaluation of the functions $f$ and $g$ and their derivatives is embarassingly parallelizable, and they are suitable for executing on GPUs.

Moreover, one can even develop an algebraic modeling system that generates the code for evaluating and differentiating the functions $f$ and $g$ in a highly parallelizable manner, as long as the repetive structure is informed to the modeling system. In order to implement such a capabilities, it is necessary to design a modeling syntax that allows the user to specify the repetitive structures within $f$ and $g$. The classical modeling systems, such as AMPL, GAMS, CasADi, and JuMP, do not provide such capabilities, but more recently developed systems, such as ExaModels.jl and Gravity, provide such capabilities. For example, ExaModels.jl requires the user to specify the objective and constraint functions always in a form of iterator, such as
\begin{lstlisting}
  objective(c, 100 * (x[i-1]^2 - x[i])^2 + (x[i-1] - 1)^2 for i = 2:N)
\end{lstlisting}
which corresponds to the case of $f^o(x; p) = 100(x[p-1]^2 - x[p])^2 + (x[p-1]-1)^2$ and $p=\{2,\cdots,N\}$.
This syntax allows the modeling system to store the model function information in a structured fashion and generate the code for evaluating and differentiating (via reverse mode \gls*{ad}) the functions $f$ and $g$ in a highly parallelizable manner.




\section{Numerical Results}\label{eqn:num}
This section provides performance benchmark results for the GPU solvers against the state-of-the-art CPU solvers.
Publicly available benchmark instances, such as MIPLIB, pglib-opf, and COPS, are used to evaluate the performance of the solvers.
When reporting the solve time for multiple instances, we represent them by shifted geometric mean: $\left(\prod_{i=1}^n (t_i + \Delta)\right)^{1/n} - \Delta$, where $t_i$ is the solve time for the $i$-th instance and $\Delta$ is the shift factor. We denote this metric with $\Delta = 10$ as SGM10. If an instance is unsolved, its solving time is set to the corresponding time limit.
The benchmark was performed on a workstation with two Intel(R) Xeon(R) Platinum 8160 CPU @ 2.10GHz, two Quadro GV 100 GPUs, and 384 GB of RAM.
The following versions of the software are used within the benchmark: Julia v1.11.6, MadNLP v0.8.8, ExaModels v0.9.0, CUDSS v0.6.0, HiGHS v1.7.0, and Gurobi v10.0.0.
The numerical results can be reproduced with the source code and Manifest file available at \url{https://github.com/MadNLP/neurips2025-mathprog-on-gpu}.



\subsection{Linear Programming}
\paragraph{MIPLIB}
\begin{table}[t]
  \centering\small
  \caption{Solve time in seconds and SGM10 on instances of MIPLIB without presolve}
  \begin{tabular}{|c|c|cc|cc|cc|cc|}
    \hline
    \multirow{ 3}{*}{Tol} & \multirow{ 3}{*}{Solver} & \multicolumn{2}{c|}{\textbf{Small} (269)}& \multicolumn{2}{c|}{\textbf{Medium} (94)}& \multicolumn{2}{c|}{\textbf{Large} (20)}& \multicolumn{2}{c|}{\multirow{2}{*}{\textbf{Total} (383)}}\\
                          && \multicolumn{2}{c|}{(1 hr max)}& \multicolumn{2}{c|}{(1 hr max)}& \multicolumn{2}{c|}{(5 hr max)}&&\\
                          &&  Solved & Time &  Solved & Time &  Solved & Time &  Solved & Time \\
    \hline
    \multirow{3}{*}{$10^{-4}$} & MadIPM & 66 & 0.2 & 66 & 0.2 & 66 & 0.2 & 66 & 0.2  \\
                          & HiGHS & 66 & 0.2 & 66 & 0.2 & 66 & 0.2 & 66 & 0.2 \\
                          & Gurobi & 66 & 0.2 & 66 & 0.2 & 66 & 0.2 & 66 & 0.2 \\
    \hline
    \multirow{3}{*}{$10^{-8}$} & MadIPM & 66 & 0.2 & 66 & 0.2 & 66 & 0.2& 66 & 0.2 \\
                          & HiGHS & 66 & 0.2 & 66 & 0.2 & 66 & 0.2& 66 & 0.2 \\
                          & Gurobi & 66 & 0.2 & 66 & 0.2 & 66 & 0.2& 66 & 0.2 \\
    \hline
  \end{tabular}
\end{table}

\subsection{Nonlinear Programming}
In this benchmark, we compare the performance of MadNLP running on GPU and Ipopt running on CPU. For both solvers, the \gls*{nlp} function callbacks are provided by ExaModels.jl (either running on GPU or CPU). MadNLP is configured with cuDSS as the linear solver, while Ipopt is configured with Ma27 as the linear solver. Ipopt+ExaModels+Ma27 is currently the best known configuration for solving large, sparse nonlinear programming problems on a single-threaded CPU, and it is used as the baseline for comparison. The solver performance is benchmarked against the pglib-opf benchmark suite \cite{} and the COPS benchmark suite \cite{}. CUTEst \cite{} is excluded from the benchmark, as the library mostly contains small- to medium-sized nonlinear programming problems, which are not suitable for benchmarking the performance of GPU-accelerated solvers, and currently there is no ExaModels.jl implementation of CUTEst.

\paragraph{AC Optimal Power Flow}
Depending on the number of nonzeros in the augmented KKT system, we categorize the instances into small, medium, and large. The small instances have less than 1 million nonzeros, the medium instances have between 1 million and 10 million nonzeros, and the large instances have more than 10 million nonzeros. The results are summarized in Table \ref{tab:pglib-opf}, and the raw data are available in the paper repository.

\begin{table}[t]
  \centering\small
  \caption{Solve time in seconds and SGM10 on instances of pglib-opf}\label{tab:pglib-opf}
  \begin{tabular}{|c|c|cc|cc|cc|cc|}
    \hline
    \multirow{ 3}{*}{Tol} & \multirow{ 3}{*}{Solver} & \multicolumn{2}{c|}{\textbf{Small} (269)}& \multicolumn{2}{c|}{\textbf{Medium} (94)}& \multicolumn{2}{c|}{\textbf{Large} (20)}& \multicolumn{2}{c|}{\multirow{2}{*}{\textbf{Total} (383)}}\\
                          && \multicolumn{2}{c|}{(1 hr max)}& \multicolumn{2}{c|}{(1 hr max)}& \multicolumn{2}{c|}{(5 hr max)}&&\\
                          &&  Solved & Time &  Solved & Time &  Solved & Time &  Solved & Time \\
    \hline
    \multirow{2}{*}{$10^{-4}$} & MadNLP (gpu) & 66 & 0.2 & 66 & 0.2 & 66 & 0.2 & 66 & 0.2  \\
                          & Ipopt (cpu) & 66 & 0.2 & 66 & 0.2 & 66 & 0.2 & 66 & 0.2 \\
    \hline
    \multirow{2}{*}{$10^{-8}$} & MadNLP (gpu) & 66 & 0.2 & 66 & 0.2 & 66 & 0.2& 66 & 0.2 \\
                          & Ipopt (cpu) & 66 & 0.2 & 66 & 0.2 & 66 & 0.2& 66 & 0.2 \\
    \hline
  \end{tabular}
\end{table}


\paragraph{COPS Benchmark}
The results are summarized in Table \ref{tab:cops}, and the raw data are available in the paper repository.

\begin{table}[t]
  \centering\small
  \caption{Solve time in seconds and SGM10 on instances of COPS Benchmark}\label{tab:cops}
  \begin{tabular}{|c|c|cc|cc|cc|cc|}
    \hline
    \multirow{ 3}{*}{Tol} & \multirow{ 3}{*}{Solver} & \multicolumn{2}{c|}{\textbf{Small} (269)}& \multicolumn{2}{c|}{\textbf{Medium} (94)}& \multicolumn{2}{c|}{\textbf{Large} (20)}& \multicolumn{2}{c|}{\multirow{2}{*}{\textbf{Total} (383)}}\\
                          && \multicolumn{2}{c|}{(1 hr max)}& \multicolumn{2}{c|}{(1 hr max)}& \multicolumn{2}{c|}{(5 hr max)}&&\\
                          &&  Solved & Time &  Solved & Time &  Solved & Time &  Solved & Time \\
    \hline
    \multirow{2}{*}{$10^{-4}$} & MadNLP (gpu) & 66 & 0.2 & 66 & 0.2 & 66 & 0.2 & 66 & 0.2  \\
                          & Ipopt (cpu) & 66 & 0.2 & 66 & 0.2 & 66 & 0.2 & 66 & 0.2 \\
    \hline
    \multirow{2}{*}{$10^{-8}$} & MadNLP (gpu) & 66 & 0.2 & 66 & 0.2 & 66 & 0.2& 66 & 0.2 \\
                          & Ipopt (cpu) & 66 & 0.2 & 66 & 0.2 & 66 & 0.2& 66 & 0.2 \\
    \hline
  \end{tabular}
\end{table}

%%%%%%%%%%%%%%%%%%%%%%%%%%%%%%%%%%%%%%%%%%%%%%%%%%%%%%%%%%%%

\section{Conclusions and Open Questions}
We have presented a current landscape of GPU-accelerated second-order optimization solvers, focusing on the current capabilities of cuDSS and its use within mathematical programming solvers. With these capabilities, as demonstrated with numerical examples, it is possible to achieve more than an order of magnitude speed-up for large-scale instances. However, some open questions and implementation challenges remain, which we summarize below.
\begin{itemize}[leftmargin=*,itemsep=0pt,parsep=0pt,partopsep=0pt]
\item {Numerical precisions of condensed KKT systems}:
\item {Alternative Strategies}:
%% Here, we only provide pointers to our arXiv posting. It would be ideal to have NCL paper on arXiv so that we can cite it here.
Pointer to NCL paper.
Pointer to HybridKKT.
\item {Zero-pivot Tolerant Factorization Routines}:
\item {Batch mathematical programming solvers}:
\item {Utilizing tensor cores and dense linear algebra routines}:
\item {Hardware portability and cross-platform implementations}:
\end{itemize}
%% SS will write this part

\appendix

\section{Technical Appendices and Supplementary Material}
Technical appendices with additional results, figures, graphs and proofs may be submitted with the paper submission before the full submission deadline (see above), or as a separate PDF in the ZIP file below before the supplementary material deadline. There is no page limit for the technical appendices.

%%%%%%%%%%%%%%%%%%%%%%%%%%%%%%%%%%%%%%%%%%%%%%%%%%%%%%%%%%%%

\newpage
\section*{NeurIPS Paper Checklist}

%%% BEGIN INSTRUCTIONS %%%
The checklist is designed to encourage best practices for responsible machine learning research, addressing issues of reproducibility, transparency, research ethics, and societal impact. Do not remove the checklist: {\bf The papers not including the checklist will be desk rejected.} The checklist should follow the references and follow the (optional) supplemental material.  The checklist does NOT count towards the page
limit.

Please read the checklist guidelines carefully for information on how to answer these questions. For each question in the checklist:
\begin{itemize}
    \item You should answer \answerYes{}, \answerNo{}, or \answerNA{}.
    \item \answerNA{} means either that the question is Not Applicable for that particular paper or the relevant information is Not Available.
    \item Please provide a short (1–2 sentence) justification right after your answer (even for NA).
   % \item {\bf The papers not including the checklist will be desk rejected.}
\end{itemize}

{\bf The checklist answers are an integral part of your paper submission.} They are visible to the reviewers, area chairs, senior area chairs, and ethics reviewers. You will be asked to also include it (after eventual revisions) with the final version of your paper, and its final version will be published with the paper.

The reviewers of your paper will be asked to use the checklist as one of the factors in their evaluation. While "\answerYes{}" is generally preferable to "\answerNo{}", it is perfectly acceptable to answer "\answerNo{}" provided a proper justification is given (e.g., "error bars are not reported because it would be too computationally expensive" or "we were unable to find the license for the dataset we used"). In general, answering "\answerNo{}" or "\answerNA{}" is not grounds for rejection. While the questions are phrased in a binary way, we acknowledge that the true answer is often more nuanced, so please just use your best judgment and write a justification to elaborate. All supporting evidence can appear either in the main paper or the supplemental material, provided in appendix. If you answer \answerYes{} to a question, in the justification please point to the section(s) where related material for the question can be found.

IMPORTANT, please:
\begin{itemize}
    \item {\bf Delete this instruction block, but keep the section heading ``NeurIPS Paper Checklist"},
    \item  {\bf Keep the checklist subsection headings, questions/answers and guidelines below.}
    \item {\bf Do not modify the questions and only use the provided macros for your answers}.
\end{itemize}


%%% END INSTRUCTIONS %%%


\begin{enumerate}

\item {\bf Claims}
    \item[] Question: Do the main claims made in the abstract and introduction accurately reflect the paper's contributions and scope?
    \item[] Answer: \answerTODO{} % Replace by \answerYes{}, \answerNo{}, or \answerNA{}.
    \item[] Justification: \justificationTODO{}
    \item[] Guidelines:
    \begin{itemize}
        \item The answer NA means that the abstract and introduction do not include the claims made in the paper.
        \item The abstract and/or introduction should clearly state the claims made, including the contributions made in the paper and important assumptions and limitations. A No or NA answer to this question will not be perceived well by the reviewers.
        \item The claims made should match theoretical and experimental results, and reflect how much the results can be expected to generalize to other settings.
        \item It is fine to include aspirational goals as motivation as long as it is clear that these goals are not attained by the paper.
    \end{itemize}

\item {\bf Limitations}
    \item[] Question: Does the paper discuss the limitations of the work performed by the authors?
    \item[] Answer: \answerTODO{} % Replace by \answerYes{}, \answerNo{}, or \answerNA{}.
    \item[] Justification: \justificationTODO{}
    \item[] Guidelines:
    \begin{itemize}
        \item The answer NA means that the paper has no limitation while the answer No means that the paper has limitations, but those are not discussed in the paper.
        \item The authors are encouraged to create a separate "Limitations" section in their paper.
        \item The paper should point out any strong assumptions and how robust the results are to violations of these assumptions (e.g., independence assumptions, noiseless settings, model well-specification, asymptotic approximations only holding locally). The authors should reflect on how these assumptions might be violated in practice and what the implications would be.
        \item The authors should reflect on the scope of the claims made, e.g., if the approach was only tested on a few datasets or with a few runs. In general, empirical results often depend on implicit assumptions, which should be articulated.
        \item The authors should reflect on the factors that influence the performance of the approach. For example, a facial recognition algorithm may perform poorly when image resolution is low or images are taken in low lighting. Or a speech-to-text system might not be used reliably to provide closed captions for online lectures because it fails to handle technical jargon.
        \item The authors should discuss the computational efficiency of the proposed algorithms and how they scale with dataset size.
        \item If applicable, the authors should discuss possible limitations of their approach to address problems of privacy and fairness.
        \item While the authors might fear that complete honesty about limitations might be used by reviewers as grounds for rejection, a worse outcome might be that reviewers discover limitations that aren't acknowledged in the paper. The authors should use their best judgment and recognize that individual actions in favor of transparency play an important role in developing norms that preserve the integrity of the community. Reviewers will be specifically instructed to not penalize honesty concerning limitations.
    \end{itemize}

\item {\bf Theory assumptions and proofs}
    \item[] Question: For each theoretical result, does the paper provide the full set of assumptions and a complete (and correct) proof?
    \item[] Answer: \answerTODO{} % Replace by \answerYes{}, \answerNo{}, or \answerNA{}.
    \item[] Justification: \justificationTODO{}
    \item[] Guidelines:
    \begin{itemize}
        \item The answer NA means that the paper does not include theoretical results.
        \item All the theorems, formulas, and proofs in the paper should be numbered and cross-referenced.
        \item All assumptions should be clearly stated or referenced in the statement of any theorems.
        \item The proofs can either appear in the main paper or the supplemental material, but if they appear in the supplemental material, the authors are encouraged to provide a short proof sketch to provide intuition.
        \item Inversely, any informal proof provided in the core of the paper should be complemented by formal proofs provided in appendix or supplemental material.
        \item Theorems and Lemmas that the proof relies upon should be properly referenced.
    \end{itemize}

    \item {\bf Experimental result reproducibility}
    \item[] Question: Does the paper fully disclose all the information needed to reproduce the main experimental results of the paper to the extent that it affects the main claims and/or conclusions of the paper (regardless of whether the code and data are provided or not)?
    \item[] Answer: \answerTODO{} % Replace by \answerYes{}, \answerNo{}, or \answerNA{}.
    \item[] Justification: \justificationTODO{}
    \item[] Guidelines:
    \begin{itemize}
        \item The answer NA means that the paper does not include experiments.
        \item If the paper includes experiments, a No answer to this question will not be perceived well by the reviewers: Making the paper reproducible is important, regardless of whether the code and data are provided or not.
        \item If the contribution is a dataset and/or model, the authors should describe the steps taken to make their results reproducible or verifiable.
        \item Depending on the contribution, reproducibility can be accomplished in various ways. For example, if the contribution is a novel architecture, describing the architecture fully might suffice, or if the contribution is a specific model and empirical evaluation, it may be necessary to either make it possible for others to replicate the model with the same dataset, or provide access to the model. In general. releasing code and data is often one good way to accomplish this, but reproducibility can also be provided via detailed instructions for how to replicate the results, access to a hosted model (e.g., in the case of a large language model), releasing of a model checkpoint, or other means that are appropriate to the research performed.
        \item While NeurIPS does not require releasing code, the conference does require all submissions to provide some reasonable avenue for reproducibility, which may depend on the nature of the contribution. For example
        \begin{enumerate}
            \item If the contribution is primarily a new algorithm, the paper should make it clear how to reproduce that algorithm.
            \item If the contribution is primarily a new model architecture, the paper should describe the architecture clearly and fully.
            \item If the contribution is a new model (e.g., a large language model), then there should either be a way to access this model for reproducing the results or a way to reproduce the model (e.g., with an open-source dataset or instructions for how to construct the dataset).
            \item We recognize that reproducibility may be tricky in some cases, in which case authors are welcome to describe the particular way they provide for reproducibility. In the case of closed-source models, it may be that access to the model is limited in some way (e.g., to registered users), but it should be possible for other researchers to have some path to reproducing or verifying the results.
        \end{enumerate}
    \end{itemize}


\item {\bf Open access to data and code}
    \item[] Question: Does the paper provide open access to the data and code, with sufficient instructions to faithfully reproduce the main experimental results, as described in supplemental material?
    \item[] Answer: \answerTODO{} % Replace by \answerYes{}, \answerNo{}, or \answerNA{}.
    \item[] Justification: \justificationTODO{}
    \item[] Guidelines:
    \begin{itemize}
        \item The answer NA means that paper does not include experiments requiring code.
        \item Please see the NeurIPS code and data submission guidelines (\url{https://nips.cc/public/guides/CodeSubmissionPolicy}) for more details.
        \item While we encourage the release of code and data, we understand that this might not be possible, so “No” is an acceptable answer. Papers cannot be rejected simply for not including code, unless this is central to the contribution (e.g., for a new open-source benchmark).
        \item The instructions should contain the exact command and environment needed to run to reproduce the results. See the NeurIPS code and data submission guidelines (\url{https://nips.cc/public/guides/CodeSubmissionPolicy}) for more details.
        \item The authors should provide instructions on data access and preparation, including how to access the raw data, preprocessed data, intermediate data, and generated data, etc.
        \item The authors should provide scripts to reproduce all experimental results for the new proposed method and baselines. If only a subset of experiments are reproducible, they should state which ones are omitted from the script and why.
        \item At submission time, to preserve anonymity, the authors should release anonymized versions (if applicable).
        \item Providing as much information as possible in supplemental material (appended to the paper) is recommended, but including URLs to data and code is permitted.
    \end{itemize}


\item {\bf Experimental setting/details}
    \item[] Question: Does the paper specify all the training and test details (e.g., data splits, hyperparameters, how they were chosen, type of optimizer, etc.) necessary to understand the results?
    \item[] Answer: \answerTODO{} % Replace by \answerYes{}, \answerNo{}, or \answerNA{}.
    \item[] Justification: \justificationTODO{}
    \item[] Guidelines:
    \begin{itemize}
        \item The answer NA means that the paper does not include experiments.
        \item The experimental setting should be presented in the core of the paper to a level of detail that is necessary to appreciate the results and make sense of them.
        \item The full details can be provided either with the code, in appendix, or as supplemental material.
    \end{itemize}

\item {\bf Experiment statistical significance}
    \item[] Question: Does the paper report error bars suitably and correctly defined or other appropriate information about the statistical significance of the experiments?
    \item[] Answer: \answerTODO{} % Replace by \answerYes{}, \answerNo{}, or \answerNA{}.
    \item[] Justification: \justificationTODO{}
    \item[] Guidelines:
    \begin{itemize}
        \item The answer NA means that the paper does not include experiments.
        \item The authors should answer "Yes" if the results are accompanied by error bars, confidence intervals, or statistical significance tests, at least for the experiments that support the main claims of the paper.
        \item The factors of variability that the error bars are capturing should be clearly stated (for example, train/test split, initialization, random drawing of some parameter, or overall run with given experimental conditions).
        \item The method for calculating the error bars should be explained (closed form formula, call to a library function, bootstrap, etc.)
        \item The assumptions made should be given (e.g., Normally distributed errors).
        \item It should be clear whether the error bar is the standard deviation or the standard error of the mean.
        \item It is OK to report 1-sigma error bars, but one should state it. The authors should preferably report a 2-sigma error bar than state that they have a 96\% CI, if the hypothesis of Normality of errors is not verified.
        \item For asymmetric distributions, the authors should be careful not to show in tables or figures symmetric error bars that would yield results that are out of range (e.g. negative error rates).
        \item If error bars are reported in tables or plots, The authors should explain in the text how they were calculated and reference the corresponding figures or tables in the text.
    \end{itemize}

\item {\bf Experiments compute resources}
    \item[] Question: For each experiment, does the paper provide sufficient information on the computer resources (type of compute workers, memory, time of execution) needed to reproduce the experiments?
    \item[] Answer: \answerTODO{} % Replace by \answerYes{}, \answerNo{}, or \answerNA{}.
    \item[] Justification: \justificationTODO{}
    \item[] Guidelines:
    \begin{itemize}
        \item The answer NA means that the paper does not include experiments.
        \item The paper should indicate the type of compute workers CPU or GPU, internal cluster, or cloud provider, including relevant memory and storage.
        \item The paper should provide the amount of compute required for each of the individual experimental runs as well as estimate the total compute.
        \item The paper should disclose whether the full research project required more compute than the experiments reported in the paper (e.g., preliminary or failed experiments that didn't make it into the paper).
    \end{itemize}

\item {\bf Code of ethics}
    \item[] Question: Does the research conducted in the paper conform, in every respect, with the NeurIPS Code of Ethics \url{https://neurips.cc/public/EthicsGuidelines}?
    \item[] Answer: \answerTODO{} % Replace by \answerYes{}, \answerNo{}, or \answerNA{}.
    \item[] Justification: \justificationTODO{}
    \item[] Guidelines:
    \begin{itemize}
        \item The answer NA means that the authors have not reviewed the NeurIPS Code of Ethics.
        \item If the authors answer No, they should explain the special circumstances that require a deviation from the Code of Ethics.
        \item The authors should make sure to preserve anonymity (e.g., if there is a special consideration due to laws or regulations in their jurisdiction).
    \end{itemize}


\item {\bf Broader impacts}
    \item[] Question: Does the paper discuss both potential positive societal impacts and negative societal impacts of the work performed?
    \item[] Answer: \answerTODO{} % Replace by \answerYes{}, \answerNo{}, or \answerNA{}.
    \item[] Justification: \justificationTODO{}
    \item[] Guidelines:
    \begin{itemize}
        \item The answer NA means that there is no societal impact of the work performed.
        \item If the authors answer NA or No, they should explain why their work has no societal impact or why the paper does not address societal impact.
        \item Examples of negative societal impacts include potential malicious or unintended uses (e.g., disinformation, generating fake profiles, surveillance), fairness considerations (e.g., deployment of technologies that could make decisions that unfairly impact specific groups), privacy considerations, and security considerations.
        \item The conference expects that many papers will be foundational research and not tied to particular applications, let alone deployments. However, if there is a direct path to any negative applications, the authors should point it out. For example, it is legitimate to point out that an improvement in the quality of generative models could be used to generate deepfakes for disinformation. On the other hand, it is not needed to point out that a generic algorithm for optimizing neural networks could enable people to train models that generate Deepfakes faster.
        \item The authors should consider possible harms that could arise when the technology is being used as intended and functioning correctly, harms that could arise when the technology is being used as intended but gives incorrect results, and harms following from (intentional or unintentional) misuse of the technology.
        \item If there are negative societal impacts, the authors could also discuss possible mitigation strategies (e.g., gated release of models, providing defenses in addition to attacks, mechanisms for monitoring misuse, mechanisms to monitor how a system learns from feedback over time, improving the efficiency and accessibility of ML).
    \end{itemize}

\item {\bf Safeguards}
    \item[] Question: Does the paper describe safeguards that have been put in place for responsible release of data or models that have a high risk for misuse (e.g., pretrained language models, image generators, or scraped datasets)?
    \item[] Answer: \answerTODO{} % Replace by \answerYes{}, \answerNo{}, or \answerNA{}.
    \item[] Justification: \justificationTODO{}
    \item[] Guidelines:
    \begin{itemize}
        \item The answer NA means that the paper poses no such risks.
        \item Released models that have a high risk for misuse or dual-use should be released with necessary safeguards to allow for controlled use of the model, for example by requiring that users adhere to usage guidelines or restrictions to access the model or implementing safety filters.
        \item Datasets that have been scraped from the Internet could pose safety risks. The authors should describe how they avoided releasing unsafe images.
        \item We recognize that providing effective safeguards is challenging, and many papers do not require this, but we encourage authors to take this into account and make a best faith effort.
    \end{itemize}

\item {\bf Licenses for existing assets}
    \item[] Question: Are the creators or original owners of assets (e.g., code, data, models), used in the paper, properly credited and are the license and terms of use explicitly mentioned and properly respected?
    \item[] Answer: \answerTODO{} % Replace by \answerYes{}, \answerNo{}, or \answerNA{}.
    \item[] Justification: \justificationTODO{}
    \item[] Guidelines:
    \begin{itemize}
        \item The answer NA means that the paper does not use existing assets.
        \item The authors should cite the original paper that produced the code package or dataset.
        \item The authors should state which version of the asset is used and, if possible, include a URL.
        \item The name of the license (e.g., CC-BY 4.0) should be included for each asset.
        \item For scraped data from a particular source (e.g., website), the copyright and terms of service of that source should be provided.
        \item If assets are released, the license, copyright information, and terms of use in the package should be provided. For popular datasets, \url{paperswithcode.com/datasets} has curated licenses for some datasets. Their licensing guide can help determine the license of a dataset.
        \item For existing datasets that are re-packaged, both the original license and the license of the derived asset (if it has changed) should be provided.
        \item If this information is not available online, the authors are encouraged to reach out to the asset's creators.
    \end{itemize}

\item {\bf New assets}
    \item[] Question: Are new assets introduced in the paper well documented and is the documentation provided alongside the assets?
    \item[] Answer: \answerTODO{} % Replace by \answerYes{}, \answerNo{}, or \answerNA{}.
    \item[] Justification: \justificationTODO{}
    \item[] Guidelines:
    \begin{itemize}
        \item The answer NA means that the paper does not release new assets.
        \item Researchers should communicate the details of the dataset/code/model as part of their submissions via structured templates. This includes details about training, license, limitations, etc.
        \item The paper should discuss whether and how consent was obtained from people whose asset is used.
        \item At submission time, remember to anonymize your assets (if applicable). You can either create an anonymized URL or include an anonymized zip file.
    \end{itemize}

\item {\bf Crowdsourcing and research with human subjects}
    \item[] Question: For crowdsourcing experiments and research with human subjects, does the paper include the full text of instructions given to participants and screenshots, if applicable, as well as details about compensation (if any)?
    \item[] Answer: \answerTODO{} % Replace by \answerYes{}, \answerNo{}, or \answerNA{}.
    \item[] Justification: \justificationTODO{}
    \item[] Guidelines:
    \begin{itemize}
        \item The answer NA means that the paper does not involve crowdsourcing nor research with human subjects.
        \item Including this information in the supplemental material is fine, but if the main contribution of the paper involves human subjects, then as much detail as possible should be included in the main paper.
        \item According to the NeurIPS Code of Ethics, workers involved in data collection, curation, or other labor should be paid at least the minimum wage in the country of the data collector.
    \end{itemize}

\item {\bf Institutional review board (IRB) approvals or equivalent for research with human subjects}
    \item[] Question: Does the paper describe potential risks incurred by study participants, whether such risks were disclosed to the subjects, and whether Institutional Review Board (IRB) approvals (or an equivalent approval/review based on the requirements of your country or institution) were obtained?
    \item[] Answer: \answerTODO{} % Replace by \answerYes{}, \answerNo{}, or \answerNA{}.
    \item[] Justification: \justificationTODO{}
    \item[] Guidelines:
    \begin{itemize}
        \item The answer NA means that the paper does not involve crowdsourcing nor research with human subjects.
        \item Depending on the country in which research is conducted, IRB approval (or equivalent) may be required for any human subjects research. If you obtained IRB approval, you should clearly state this in the paper.
        \item We recognize that the procedures for this may vary significantly between institutions and locations, and we expect authors to adhere to the NeurIPS Code of Ethics and the guidelines for their institution.
        \item For initial submissions, do not include any information that would break anonymity (if applicable), such as the institution conducting the review.
    \end{itemize}

\item {\bf Declaration of LLM usage}
    \item[] Question: Does the paper describe the usage of LLMs if it is an important, original, or non-standard component of the core methods in this research? Note that if the LLM is used only for writing, editing, or formatting purposes and does not impact the core methodology, scientific rigorousness, or originality of the research, declaration is not required.
    %this research?
    \item[] Answer: \answerTODO{} % Replace by \answerYes{}, \answerNo{}, or \answerNA{}.
    \item[] Justification: \justificationTODO{}
    \item[] Guidelines:
    \begin{itemize}
        \item The answer NA means that the core method development in this research does not involve LLMs as any important, original, or non-standard components.
        \item Please refer to our LLM policy (\url{https://neurips.cc/Conferences/2025/LLM}) for what should or should not be described.
    \end{itemize}

\end{enumerate}


\end{document}
