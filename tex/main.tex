\documentclass{article}

% if you need to pass options to natbib, use, e.g.:
%     \PassOptionsToPackage{numbers, compress}{natbib}
% before loading neurips_2025

% The authors should use one of these tracks.
% Before accepting by the NeurIPS conference, select one of the options below.
% 0. "default" for submission
\usepackage[preprint]{neurips_2025}
% the "default" option is equal to the "main" option, which is used for the Main Track with double-blind reviewing.
% 1. "main" option is used for the Main Track
%  \usepackage[main]{neurips_2025}
% 2. "position" option is used for the Position Paper Track
%  \usepackage[position]{neurips_2025}
% 3. "dandb" option is used for the Datasets & Benchmarks Track
 % \usepackage[dandb]{neurips_2025}
% 4. "creativeai" option is used for the Creative AI Track
%  \usepackage[creativeai]{neurips_2025}
% 5. "sglblindworkshop" option is used for the Workshop with single-blind reviewing
 % \usepackage[sglblindworkshop]{neurips_2025}
% 6. "dblblindworkshop" option is used for the Workshop with double-blind reviewing
%  \usepackage[dblblindworkshop]{neurips_2025}

% After being accepted, the authors should add "final" behind the track to compile a camera-ready version.
% 1. Main Track
 % \usepackage[main, final]{neurips_2025}
% 2. Position Paper Track
%  \usepackage[position, final]{neurips_2025}
% 3. Datasets & Benchmarks Track
 % \usepackage[dandb, final]{neurips_2025}
% 4. Creative AI Track
%  \usepackage[creativeai, final]{neurips_2025}
% 5. Workshop with single-blind reviewing
%  \usepackage[sglblindworkshop, final]{neurips_2025}
% 6. Workshop with double-blind reviewing
%  \usepackage[dblblindworkshop, final]{neurips_2025}
% Note. For the workshop paper template, both \title{} and \workshoptitle{} are required, with the former indicating the paper title shown in the title and the latter indicating the workshop title displayed in the footnote.
% For workshops (5., 6.), the authors should add the name of the workshop, "\workshoptitle" command is used to set the workshop title.
% \workshoptitle{WORKSHOP TITLE}

% "preprint" option is used for arXiv or other preprint submissions
 % \usepackage[preprint]{neurips_2025}

% to avoid loading the natbib package, add option nonatbib:
% \usepackage[nonatbib]{neurips_2025}


\usepackage[utf8]{inputenc} % allow utf-8 input
\usepackage[T1]{fontenc}    % use 8-bit T1 fonts
\usepackage{hyperref}       % hyperlinks
\usepackage{url}            % simple URL typesetting
\usepackage{booktabs}       % professional-quality tables
\usepackage{amsfonts}       % blackboard math symbols
\usepackage{nicefrac}       % compact symbols for 1/2, etc.
\usepackage{microtype}      % microtypography
\usepackage{xcolor}         % colors
\usepackage{glossaries}
\usepackage{enumitem}
\usepackage{listings}
\lstset{basicstyle=\ttfamily\small}
\usepackage{multirow}
\usepackage{cleveref}
\usepackage{graphicx}

\crefname{equation}{}{}


\newacronym{pde}{PDE}{partial differential equation}
\newacronym{ad}{AD}{automated differentiation}
\newacronym{alcf}{ALCF}{Argonne Leadership Computing Facility}
\newacronym{blas}{BLAS}{basic linear algebra subprograms}
\newacronym{iap}{IAP}{independent activities period}
\newacronym{der}{DER}{distributed energy resource}
\newacronym{derms}{DERMS}{distributed energy resource management system}
\newacronym{minlp}{MINLP}{mixed-integer nonlinear programming}
\newacronym{nlp}{NLP}{nonlinear programming}
\newacronym{kkt}{KKT}{Karush-Kuhn-Tucker}
\newacronym{sqp}{SQP}{sequential quadratic programming}
\newacronym{ipm}{IPM}{interior-point method}
\newacronym{cpu}{CPU}{central processing units}
\newacronym{gpu}{GPU}{graphics processing units}
\newacronym{mpc}{MPC}{model predictive control}
\newacronym{ac}{AC}{alternating current}
\newacronym{dc}{DC}{direct current}
\newacronym{opf}{OPF}{optimal power flow}
\newacronym{hpc}{HPC}{high-performance computing}
\newacronym{pcg}{PCG}{projected conjugate gradient}
\newacronym{alm}{ALM}{augmented Lagrangian method}
\newacronym{dac}{DAC}{direct air capture}
\newacronym{pem}{PEM}{proton exchange membrane}
\newacronym{tea}{TEA}{technoeconomic analysis}
\newacronym{lca}{LCA}{life cycle assessment}
\newacronym{ft}{FT}{Fischer-Tropsch}
\newacronym{bess}{BESS}{battery energy storage system}
\newacronym{cqa}{CQA}{critical quality attribute}
\newacronym{lnp}{LNP}{lipid nanoparticle}
\newacronym{iso}{ISO}{independent system operator}
\newacronym{simd}{SIMD}{single instruction, multiple data}
\newacronym{mimd}{MIMD}{multiple instruction, multiple data}
\newacronym{orcd}{ORCD}{Office of Research Computing and Data}
\newacronym{mitei}{MITei}{MIT Energy Initiative}
\newacronym{hipsat}{HIP-SAT}{High School Introduction to the Physical Sciences and Advanced Technologies}
\newacronym{mpi}{MPI}{message passing interface}
\newacronym{goc}{GOC}{grid optimization competition}
\newacronym{sc}{SC}{security-constrained}
\newacronym{mp}{MP}{multi-period}
\newacronym{admm}{ADMM}{alternating direction method of multipliers}
\newacronym{gmres}{GMRES}{generalized mean residual}
\newacronym{pdhg}{PDHG}{primal-dual hybrid gradient}
\newacronym{lp}{LP}{linear programming}
\newacronym{ams}{AMS}{algebraic modeling system}
\newacronym{ncl}{NCL}{nonlinear constrained Lagrangian}
\newacronym{bcl}{BCL}{bound constrained Lagrangian}
\newacronym{sqd}{SQD}{symmetric quasi-definite}
\newacronym{spd}{SPD}{symmetric positive definite}


% Note. For the workshop paper template, both \title{} and \workshoptitle{} are required, with the former indicating the paper title shown in the title and the latter indicating the workshop title displayed in the footnote.
\title{On the GPU Implementation of Second-Order Linear and Nonlinear Programming Solvers}




% The \author macro works with any number of authors. There are two commands
% used to separate the names and addresses of multiple authors: \And and \AND.
%
% Using \And between authors leaves it to LaTeX to determine where to break the
% lines. Using \AND forces a line break at that point. So, if LaTeX puts 3 of 4
% authors names on the first line, and the last on the second line, try using
% \AND instead of \And before the third author name.


\author{%
  Alexis Montoison\\
  Mathematics and Computer Science Division\\
  Argonne National Laboratory\\
  Lemont, IL 60439\\
  \texttt{amontoison@anl.gov}\\
  \And
  Fran\c{c}ois Pacaud\\
  Centre Automatique et Systèmes\\
  Mines Paris-PSL\\
  Paris, 75006 \\
  \texttt{francois.pacaud@minesparis.psl.eu}\\
  \And
  Sungho Shin\\
  Department of Chemical Engineering\\
  Massachusetts Institute of Technology\\
  Cambridge, MA 02139\\
  \texttt{sushin@mit.edu}\\
  \And
  Mihai Anitescu\\
  Mathematics and Computer Science Division\\
  Argonne National Laboratory\\
  Lemont, IL 60439\\
  \texttt{anitescu@mcs.anl.gov}\\
}


\begin{document}


\maketitle


\begin{abstract}
In recent years, GPU-accelerated optimization solvers based on second-order methods (e.g., interior-point) have gained momentum with the advent of mature and efficient GPU-accelerated direct sparse linear solvers, such as cuDSS.
This paper provides an overview of the state of the art in GPU-based second-order solvers, focusing on \emph{pivoting-free interior-point methods} for linear and nonlinear programming.
We begin by highlighting the capabilities and limitations of the currently available GPU-accelerated sparse linear solvers.
Next, we discuss different formulations of the Karush-Kuhn-Tucker systems for second-order methods and evaluate their suitability for (pivoting-free) GPU implementations.
We also discuss strategies for computing sparse Jacobians and Hessians on GPUs, which serve as key building blocks of the KKT systems.
Finally, we present numerical experiments demonstrating the scalability of GPU-based optimization solvers, with speedups often exceeding 10× compared to the best CPU alternatives on large-scale instances at medium precision, and we examine the current limitations of existing approaches.
\end{abstract}

\section{Introduction}\label{eqn:intro}

This paper focuses on the implementation of solvers for problems of the following form:
\begin{align}\label{eqn:opt}
  \min_{x } \; f(x) \quad \text{s.t.} \quad g(x) \geq 0,
\end{align}
where \(x \in \mathbb{R}^n\) is the decision variable, and \(f: \mathbb{R}^n \to \mathbb{R}\) and \(g: \mathbb{R}^n \to \mathbb{R}^m\) are smooth objective and constraint functions. 
For simplicity, we do not explicitly consider equality constraints. 
We will discuss both \gls*{lp} (where \(f\) and \(g\) are affine) and \gls*{nlp} (where \(f\) and \(g\) are nonlinear), with an emphasis on algorithms designed for large and sparse instances.

Despite advances in general-purpose GPU computing, state-of-the-art mathematical programming solvers have not widely adopted these techniques. GPUs excel in repetitive computations on large data sets, such as dense matrix multiplication in AI model training. However, many mathematical programming problems in classical application areas are sparse, lack a uniform memory layout, and therefore do not benefit from the same kind of parallelism as in dense linear algebra. As a result, integrating GPUs into mathematical programming solvers poses greater challenges and often necessitates substantial modifications to the overall algorithm.


First-order algorithms have emerged as a suitable class for GPU implementation. Since these algorithms typically rely on sparse matrix-vector multiplication and simple vector operations, implementing GPU acceleration is less likely to be hindered by the unavailability of low-level kernels. Recent successful implementations include cuPDLP \cite{luCuPDLPCStrengthenedImplementation2024,luCuPDLPFurtherEnhanced2025}, cuOSQP \cite{schubigerGPUAccelerationADMM2020}, and cuOPT \cite{NVIDIACuopt2025}. However, the linear convergence rate of first-order methods restricts their effectiveness in applications requiring fast convergence, prompting the exploration of second-order alternatives.

Second-order solvers inherently rely on \emph{direct linear solvers}. For example, within \gls*{ipm}, each barrier iteration necessitates solving a linear system known as the \gls*{kkt} system. These systems become highly ill-conditioned near the solution, which renders the use of iterative linear solvers, such as preconditioned Krylov methods, ineffective in most cases. The issue is exacerbated in nonconvex settings, where KKT systems lack \gls*{sqd} structure unless aggressive regularization is employed.
For years, the development of GPU-accelerated second-order solvers has been constrained by the lack of robust and efficient sparse direct linear solvers on GPUs and only dense factorizations were provided by GPU vendors.
This severely limited the problem sizes that could be addressed due to GPU memory constraints and, at the same time, prevented any advantage from exploiting sparsity.
Even the dense Bunch-Kaufman factorization and the corresponding backsolve (\emph{sytrf} and \emph{sytrs}) needed for symmetric indefinite systems took considerable time to be fully ported to the GPU.
It is available since CUDA 11.4 for NVIDIA, and still not fully supported on AMD and Intel GPUs.

This status quo has changed with NVIDIA's release of cuDSS, a library of direct sparse linear solvers for GPUs.
It provide sparse Cholesky, LDL$^\top$ and LU factorizations.
While it currently lacks the LBL$^\top$ factorization capabilities commonly used for \gls*{nlp} solvers, its LDL$^\top$ and Cholesky functionalities are sufficient for implementing modified versions of the \gls*{ipm}.
Consequently, cuDSS has spurred advances in GPU-accelerated second-order solvers, including MadNLP \cite{shinAcceleratingOptimalPower2024} and Clarabel \cite{goulartClarabelInteriorpointSolver2024}, integrating it and achieving significant speedups on large-scale instances \cite{shinNVIDIACuDSSLibrary2024,shinAcceleratingOptimalPower2024,pacaudCondensedspaceMethodsNonlinear2024,shinScalableMultiPeriodAC2024,pacaudGPUacceleratedDynamicNonlinear2024}.

This paper provides an overview of the current state of the art in GPU implementations of second-order optimization solvers, with an emphasis on the following aspects:
(i) The \gls*{ipm} is considered as the primary mechanism for handling inequality constraints, since the alternative paradigm, active-set methods, is generally regarded as less scalable \cite{nocedalNumericalOptimization2006}.
(ii) The focus is on solving KKT systems, which present the greatest challenges; other components, such as barrier parameter tuning and line search, can be ported to GPUs relatively straightforwardly using low-level kernels via \texttt{map} or \texttt{reduce} operations.
(iii) The discussion centers on NVIDIA GPUs and their software stack, as NVIDIA currently offers the most mature direct sparse solver capabilities.
(iv) Due to page constraints, hybrid \gls*{kkt} strategies \cite{regevHyKKTHybridDirectiterative2023}, reduced-space methods \cite{pacaudAcceleratingCondensedInteriorPoint2023}, and other domain-specific approaches \cite{adabagMPCGPURealTimeNonlinear2024} are not covered.

\section{Direct Linear Solvers for Optimization}\label{eqn:linear}
This section provides an overview of the direct linear algebra methods frequently employed in second-order methods and discusses the rationale behind the development of \emph{pivoting-free \gls*{ipm}}.

\paragraph{LDL$^\top$ Factorization}
LDL$^\top$ factorization, a signed variant of Cholesky decomposition, decomposes a matrix $A$ into $LDL^\top$, where $L$ is lower triangular and $D$ is diagonal (for sparse systems, a fill-in reducing permutation (reordering) $P$ must be employed, resulting in the permuted decomposition $A = P L D L^\top P^\top$). This method can be utilized to solve $Ax = b$, where the solution is obtained by first solving the lower triangular system $Ly = b$, followed by diagonal scaling with $D^{-1}$ and solving the upper triangular system $L^\top x = y$. A notable property of LDL$^\top$ factorization is that, provided the matrix $A$ is \gls*{sqd}, LDL$^\top$ factorization exists for any given permutation of the matrix (so-called strongly factorizable) \cite{vanderbeiSymmetricQuasidefiniteMatrices1995}.
Many of the saddle point systems encountered in mathematical programming solvers are \gls*{sqd}, can become \gls*{sqd} with infinitesimal regularization), or can be converted to \gls*{sqd} systems.

\paragraph{Numerical Pivoting}

For general indefinite matrices without \gls*{sqd} structure (e.g., augmented systems \cite{wachterImplementationInteriorpointFilter2006} arising from nonconvex \glspl*{nlp}), the LDL$^\top$ factorization is not guaranteed to exist, and dynamic numerical pivoting is commonly employed in an attempt to avoid zero pivots and improve the numerical stability of the factorization process.
Dynamic numerical pivoting procedures examine a limited set of candidate pivots, typically within a row and column, and select the most suitable one according to a stability criterion, such as the largest absolute value \cite{schenkFASTFACTORIZATIONPIVOTING}.
However, this requires deviating from the fill-in reducing reordering $P$ computed a priori, leading to additional fill-in (and thus memory reallocation) and potentially breaking parallelizability.
Three widely used dynamic pivoting strategies are Bunch–Kaufman, rook, and delayed pivoting, which select $1 \times 1$ or $2 \times 2$ pivots, although other variants and hybrid approaches also exist \cite{duff2017}.
The variant of LDL$^\top$ with $2 \times 2$ pivots is often called LBL$^\top$ factorization.
If none of these methods succeed, the pivot is perturbed by a small value (typically $\mathbf{u}\|A\|_1$ for some small $\mathbf{u} > 0$) \cite{schenkFASTFACTORIZATIONPIVOTING}.
This procedure introduces numerical error, which must be corrected through iterative refinements.


\paragraph{State-of-the-Art GPU Solvers}
As described above, the numerical pivoting procedure is crucial for ensuring the numerical stability of direct sparse linear solvers.
However, implementing numerical pivoting has been recognized as one of the most challenging components of direct sparse linear solvers on GPUs, as these strategies are serial in nature \cite{swirydowiczLinearSolversPower2022}.
Moreover, since coarse-grained tree-level parallelism must be employed to exploit GPU parallelism, numerical pivoting should be applied in such a way that does not break the parallelism at the elimination tree level, which further complicates the implementation.
The current version of cuDSS has partial pivoting capabilities, but it does not support the LBL$^\top$ factorization as seen in CPU solvers \cite{nvidiaNVIDIACuDSSPreview}.

% On the other hand, when the matrix is \gls*{sqd}, LDL$^\top$ factorization on GPU can achieve a high degree of robustness even without numerical pivoting \cite{vanderbeiSymmetricQuasidefiniteMatrices1995}.
% Furthermore, these methods can effectively utilize parallelism; the factorization process is often highly parallelizable, while the triangular factorization tends to be less parallel but can still benefit from GPU parallelism \cite{naumovParallelSolutionSparse}.
Although algorithms for computing fill-in-reducing reorderings (e.g., minimum degree ordering \cite{amestoyApproximateMinimumDegree1996} or nested dissection \cite{karypisMETISSoftwarePackage1997}) are inherently serial (cuDSS, for instance, performs this operation entirely on the CPU), the reordering needs to be computed only once and can be reused across multiple factorizations, allowing the overhead to be amortized.
Therefore, to fully exploit the benefits of existing GPU direct solvers, it is crucial to ensure that \emph{the \gls*{kkt} system can be solved without pivoting}, which motivates the development of \emph{pivoting-free interior-point methods}.
Specifically, this can be achieved by always converting the \gls*{kkt} systems into an \gls*{sqd} form.


\section{Pivoting-Free Interior-Point Methods}\label{eqn:ipm}
We now explain how the \gls*{ipm} can be adapted to avoid numerical pivoting, thereby enabling the use of GPU direct solvers with limited pivoting capabilities. We first provide a brief overview of the \gls*{ipm} and its KKT system formulation, followed by a discussion of the condensed KKT systems.


\paragraph{Interior-Point Methods and KKT Systems}  
The \gls*{ipm} is a class of optimization algorithms designed to solve inequality-constrained optimization problems \cite{nocedalNumericalOptimization2006}. These methods transform \cref{eqn:opt} into a sequence of log-barrier subproblems:
\begin{align}\label{eqn:barrier}
  \min_{x,s} \; & f(x) - \mu \sum_{i=1}^m \log(s_i) \quad \text{s.t.} \; g(x) - s = 0,
\end{align}
where $s \in \mathbb{R}^m$ denotes the slack variable, and $\mu > 0$ is the barrier parameter, which is gradually reduced to zero.  
\Gls*{ipm} attempts to iteratively solve the \gls*{kkt} conditions for \cref{eqn:barrier}:
\begin{align}\label{eqn:kkt}
  \nabla f(x) - \nabla g(x)^\top \lambda = 0, \quad
  S \Lambda e = \mu e, \quad
  g(x) - s = 0,
\end{align}
where $\lambda \in \mathbb{R}^m$ are the Lagrange multipliers, $S = \text{diag}(s)$, $\Lambda = \text{diag}(\lambda)$, and $e$ is the vector of ones, using Newton's method.
At each iteration, we obtain the search direction by solving the following (regularized) KKT system:
\begin{align}\label{eqn:kkt_system}
  \begin{bmatrix}
    \nabla^2_{x x} \mathcal{L}(x,\lambda) + \delta_p I & & \nabla g(x)^\top  \\
                                          & S^{-1}\Lambda & -I \\
    \nabla g(x) & -I &  - \delta_d I\\
  \end{bmatrix}
  \begin{bmatrix}
    \phantom{-}d_x\\
    \phantom{-}d_s \\
    -d_\lambda
  \end{bmatrix} =
  -\begin{bmatrix}
    \nabla f(x) - \nabla g(x)^\top \lambda\\
    \Lambda e - \mu S^{-1} e \\
    g(x) - s\\
  \end{bmatrix},
\end{align}
where $\mathcal{L}(x,\lambda,s) := f(x) - \lambda^\top (g(x)-s)$ is the Lagrangian function and $\delta_p, \delta_d \geq 0$ are the primal and dual regularization parameters.


\paragraph{Regularization}
The regularization parameters are used to ensure (i) the well-posedness of \cref{eqn:kkt_system} and/or (ii) the descent property of the Newton step. For convex problems, infinitesimal $\delta_p, \delta_d > 0$ ensures the \gls*{sqd} condition for the matrix in \eqref{eqn:kkt_system}, thus ensuring strong factorizability. This idea has led to several robust \gls{ipm} implementations on CPUs~\cite{friedlanderPrimalDualRegularized2012}. For nonconvex cases, primal-dual regularization provides a mechanism to not only impose the \gls*{sqd} structure but also to ensure that the Newton step is a descent direction for a merit function. Traditional implementations of \gls{ipm} use a procedure known as \emph{inertia correction}, where the regularization parameters $(\delta_p, \delta_d)$ are increased until the number of positive, negative, and zero eigenvalues (referred to as inertia, available as a byproduct of the LDL$^\top$ and LBL$^\top$ factorizations) equals $(n+m, m, 0)$. Excessive regularization is not desirable, as it can potentially distort the step direction, leading to slow convergence.


\paragraph{Condensed KKT Systems}
While \cref{eqn:kkt_system} is directly addressed by some solvers (e.g., Ipopt \cite{wachterImplementationInteriorpointFilter2006}), it can be further \emph{condensed}, leading to so-called \emph{condensed \gls*{kkt} systems}.
In the context of GPU implementation, condensation offers advantages either through (i) reducing the system size and increasing its density—thereby providing more opportunities for parallelism—or (ii) enforcing the \gls*{sqd} structure, which enables a pivoting-free implementation.
However, depending on the sparsity pattern, the condensed system can become significantly denser, leading to higher memory requirements and computational overhead.
Moreover, since the eliminated blocks are often highly ill-conditioned near the optimum, the resulting condensed system may also suffer from ill-conditioning, potentially compromising solution accuracy.

Below, we outline such condensation strategies. Note that these analyses are subject to change if equality constraints are present, but one can convert equalities into inequalities by introducing pairs of inequalities or slightly relaxing the equality.
\begin{itemize}[leftmargin=*,itemsep=0pt,parsep=0pt,partopsep=0pt]
\item \textit{Primal-Dual Condensed System}:  
Since the \((2,2)\)-block \(S^{-1}\Lambda\) in \cref{eqn:kkt_system} is always invertible due to the nature of the \gls*{ipm}, we can eliminate it to obtain the following \(2 \times 2\) block system:  
\begin{equation}\label{eqn:augmentedKKT}
  \begin{bmatrix}
    \nabla^2_{xx} \mathcal{L}(x,\lambda) + \delta_p I & \nabla g(x)^\top \\
    \nabla g(x) &  - \delta_d I - \Lambda^{-1} S
  \end{bmatrix}
  \begin{bmatrix}
    \phantom{-}d_x\\
    - d_\lambda
  \end{bmatrix} =
  -\begin{bmatrix}
    \nabla f(x) - \nabla g(x)^\top \lambda\\
    g(x) - \mu \Lambda^{-1} e
  \end{bmatrix}.
\end{equation}  
This elimination does not incur significant computational overhead, and the number of non-zero entries in the resulting system does not increase.

\item \textit{Primal Condensed System}:
The $\delta_d I + \Lambda^{-1}S$ block within \eqref{eqn:augmentedKKT} is always invertible, and its elimination gives rise to a \emph{primal condensed KKT system}:
\begin{align}\label{eqn:kkt_primal}
  \left(\nabla^2_{x x} \mathcal{L}(x, \lambda) + \delta_p I + \nabla g(x)^\top (\delta_d I + \Lambda^{-1} S)^{-1} \nabla g(x) \right) d_x = - r_p \; ,
\end{align}
where $r_p$ is an appropriate right-hand side derived from \eqref{eqn:augmentedKKT}.
Compared to \cref{eqn:augmentedKKT}, the system size is further reduced.
This condensation has the key advantage for \glspl*{nlp}, as this system becomes \gls*{spd} under the application of primal-dual regularization $(\delta_p, \delta_d)$ chosen based on the standard inertia correction procedure \cite{shinAcceleratingOptimalPower2024}.
However, since the Jacobian $\nabla g(x)$ can have dense rows, the condensed system can become arbitrarily dense, potentially causing serious computational overhead (in such a case, a specialized linear solver needs to be used based on the Woodbury formula).


\item \textit{Dual Condensed System}:
When the problem is strongly convex or when the regularization parameter $\delta_p$ is sufficiently large, the $\nabla^2 \mathcal{L}(x,\lambda) + \delta_p I$ block is invertible, and by eliminating it, we obtain the \emph{dual condensed KKT system}:
\begin{align}\label{eqn:kkt_dual}
  \left(\delta_d I + \Lambda^{-1}S + \nabla g(x)\left(\nabla_{x x}^2 \mathcal{L}(x,\lambda) + \delta_p I\right)^{-1} \nabla g(x)^\top\right)
  d_\lambda = - r_d \; ,
\end{align}
with $r_d$ an appropriate right-hand-side.
In the context of linear programming, this is always applicable when $\delta_p>0$, and this system is often called \emph{normal equations} and is often used by default as the diagonal block $\nabla^2_{x x} \mathcal{L}(x, \lambda) + \delta_p I$ is easy to invert.
Assuming that the primal Hessian is \gls*{spd}, this system is always \gls*{spd}.
Similarly to the primal condensed system, this system can be arbitrarily dense (when there is a dense column in $\nabla g(x)$.
\end{itemize}


\paragraph{Pivoting-Free IPM}
We now explain which KKT system formulation among \cref{eqn:kkt_system,eqn:augmentedKKT,eqn:kkt_dual,eqn:kkt_primal} is suitable for pivoting-free \gls*{ipm} implementations. The key requirement is that the KKT system matrix must be \gls*{sqd} or can be made \gls*{sqd} with infinitesimal regularization $\delta_p, \delta_d > 0$. We explain the conditions for each case below.
\begin{itemize}[leftmargin=*,itemsep=0pt,parsep=0pt,partopsep=0pt]
\item \textit{Convex Case}:
  For convex programs, all four formulations \cref{eqn:kkt_system,eqn:augmentedKKT,eqn:kkt_dual,eqn:kkt_primal} are appropriate, as any of these systems can become SQD for infinitesimal $\delta_p, \delta_d > 0$. Thus, one can choose the best option based on the sparsity pattern and degree of ill-conditioning of the KKT system matrix.  
\item \textit{Nonconvex Case}:
  For nonconvex problems, the primal condensed system \cref{eqn:kkt_primal} is the most suitable, as it can be made \gls*{spd} by choosing the primal-dual regularization parameters $(\delta_p, \delta_d)$ simply based on inertia correction (i.e., strong factorizability is ensured without incurring excessive regularization).
  The augmented or primal-dual condensed systems \cref{eqn:kkt_system,eqn:augmentedKKT} are not suitable because they are not guaranteed to be \gls*{sqd} unless \emph{additionally strong} (beyond what's necessary to ensure the descent condition) regularization parameters $(\delta_p, \delta_d)$ are used.
  The dual condensed system \cref{eqn:kkt_dual} is not suitable either, as the $\nabla^2_{x x} \mathcal{L}(x, \lambda) + \delta_p I$ block is typically not diagonal (due to nonlinear constraints).
\end{itemize}



\section{Algebraic Modeling Systems and Automatic Differentiation}\label{eqn:ad}
\Gls*{nlp} solvers require external oracles to evaluate $f$, $g$, and their first and second-order derivatives. In most modern optimization software stacks, the derivative evaluation code (either compiled or interpreted) is generated in a fully automated fashion through the so-called \emph{algebraic modeling systems}, which are typically equipped with \gls*{ad} capabilities, such as AMPL \cite{fourerModelingLanguageMathematical1990}, CasADi \cite{anderssonCasADiSoftwareFramework2019}, JuMP \cite{dunningJuMPModelingLanguage2017}, Pyomo \cite{hartPyomoModelingSolving2011}, and Gravity \cite{hijaziGravityMathematicalModeling2018}. As mathematical programming applications are primarily sparse, these systems have historically been developed independently of machine learning frameworks, which typically focus on dense matrix operations.

To enable efficient derivative evaluations and ensure a fully GPU-resident optimization workflow, it is crucial to develop algebraic modeling systems that provide derivative evaluation code in the form of GPU kernels. To achieve this, one can focus on the observation that many practical instances of large-scale sparse mathematical programs exhibit highly repetitive structures. For example, $f$ may be a sum of many terms (e.g., $f(x) = \sum_{p\in P} f^o(x; p)$), and $g$ may be a collection of many constraints generated from a common template (e.g., $g(x) = \left\{g^o(x; p)\right\}_{p\in P}$). If such a structure exists, the evaluation and differentiation of $f$ and $g$ become embarrassingly parallel, making it feasible to construct them as GPU kernels. Emerging algebraic modeling systems, such as ExaModels.jl \cite{shinAcceleratingOptimalPower2024} or PyOptInterface \cite{yangPyOptInterfaceDesignImplementation2024}, are designed to capture such repeated structures and provide derivative evaluation kernels in an automated fashion. For instance, ExaModels.jl requires users to specify the objective and constraint functions in the form of an iterator, such as
\begin{lstlisting}
  objective(c, 100 * (x[i-1]^2 - x[i])^2 + (x[i-1] - 1)^2 for i = 2:N)
\end{lstlisting}
which corresponds to the case of $f_o(x, p) =$ \lstinline|100 * (x[p-1]^2 - x[p])^2 + (x[p-1] - 1)^2| for $p\in\{2,\cdots,N\}$. This syntax allows the user to inform the modeling system of repeated structures in the model so that the GPU kernel for derivative evaluation can be generated later.


\section{Numerical Results}\label{eqn:num} 
We benchmarked the performance of two GPU implementations (MadIPM for \glspl*{lp} and MadNLP for \glspl*{nlp}) against the CPU solvers (Gurobi for \glspl*{lp} and Ipopt for \glspl*{nlp}).
MadNLP is configured with cuDSS as the linear solver, while Ipopt is set up with either Ma27 (for pglib-opf) or Ma57 (for COPS), utilizing OpenBLAS as the BLAS and LAPACK backends. All nonlinear models are implemented with ExaModels (running on either GPU or CPU).
For \glspl*{lp}, we performed the benchmark on the MIPLIB 2017 benchmark library \cite{gleixnerMIPLIB2017Datadriven2021}, while for \glspl*{nlp}, the benchmark was conducted on the pglib-opf \cite{babaeinejadsarookolaeePowerGridLibrary2021} and the COPS library \cite{dolanBenchmarkingOptimizationSoftware2001}.
The results are summarized in \Cref{tab:results}.
The instances are classified into small, medium, and large categories based on the number of nonzeros in the Lagrangian Hessian and constraint Jacobian.
For each instance, we separately report the results for medium precision (tol$=10^{-4}$) and high precision (tol$=10^{-8}$).
More details of the numerical experiments can be found in \Cref{apx:num}.
When reporting the solve time for multiple instances, we represent them using the shifted geometric mean: $(\prod_{i=1}^n (t_i + \Delta))^{1/n} - \Delta$, where $t_i$ is the solve time for the $i$-th instance and $\Delta$ is the shift factor. We denote this metric with $\Delta = 10$ as SGM10.
If an instance is unsolved, its solving time is set to the corresponding time limit.
All numerical results are shown in \Cref{tab:results}.
The numerical results can be reproduced with the source code and Manifest file available at \url{https://github.com/MadNLP/neurips2025-mathprog-on-gpu}.
The benchmark was performed on a workstation with two Intel(R) Xeon(R) Platinum 6130 CPUs @ 2.10GHz, two Quadro GV 100 GPUs, and 128 GB of RAM.

\begin{table} 
  \footnotesize
  \begin{tabular}{|c|c|c|cc|cc|cc|cc|}
  \hline
  &\multirow{ 3}{*}{\bfseries Tol} & \multirow{ 3}{*}{\bfseries Solver} & \multicolumn{2}{c|}{\textbf{Small}}& \multicolumn{2}{c|}{\textbf{Medium}}& \multicolumn{2}{c|}{\textbf{Large}}& \multicolumn{2}{c|}{\multirow{2}{*}{\textbf{Total}}}\\
  &&& \multicolumn{2}{c|}{nnz $<2^{18}$}& \multicolumn{2}{c|}{$2^{18}\leq$ nnz $<2^{20}$}& \multicolumn{2}{c|}{$2^{20}\leq$ nnz}&&\\
  &&&  Solved & Time &  Solved & Time &  Solved & Time &  Solved & Time \\
  \hline\hline
  \multirow{4}{*}{\rotatebox{90}{\bfseries MIPLIB}}&    \multirow{2}{*}{$10^{-4}$} & MadIPM & 87 & 1.3013 & 56 & 5.0480 & 27 & 19.7925 & 170 & 4.5319  \\
  && Gurobi & 88 & 1.5439 & 58 & 10.4671 & 23 & 78.5783 & 169 & 9.3939  \\
  \cline{2-11}
  &\multirow{2}{*}{$10^{-8}$} & MadIPM & 85 & 2.8157 & 48 & 18.2642 & 25 & 33.1676 & 158 & 10.2820  \\
  && Gurobi & 88 & 1.5708 & 58 & 10.6148 & 24 & 76.3206 & 170 & 9.3826  \\
  \hline\hline
  \multirow{4}{*}{\rotatebox{90}{\bfseries OPF}}&\multirow{2}{*}{1e-4} & MadNLP & 31 & 0.4166 & 24 & 2.6380 & 11 & 3.7040 & 66 & 1.6979  \\
  && Ipopt & 31 & 0.3970 & 24 & 5.0697 & 11 & 38.5053 & 66 & 5.3817  \\
  \cline{2-11}
  &\multirow{2}{*}{1e-8} & MadNLP & 30 & 2.5037 & 24 & 4.6016 & 10 & 12.8040 & 64 & 4.6228  \\
  && Ipopt & 31 & 0.5100 & 24 & 5.4292 & 11 & 37.7818 & 66 & 5.5541  \\
  \hline\hline
  \multirow{4}{*}{\rotatebox{90}{\bfseries COPS}}&\multirow{2}{*}{1e-4} & MadNLP & 13 & 0.8665 & 15 & 4.8665 & 16 & 3.8194 & 44 & 3.2314  \\
  && Ipopt & 13 & 5.2315 & 15 & 15.9701 & 15 & 45.8411 & 43 & 19.2243  \\
  \cline{2-11}
  &\multirow{2}{*}{1e-8} & MadNLP & 13 & 0.8575 & 16 & 1.5572 & 16 & 8.3549 & 45 & 3.3797  \\
  && Ipopt & 13 & 5.9413 & 15 & 17.6758 & 15 & 40.8639 & 43 & 19.2999  \\ 
  \hline
\end{tabular}

%%% Local Variables:
%%% mode: LaTeX
%%% TeX-master: "main"
%%% End:

  \caption{Solve times of CPU solvers (Gurobi and Ipopt) and GPU solvers (MadIPM and MadNLP) in SGM10 (seconds; with a maximum wall time of 900 seconds) on MIPLIB (\glspl*{lp}), pglib-opf (31 small, 24 medium, and 11 large \glspl*{nlp}), and COPS (13 small, 16 medium, and 16 large \glspl*{nlp}) instances.}\label{tab:results} 
\end{table}


\paragraph{MIPLIB}


\paragraph{PGLIB-OPF}
We have benchmarked the solver performance for solving AC OPF problems based on polar power flow formulations using the approach in \cite{PowerModelsJLOpenSource}. The results in \Cref{tab:results} indicate that the GPU solver can achieve, on average, more than a 10x speed-up for the 11 largest instances (with more than $2^{20}$ non-zeros) when the problems are solved to medium precision. The speed-up is relatively modest for medium-sized instances, and there is practically no advantage for small instances. This is expected, as the GPU solver is designed to handle large-scale problems, and small-scale problems cannot fully utilize the available parallel cores. In such cases, the overhead related to parallelism, such as task scheduling and thread launching, dominates the computation time rather than providing actual performance gains. For high precision, however, the GPU solver does not achieve the same level of performance as the CPU solver. This is anticipated, as the condensed system utilized by the GPU solvers often encounters worse conditioning in practice. Overall, the GPU solver converges in two fewer instances than the CPU solver. However, for the converged instances, the speed-up is still significant (3x speed-up in SGM10 on large instances).

\paragraph{COPS Benchmark}
The results are similar to those of the pglib-opf benchmark, but the speed-up is more pronounced in these instances, as the COPS benchmark library is designed to be scalable~\cite{dolanBenchmarkingOptimizationSoftware2001} (thus exhibiting a highly regular structure, which provides more opportunities for exploiting parallelism).
Again, for large instances, one can achieve more than a 10x speed-up on average. The speed-up is also substantial for high-precision solves, although the speed-up factor is smaller overall.

\section{Conclusions and Future Outlook}\label{eqn:conclusion}
We have presented a current landscape of GPU-accelerated second-order optimization solvers.
We have also demonstrated with numerical examples that GPU acceleration can achieve more than an order of magnitude speed-up for large instances when solved up to medium precision. However, some open questions and implementation challenges remain, which we summarize below.

\begin{itemize}[leftmargin=*,itemsep=0pt,parsep=0pt,partopsep=0pt]
\item \textit{Numerical Precision of Condensed KKT Systems}: Condensed \gls*{kkt} systems are often preferred in pivoting-free implementations, but condensation may detrimentally affect numerical stability. Further research is needed to understand the numerical properties of condensed KKT systems and to develop strategies to mitigate the numerical issues for high-precision solves.
\item \textit{Alternative Strategies}: While we have presented the plain \gls*{ipm}, it can be combined with other optimization strategies. Augmented Lagrangian or penalty methods, which naturally provide a mechanism to eliminate the indefiniteness of the KKT system, may be worth investigating.
\item \textit{Batch Solvers}: GPU-accelerated solvers can solve many small- and medium-sized problems in parallel; thus, developing batch mathematical programming solvers is worthwhile. Batch linear systems (with or without uniform sparsity patterns) have been supported by CUDSS since v0.6.
\item \textit{Hardware Portability}: Currently, most existing optimization and linear solvers are limited to NVIDIA GPUs, but there is interest in developing hardware-agnostic solvers that can run on a variety of GPU architectures, including AMD and Intel GPUs. A key requirement for this will be the development of portable sparse LDL$^\top$ factorizations.
\end{itemize}


\bibliographystyle{plain}
\bibliography{shin}

\appendix

\section{More Details on Numerical Results}\label{apx:num}

\subsection{Solver Configuration}
\subsubsection{LP Solvers}
\paragraph{MadIPM}

\paragraph{Gurobi}
\subsubsection{NLP Solvers}
\paragraph{MadNLP}


\paragraph{Ipopt}

\newpage
\section*{NeurIPS Paper Checklist}

%%% BEGIN INSTRUCTIONS %%%
The checklist is designed to encourage best practices for responsible machine learning research, addressing issues of reproducibility, transparency, research ethics, and societal impact. Do not remove the checklist: {\bf The papers not including the checklist will be desk rejected.} The checklist should follow the references and follow the (optional) supplemental material.  The checklist does NOT count towards the page
limit.

Please read the checklist guidelines carefully for information on how to answer these questions. For each question in the checklist:
\begin{itemize}
    \item You should answer \answerYes{}, \answerNo{}, or \answerNA{}.
    \item \answerNA{} means either that the question is Not Applicable for that particular paper or the relevant information is Not Available.
    \item Please provide a short (1–2 sentence) justification right after your answer (even for NA).
   % \item {\bf The papers not including the checklist will be desk rejected.}
\end{itemize}

{\bf The checklist answers are an integral part of your paper submission.} They are visible to the reviewers, area chairs, senior area chairs, and ethics reviewers. You will be asked to also include it (after eventual revisions) with the final version of your paper, and its final version will be published with the paper.

The reviewers of your paper will be asked to use the checklist as one of the factors in their evaluation. While "\answerYes{}" is generally preferable to "\answerNo{}", it is perfectly acceptable to answer "\answerNo{}" provided a proper justification is given (e.g., "error bars are not reported because it would be too computationally expensive" or "we were unable to find the license for the dataset we used"). In general, answering "\answerNo{}" or "\answerNA{}" is not grounds for rejection. While the questions are phrased in a binary way, we acknowledge that the true answer is often more nuanced, so please just use your best judgment and write a justification to elaborate. All supporting evidence can appear either in the main paper or the supplemental material, provided in appendix. If you answer \answerYes{} to a question, in the justification please point to the section(s) where related material for the question can be found.

IMPORTANT, please:
\begin{itemize}
    \item {\bf Delete this instruction block, but keep the section heading ``NeurIPS Paper Checklist"},
    \item  {\bf Keep the checklist subsection headings, questions/answers and guidelines below.}
    \item {\bf Do not modify the questions and only use the provided macros for your answers}.
\end{itemize}


%%% END INSTRUCTIONS %%%


\begin{enumerate}

\item {\bf Claims}
    \item[] Question: Do the main claims made in the abstract and introduction accurately reflect the paper's contributions and scope?
    \item[] Answer: \answerTODO{} % Replace by \answerYes{}, \answerNo{}, or \answerNA{}.
    \item[] Justification: \justificationTODO{}
    \item[] Guidelines:
    \begin{itemize}
        \item The answer NA means that the abstract and introduction do not include the claims made in the paper.
        \item The abstract and/or introduction should clearly state the claims made, including the contributions made in the paper and important assumptions and limitations. A No or NA answer to this question will not be perceived well by the reviewers.
        \item The claims made should match theoretical and experimental results, and reflect how much the results can be expected to generalize to other settings.
        \item It is fine to include aspirational goals as motivation as long as it is clear that these goals are not attained by the paper.
    \end{itemize}

\item {\bf Limitations}
    \item[] Question: Does the paper discuss the limitations of the work performed by the authors?
    \item[] Answer: \answerTODO{} % Replace by \answerYes{}, \answerNo{}, or \answerNA{}.
    \item[] Justification: \justificationTODO{}
    \item[] Guidelines:
    \begin{itemize}
        \item The answer NA means that the paper has no limitation while the answer No means that the paper has limitations, but those are not discussed in the paper.
        \item The authors are encouraged to create a separate "Limitations" section in their paper.
        \item The paper should point out any strong assumptions and how robust the results are to violations of these assumptions (e.g., independence assumptions, noiseless settings, model well-specification, asymptotic approximations only holding locally). The authors should reflect on how these assumptions might be violated in practice and what the implications would be.
        \item The authors should reflect on the scope of the claims made, e.g., if the approach was only tested on a few datasets or with a few runs. In general, empirical results often depend on implicit assumptions, which should be articulated.
        \item The authors should reflect on the factors that influence the performance of the approach. For example, a facial recognition algorithm may perform poorly when image resolution is low or images are taken in low lighting. Or a speech-to-text system might not be used reliably to provide closed captions for online lectures because it fails to handle technical jargon.
        \item The authors should discuss the computational efficiency of the proposed algorithms and how they scale with dataset size.
        \item If applicable, the authors should discuss possible limitations of their approach to address problems of privacy and fairness.
        \item While the authors might fear that complete honesty about limitations might be used by reviewers as grounds for rejection, a worse outcome might be that reviewers discover limitations that aren't acknowledged in the paper. The authors should use their best judgment and recognize that individual actions in favor of transparency play an important role in developing norms that preserve the integrity of the community. Reviewers will be specifically instructed to not penalize honesty concerning limitations.
    \end{itemize}

\item {\bf Theory assumptions and proofs}
    \item[] Question: For each theoretical result, does the paper provide the full set of assumptions and a complete (and correct) proof?
    \item[] Answer: \answerTODO{} % Replace by \answerYes{}, \answerNo{}, or \answerNA{}.
    \item[] Justification: \justificationTODO{}
    \item[] Guidelines:
    \begin{itemize}
        \item The answer NA means that the paper does not include theoretical results.
        \item All the theorems, formulas, and proofs in the paper should be numbered and cross-referenced.
        \item All assumptions should be clearly stated or referenced in the statement of any theorems.
        \item The proofs can either appear in the main paper or the supplemental material, but if they appear in the supplemental material, the authors are encouraged to provide a short proof sketch to provide intuition.
        \item Inversely, any informal proof provided in the core of the paper should be complemented by formal proofs provided in appendix or supplemental material.
        \item Theorems and Lemmas that the proof relies upon should be properly referenced.
    \end{itemize}

    \item {\bf Experimental result reproducibility}
    \item[] Question: Does the paper fully disclose all the information needed to reproduce the main experimental results of the paper to the extent that it affects the main claims and/or conclusions of the paper (regardless of whether the code and data are provided or not)?
    \item[] Answer: \answerTODO{} % Replace by \answerYes{}, \answerNo{}, or \answerNA{}.
    \item[] Justification: \justificationTODO{}
    \item[] Guidelines:
    \begin{itemize}
        \item The answer NA means that the paper does not include experiments.
        \item If the paper includes experiments, a No answer to this question will not be perceived well by the reviewers: Making the paper reproducible is important, regardless of whether the code and data are provided or not.
        \item If the contribution is a dataset and/or model, the authors should describe the steps taken to make their results reproducible or verifiable.
        \item Depending on the contribution, reproducibility can be accomplished in various ways. For example, if the contribution is a novel architecture, describing the architecture fully might suffice, or if the contribution is a specific model and empirical evaluation, it may be necessary to either make it possible for others to replicate the model with the same dataset, or provide access to the model. In general. releasing code and data is often one good way to accomplish this, but reproducibility can also be provided via detailed instructions for how to replicate the results, access to a hosted model (e.g., in the case of a large language model), releasing of a model checkpoint, or other means that are appropriate to the research performed.
        \item While NeurIPS does not require releasing code, the conference does require all submissions to provide some reasonable avenue for reproducibility, which may depend on the nature of the contribution. For example
        \begin{enumerate}
            \item If the contribution is primarily a new algorithm, the paper should make it clear how to reproduce that algorithm.
            \item If the contribution is primarily a new model architecture, the paper should describe the architecture clearly and fully.
            \item If the contribution is a new model (e.g., a large language model), then there should either be a way to access this model for reproducing the results or a way to reproduce the model (e.g., with an open-source dataset or instructions for how to construct the dataset).
            \item We recognize that reproducibility may be tricky in some cases, in which case authors are welcome to describe the particular way they provide for reproducibility. In the case of closed-source models, it may be that access to the model is limited in some way (e.g., to registered users), but it should be possible for other researchers to have some path to reproducing or verifying the results.
        \end{enumerate}
    \end{itemize}


\item {\bf Open access to data and code}
    \item[] Question: Does the paper provide open access to the data and code, with sufficient instructions to faithfully reproduce the main experimental results, as described in supplemental material?
    \item[] Answer: \answerTODO{} % Replace by \answerYes{}, \answerNo{}, or \answerNA{}.
    \item[] Justification: \justificationTODO{}
    \item[] Guidelines:
    \begin{itemize}
        \item The answer NA means that paper does not include experiments requiring code.
        \item Please see the NeurIPS code and data submission guidelines (\url{https://nips.cc/public/guides/CodeSubmissionPolicy}) for more details.
        \item While we encourage the release of code and data, we understand that this might not be possible, so “No” is an acceptable answer. Papers cannot be rejected simply for not including code, unless this is central to the contribution (e.g., for a new open-source benchmark).
        \item The instructions should contain the exact command and environment needed to run to reproduce the results. See the NeurIPS code and data submission guidelines (\url{https://nips.cc/public/guides/CodeSubmissionPolicy}) for more details.
        \item The authors should provide instructions on data access and preparation, including how to access the raw data, preprocessed data, intermediate data, and generated data, etc.
        \item The authors should provide scripts to reproduce all experimental results for the new proposed method and baselines. If only a subset of experiments are reproducible, they should state which ones are omitted from the script and why.
        \item At submission time, to preserve anonymity, the authors should release anonymized versions (if applicable).
        \item Providing as much information as possible in supplemental material (appended to the paper) is recommended, but including URLs to data and code is permitted.
    \end{itemize}


\item {\bf Experimental setting/details}
    \item[] Question: Does the paper specify all the training and test details (e.g., data splits, hyperparameters, how they were chosen, type of optimizer, etc.) necessary to understand the results?
    \item[] Answer: \answerTODO{} % Replace by \answerYes{}, \answerNo{}, or \answerNA{}.
    \item[] Justification: \justificationTODO{}
    \item[] Guidelines:
    \begin{itemize}
        \item The answer NA means that the paper does not include experiments.
        \item The experimental setting should be presented in the core of the paper to a level of detail that is necessary to appreciate the results and make sense of them.
        \item The full details can be provided either with the code, in appendix, or as supplemental material.
    \end{itemize}

\item {\bf Experiment statistical significance}
    \item[] Question: Does the paper report error bars suitably and correctly defined or other appropriate information about the statistical significance of the experiments?
    \item[] Answer: \answerTODO{} % Replace by \answerYes{}, \answerNo{}, or \answerNA{}.
    \item[] Justification: \justificationTODO{}
    \item[] Guidelines:
    \begin{itemize}
        \item The answer NA means that the paper does not include experiments.
        \item The authors should answer "Yes" if the results are accompanied by error bars, confidence intervals, or statistical significance tests, at least for the experiments that support the main claims of the paper.
        \item The factors of variability that the error bars are capturing should be clearly stated (for example, train/test split, initialization, random drawing of some parameter, or overall run with given experimental conditions).
        \item The method for calculating the error bars should be explained (closed form formula, call to a library function, bootstrap, etc.)
        \item The assumptions made should be given (e.g., Normally distributed errors).
        \item It should be clear whether the error bar is the standard deviation or the standard error of the mean.
        \item It is OK to report 1-sigma error bars, but one should state it. The authors should preferably report a 2-sigma error bar than state that they have a 96\% CI, if the hypothesis of Normality of errors is not verified.
        \item For asymmetric distributions, the authors should be careful not to show in tables or figures symmetric error bars that would yield results that are out of range (e.g. negative error rates).
        \item If error bars are reported in tables or plots, The authors should explain in the text how they were calculated and reference the corresponding figures or tables in the text.
    \end{itemize}

\item {\bf Experiments compute resources}
    \item[] Question: For each experiment, does the paper provide sufficient information on the computer resources (type of compute workers, memory, time of execution) needed to reproduce the experiments?
    \item[] Answer: \answerTODO{} % Replace by \answerYes{}, \answerNo{}, or \answerNA{}.
    \item[] Justification: \justificationTODO{}
    \item[] Guidelines:
    \begin{itemize}
        \item The answer NA means that the paper does not include experiments.
        \item The paper should indicate the type of compute workers CPU or GPU, internal cluster, or cloud provider, including relevant memory and storage.
        \item The paper should provide the amount of compute required for each of the individual experimental runs as well as estimate the total compute.
        \item The paper should disclose whether the full research project required more compute than the experiments reported in the paper (e.g., preliminary or failed experiments that didn't make it into the paper).
    \end{itemize}

\item {\bf Code of ethics}
    \item[] Question: Does the research conducted in the paper conform, in every respect, with the NeurIPS Code of Ethics \url{https://neurips.cc/public/EthicsGuidelines}?
    \item[] Answer: \answerTODO{} % Replace by \answerYes{}, \answerNo{}, or \answerNA{}.
    \item[] Justification: \justificationTODO{}
    \item[] Guidelines:
    \begin{itemize}
        \item The answer NA means that the authors have not reviewed the NeurIPS Code of Ethics.
        \item If the authors answer No, they should explain the special circumstances that require a deviation from the Code of Ethics.
        \item The authors should make sure to preserve anonymity (e.g., if there is a special consideration due to laws or regulations in their jurisdiction).
    \end{itemize}


\item {\bf Broader impacts}
    \item[] Question: Does the paper discuss both potential positive societal impacts and negative societal impacts of the work performed?
    \item[] Answer: \answerTODO{} % Replace by \answerYes{}, \answerNo{}, or \answerNA{}.
    \item[] Justification: \justificationTODO{}
    \item[] Guidelines:
    \begin{itemize}
        \item The answer NA means that there is no societal impact of the work performed.
        \item If the authors answer NA or No, they should explain why their work has no societal impact or why the paper does not address societal impact.
        \item Examples of negative societal impacts include potential malicious or unintended uses (e.g., disinformation, generating fake profiles, surveillance), fairness considerations (e.g., deployment of technologies that could make decisions that unfairly impact specific groups), privacy considerations, and security considerations.
        \item The conference expects that many papers will be foundational research and not tied to particular applications, let alone deployments. However, if there is a direct path to any negative applications, the authors should point it out. For example, it is legitimate to point out that an improvement in the quality of generative models could be used to generate deepfakes for disinformation. On the other hand, it is not needed to point out that a generic algorithm for optimizing neural networks could enable people to train models that generate Deepfakes faster.
        \item The authors should consider possible harms that could arise when the technology is being used as intended and functioning correctly, harms that could arise when the technology is being used as intended but gives incorrect results, and harms following from (intentional or unintentional) misuse of the technology.
        \item If there are negative societal impacts, the authors could also discuss possible mitigation strategies (e.g., gated release of models, providing defenses in addition to attacks, mechanisms for monitoring misuse, mechanisms to monitor how a system learns from feedback over time, improving the efficiency and accessibility of ML).
    \end{itemize}

\item {\bf Safeguards}
    \item[] Question: Does the paper describe safeguards that have been put in place for responsible release of data or models that have a high risk for misuse (e.g., pretrained language models, image generators, or scraped datasets)?
    \item[] Answer: \answerTODO{} % Replace by \answerYes{}, \answerNo{}, or \answerNA{}.
    \item[] Justification: \justificationTODO{}
    \item[] Guidelines:
    \begin{itemize}
        \item The answer NA means that the paper poses no such risks.
        \item Released models that have a high risk for misuse or dual-use should be released with necessary safeguards to allow for controlled use of the model, for example by requiring that users adhere to usage guidelines or restrictions to access the model or implementing safety filters.
        \item Datasets that have been scraped from the Internet could pose safety risks. The authors should describe how they avoided releasing unsafe images.
        \item We recognize that providing effective safeguards is challenging, and many papers do not require this, but we encourage authors to take this into account and make a best faith effort.
    \end{itemize}

\item {\bf Licenses for existing assets}
    \item[] Question: Are the creators or original owners of assets (e.g., code, data, models), used in the paper, properly credited and are the license and terms of use explicitly mentioned and properly respected?
    \item[] Answer: \answerTODO{} % Replace by \answerYes{}, \answerNo{}, or \answerNA{}.
    \item[] Justification: \justificationTODO{}
    \item[] Guidelines:
    \begin{itemize}
        \item The answer NA means that the paper does not use existing assets.
        \item The authors should cite the original paper that produced the code package or dataset.
        \item The authors should state which version of the asset is used and, if possible, include a URL.
        \item The name of the license (e.g., CC-BY 4.0) should be included for each asset.
        \item For scraped data from a particular source (e.g., website), the copyright and terms of service of that source should be provided.
        \item If assets are released, the license, copyright information, and terms of use in the package should be provided. For popular datasets, \url{paperswithcode.com/datasets} has curated licenses for some datasets. Their licensing guide can help determine the license of a dataset.
        \item For existing datasets that are re-packaged, both the original license and the license of the derived asset (if it has changed) should be provided.
        \item If this information is not available online, the authors are encouraged to reach out to the asset's creators.
    \end{itemize}

\item {\bf New assets}
    \item[] Question: Are new assets introduced in the paper well documented and is the documentation provided alongside the assets?
    \item[] Answer: \answerTODO{} % Replace by \answerYes{}, \answerNo{}, or \answerNA{}.
    \item[] Justification: \justificationTODO{}
    \item[] Guidelines:
    \begin{itemize}
        \item The answer NA means that the paper does not release new assets.
        \item Researchers should communicate the details of the dataset/code/model as part of their submissions via structured templates. This includes details about training, license, limitations, etc.
        \item The paper should discuss whether and how consent was obtained from people whose asset is used.
        \item At submission time, remember to anonymize your assets (if applicable). You can either create an anonymized URL or include an anonymized zip file.
    \end{itemize}

\item {\bf Crowdsourcing and research with human subjects}
    \item[] Question: For crowdsourcing experiments and research with human subjects, does the paper include the full text of instructions given to participants and screenshots, if applicable, as well as details about compensation (if any)?
    \item[] Answer: \answerTODO{} % Replace by \answerYes{}, \answerNo{}, or \answerNA{}.
    \item[] Justification: \justificationTODO{}
    \item[] Guidelines:
    \begin{itemize}
        \item The answer NA means that the paper does not involve crowdsourcing nor research with human subjects.
        \item Including this information in the supplemental material is fine, but if the main contribution of the paper involves human subjects, then as much detail as possible should be included in the main paper.
        \item According to the NeurIPS Code of Ethics, workers involved in data collection, curation, or other labor should be paid at least the minimum wage in the country of the data collector.
    \end{itemize}

\item {\bf Institutional review board (IRB) approvals or equivalent for research with human subjects}
    \item[] Question: Does the paper describe potential risks incurred by study participants, whether such risks were disclosed to the subjects, and whether Institutional Review Board (IRB) approvals (or an equivalent approval/review based on the requirements of your country or institution) were obtained?
    \item[] Answer: \answerTODO{} % Replace by \answerYes{}, \answerNo{}, or \answerNA{}.
    \item[] Justification: \justificationTODO{}
    \item[] Guidelines:
    \begin{itemize}
        \item The answer NA means that the paper does not involve crowdsourcing nor research with human subjects.
        \item Depending on the country in which research is conducted, IRB approval (or equivalent) may be required for any human subjects research. If you obtained IRB approval, you should clearly state this in the paper.
        \item We recognize that the procedures for this may vary significantly between institutions and locations, and we expect authors to adhere to the NeurIPS Code of Ethics and the guidelines for their institution.
        \item For initial submissions, do not include any information that would break anonymity (if applicable), such as the institution conducting the review.
    \end{itemize}

\item {\bf Declaration of LLM usage}
    \item[] Question: Does the paper describe the usage of LLMs if it is an important, original, or non-standard component of the core methods in this research? Note that if the LLM is used only for writing, editing, or formatting purposes and does not impact the core methodology, scientific rigorousness, or originality of the research, declaration is not required.
    %this research?
    \item[] Answer: \answerTODO{} % Replace by \answerYes{}, \answerNo{}, or \answerNA{}.
    \item[] Justification: \justificationTODO{}
    \item[] Guidelines:
    \begin{itemize}
        \item The answer NA means that the core method development in this research does not involve LLMs as any important, original, or non-standard components.
        \item Please refer to our LLM policy (\url{https://neurips.cc/Conferences/2025/LLM}) for what should or should not be described.
    \end{itemize}

\end{enumerate}


\end{document}

%%% Local Variables:
%%% mode: LaTeX
%%% TeX-master: t
%%% End:
