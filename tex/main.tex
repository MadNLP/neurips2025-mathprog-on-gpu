\documentclass{article}

% if you need to pass options to natbib, use, e.g.:
%     \PassOptionsToPackage{numbers, compress}{natbib}
% before loading neurips_2025

% The authors should use one of these tracks.
% Before accepting by the NeurIPS conference, select one of the options below.
% 0. "default" for submission
\usepackage[main]{neurips_2025}
% the "default" option is equal to the "main" option, which is used for the Main Track with double-blind reviewing.
% 1. "main" option is used for the Main Track
%  \usepackage[main]{neurips_2025}
% 2. "position" option is used for the Position Paper Track
%  \usepackage[position]{neurips_2025}
% 3. "dandb" option is used for the Datasets & Benchmarks Track
 % \usepackage[dandb]{neurips_2025}
% 4. "creativeai" option is used for the Creative AI Track
%  \usepackage[creativeai]{neurips_2025}
% 5. "sglblindworkshop" option is used for the Workshop with single-blind reviewing
 % \usepackage[sglblindworkshop]{neurips_2025}
% 6. "dblblindworkshop" option is used for the Workshop with double-blind reviewing
%  \usepackage[dblblindworkshop]{neurips_2025}

% After being accepted, the authors should add "final" behind the track to compile a camera-ready version.
% 1. Main Track
 % \usepackage[main, final]{neurips_2025}
% 2. Position Paper Track
%  \usepackage[position, final]{neurips_2025} 
% 3. Datasets & Benchmarks Track
 % \usepackage[dandb, final]{neurips_2025}
% 4. Creative AI Track
%  \usepackage[creativeai, final]{neurips_2025}
% 5. Workshop with single-blind reviewing
%  \usepackage[sglblindworkshop, final]{neurips_2025}
% 6. Workshop with double-blind reviewing
%  \usepackage[dblblindworkshop, final]{neurips_2025}
% Note. For the workshop paper template, both \title{} and \workshoptitle{} are required, with the former indicating the paper title shown in the title and the latter indicating the workshop title displayed in the footnote.
% For workshops (5., 6.), the authors should add the name of the workshop, "\workshoptitle" command is used to set the workshop title.
% \workshoptitle{WORKSHOP TITLE}

% "preprint" option is used for arXiv or other preprint submissions
 % \usepackage[preprint]{neurips_2025}

% to avoid loading the natbib package, add option nonatbib:
% \usepackage[nonatbib]{neurips_2025}


\usepackage[utf8]{inputenc} % allow utf-8 input
\usepackage[T1]{fontenc}    % use 8-bit T1 fonts
\usepackage{hyperref}       % hyperlinks
\usepackage{url}            % simple URL typesetting
\usepackage{booktabs}       % professional-quality tables
\usepackage{amsfonts}       % blackboard math symbols
\usepackage{nicefrac}       % compact symbols for 1/2, etc. 
\usepackage{microtype}      % microtypography
\usepackage{xcolor}         % colors
\usepackage{glossaries}
\usepackage{enumitem}
\usepackage{listings}
\usepackage{verbatim} 
\lstset{basicstyle=\ttfamily\small}
\usepackage{multirow}
\usepackage{cleveref}
\usepackage{graphicx}

\crefname{equation}{}{}
\usepackage{listings}
\lstset{
  float=htb!,
}  
\sloppy
\newacronym{pde}{PDE}{partial differential equation}
\newacronym{ad}{AD}{automatic differentiation}
\newacronym{alcf}{ALCF}{Argonne Leadership Computing Facility}
\newacronym{blas}{BLAS}{basic linear algebra subprograms}
\newacronym{iap}{IAP}{independent activities period}
\newacronym{der}{DER}{distributed energy resource}
\newacronym{derms}{DERMS}{distributed energy resource management system}
\newacronym{minlp}{MINLP}{mixed-integer nonlinear programming}
\newacronym{nlp}{NLP}{nonlinear programming}
\newacronym{kkt}{KKT}{Karush-Kuhn-Tucker}
\newacronym{sqp}{SQP}{sequential quadratic programming}
\newacronym{ipm}{IPM}{interior-point method}
\newacronym{cpu}{CPU}{central processing units}
\newacronym{gpu}{GPU}{graphics processing units}
\newacronym{mpc}{MPC}{model predictive control}
\newacronym{ac}{AC}{alternating current}
\newacronym{dc}{DC}{direct current}
\newacronym{opf}{OPF}{optimal power flow}
\newacronym{hpc}{HPC}{high-performance computing}
\newacronym{pcg}{PCG}{projected conjugate gradient}
\newacronym{alm}{ALM}{augmented Lagrangian method}
\newacronym{dac}{DAC}{direct air capture}
\newacronym{pem}{PEM}{proton exchange membrane}
\newacronym{tea}{TEA}{technoeconomic analysis}
\newacronym{lca}{LCA}{life cycle assessment}
\newacronym{ft}{FT}{Fischer-Tropsch}
\newacronym{bess}{BESS}{battery energy storage system}
\newacronym{cqa}{CQA}{critical quality attribute}
\newacronym{lnp}{LNP}{lipid nanoparticle}
\newacronym{iso}{ISO}{independent system operator}
\newacronym{simd}{SIMD}{single instruction, multiple data}
\newacronym{mimd}{MIMD}{multiple instruction, multiple data}
\newacronym{orcd}{ORCD}{Office of Research Computing and Data}
\newacronym{mitei}{MITei}{MIT Energy Initiative}
\newacronym{hipsat}{HIP-SAT}{High School Introduction to the Physical Sciences and Advanced Technologies}
\newacronym{mpi}{MPI}{message passing interface}
\newacronym{goc}{GOC}{grid optimization competition}
\newacronym{sc}{SC}{security-constrained}
\newacronym{mp}{MP}{multi-period}
\newacronym{admm}{ADMM}{alternating direction method of multipliers}
\newacronym{gmres}{GMRES}{generalized mean residual}
\newacronym{pdhg}{PDHG}{primal-dual hybrid gradient}
\newacronym{lp}{LP}{linear programming}
\newacronym{ams}{AMS}{algebraic modeling system}
\newacronym{ncl}{NCL}{nonlinear constrained Lagrangian}
\newacronym{bcl}{BCL}{bound constrained Lagrangian}
\newacronym{sqd}{SQD}{symmetric quasi-definite}
\newacronym{spd}{SPD}{symmetric positive definite}


% Note. For the workshop paper template, both \title{} and \workshoptitle{} are required, with the former indicating the paper title shown in the title and the latter indicating the workshop title displayed in the footnote.
\title{GPU Implementation of Second-Order Linear and Nonlinear Programming Solvers}




% The \author macro works with any number of authors. There are two commands
% used to separate the names and addresses of multiple authors: \And and \AND.
%
% Using \And between authors leaves it to LaTeX to determine where to break the
% lines. Using \AND forces a line break at that point. So, if LaTeX puts 3 of 4
% authors names on the first line, and the last on the second line, try using
% \AND instead of \And before the third author name.


\author{%
  Alexis Montoison\\
  Mathematics and Computer Science Division\\
  Argonne National Laboratory\\
  Lemont, IL 60439\\
  \texttt{amontoison@anl.gov}\\
  \And
  Fran\c{c}ois Pacaud\\
  Centre Automatique et Systèmes\\
  Mines Paris-PSL\\
  Paris, 75006 \\
  \texttt{francois.pacaud@minesparis.psl.eu}\\
  \And
  Sungho Shin\\
  Department of Chemical Engineering\\
  Massachusetts Institute of Technology\\
  Cambridge, MA 02139\\
  \texttt{sushin@mit.edu}\\
  \And
  Mihai Anitescu\\
  Mathematics and Computer Science Division\\
  Argonne National Laboratory\\
  Lemont, IL 60439\\
  \texttt{anitescu@mcs.anl.gov}\\
}


\begin{document}


\maketitle


\begin{abstract}
In recent years, GPU-accelerated optimization solvers based on second-order methods (e.g., interior-point methods) have gained momentum with the advent of mature and efficient GPU-accelerated direct sparse linear solvers, such as cuDSS. This paper provides an overview of the state of the art in GPU-based second-order solvers, focusing on \emph{pivoting-free interior-point methods} for large and sparse linear and nonlinear programs. We begin by highlighting the capabilities and limitations of the currently available GPU-accelerated sparse linear solvers. Next, we discuss different formulations of the Karush-Kuhn-Tucker systems for second-order methods and evaluate their suitability for pivoting-free GPU implementations. We also discuss strategies for computing sparse Jacobians and Hessians on GPUs for nonlinear programming. Finally, we present numerical experiments demonstrating the scalability of GPU-based optimization solvers. We observe speedups often exceeding 10× compared to comparable CPU implementations on large-scale instances when solved up to medium precision. Additionally, we examine the current limitations of existing approaches.

\end{abstract}

\section{Introduction}\label{eqn:intro}

This paper focuses on the implementation of solvers for problems of the following form:
\begin{align}\label{eqn:opt}
  \min_{x } \; f(x) \quad \text{s.t.} \quad g(x) \geq 0,
\end{align}
where \(x \in \mathbb{R}^n\) is the decision variable, and \(f: \mathbb{R}^n \to \mathbb{R}\) and \(g: \mathbb{R}^n \to \mathbb{R}^m\) are the smooth objective and constraint functions, respectively. 
For simplicity, we do not explicitly consider equality constraints—these can always be reformulated as pairs of inequality constraints. 
We will discuss both \gls*{lp} (where \(f\) and \(g\) are affine) and \gls*{nlp} (where \(f\) and \(g\) are nonlinear), with an emphasis on algorithms designed for large, sparse instances.

Despite advances in general-purpose GPU computing, state-of-the-art mathematical programming solvers have not widely adopted these techniques. GPUs excel in repetitive computations on large data sets, such as dense matrix multiplication in AI model training. However, many mathematical programming problems in classical application areas are sparse, lack a uniform memory layout, and therefore do not benefit from the same kind of parallelism as dense linear algebra. As a result, integrating GPUs into mathematical programming solvers poses greater challenges and often necessitates substantial modifications to the overall algorithm.

On the one hand, first-order algorithms have emerged as a suitable class for GPU implementation. Since these algorithms rely on sparse matrix-vector multiplication and simple vector operations, implementing GPU acceleration is usually straightforward. Recent successful implementations include cuPDLP \cite{luCuPDLPCStrengthenedImplementation2024,luCuPDLPFurtherEnhanced2025}, cuOSQP \cite{schubigerGPUAccelerationADMM2020}, and cuOPT \cite{NVIDIACuopt2025}. However, the linear convergence rate of first-order methods restricts their effectiveness in applications requiring fast convergence, prompting the exploration of second-order alternatives for applications that require higher accuracy.

On the other hand, second-order solvers inherently rely on \emph{direct linear solvers}. For example, within \gls*{ipm}, each barrier iteration necessitates solving a linear system known as the \gls*{kkt} system. These systems become increasingly ill-conditioned as the iterate approaches the solution, rendering the use of iterative linear solvers, such as preconditioned Krylov methods, ineffective in most cases. Therefore, a reliable direct linear solver is a prerequisite for the effectiveness of second-order solvers. For years, the development of GPU-accelerated second-order solvers has been hindered by the absence of robust and efficient sparse direct linear solvers.

This status quo has changed with NVIDIA's release of cuDSS, a library of direct sparse linear solvers for GPUs \cite{nvidiaNVIDIACuDSSPreview}. It provides sparse Cholesky, LDL$^\top$, and LU factorization routines. While it currently lacks the LBL$^\top$ factorization capabilities commonly used for \gls*{nlp} solvers, its LDL$^\top$ and Cholesky functionalities are sufficient for implementing modified versions of the \gls*{ipm}. Consequently, cuDSS has spurred advances in GPU-accelerated second-order solvers, including MadNLP \cite{shinAcceleratingOptimalPower2024} and Clarabel \cite{goulartClarabelInteriorpointSolver2024}, achieving significant speedups on large-scale instances \cite{shinNVIDIACuDSSLibrary2024,shinAcceleratingOptimalPower2024,pacaudCondensedspaceMethodsNonlinear2024,shinScalableMultiPeriodAC2024,pacaudGPUacceleratedDynamicNonlinear2024}.

This paper provides an overview of the current state of the art in GPU implementations of second-order optimization solvers, with an emphasis on the following aspects: (i) The \gls*{ipm} is considered the primary mechanism for handling inequality constraints, as active-set methods are generally regarded as less scalable \cite{nocedalNumericalOptimization2006}. (ii) We mainly focus on solving KKT systems, since other components, such as line search and barrier updates, can be ported to GPUs straightforwardly using \texttt{map} or \texttt{reduce} operations. (iii) We primarily consider NVIDIA GPUs and the CUDA software stack, as they currently offer the most mature direct sparse solver implementation. (iv) Due to space constraints, hybrid \gls*{kkt} strategies \cite{regevHyKKTHybridDirectiterative2023}, reduced-space methods \cite{pacaudAcceleratingCondensedInteriorPoint2023}, and other domain-specific approaches \cite{adabagMPCGPURealTimeNonlinear2024} are not covered.



\section{Direct Linear Solvers for Optimization}\label{eqn:linear}
This section provides an overview of the direct linear algebra methods frequently employed in second-order methods and discusses the rationale behind the development of \emph{pivoting-free \gls*{ipm}}.

\paragraph{LDL$^\top$ Factorization.}
LDL$^\top$ factorization, a signed variant of Cholesky decomposition, decomposes a matrix $A$ into $LDL^\top$, where $L$ is lower triangular and $D$ is diagonal (for sparse systems, a fill-in reducing reordering $P$ must be employed, resulting in $P^\top A P = L D L^\top$). This method can be utilized to solve $Ax = b$, where the solution is obtained by first solving the lower triangular system $Ly = b$, followed by diagonal scaling with $D^{-1}$ and solving the upper triangular system $L^\top x = y$.

A notable property of LDL$^\top$ factorization is that, provided the matrix $A$ is \gls*{sqd}, the LDL$^\top$ factorization exists for any given permutation of the matrix (so-called \emph{strongly factorizable}) \cite{vanderbeiSymmetricQuasidefiniteMatrices1995}. This \emph{does not imply that numerical stability is guaranteed} for any reordering (see \cite{vanderbeiSymmetricQuasidefiniteMatrices1995}), but in practice, strong factorizability is often sufficient to ensure that these methods can be effectively utilized within optimization solvers \cite{stellatoOSQPOperatorSplitting2020}. Many \gls*{kkt} systems in optimization are \gls*{sqd}, can become \gls*{sqd} with infinitesimal regularization, or can be converted to \gls*{sqd} systems. If the system is \gls*{spd}, which is a sufficient condition for \gls*{sqd}, the LDL$^\top$ factorization or Cholesky factorization exists in a fill-in reducing manner, and the factorization process is always numerically stable. \gls*{spd} systems arise from unconstrained optimization problems or are obtained as a result of condensation, which will be discussed in \Cref{sec:ipm}.

\paragraph{Numerical Pivoting.}

For general indefinite matrices without \gls*{sqd} structure (e.g., augmented systems arising from nonconvex \glspl*{nlp}~\cite{wachterImplementationInteriorpointFilter2006}), the LDL$^\top$ factorization is not guaranteed to exist, and dynamic numerical pivoting is commonly employed to avoid zero pivots and improve the numerical stability of the factorization process. Dynamic numerical pivoting procedures examine a limited set of candidate pivots—typically within a row and column—and select the most suitable pivot according to a stability criterion \cite{schenkFASTFACTORIZATIONPIVOTING}. Three widely used dynamic pivoting strategies are Bunch–Kaufman, rook, and delayed pivoting, which select $1 \times 1$ or $2 \times 2$ pivots, although other variants and hybrid approaches also exist \cite{duffDirectMethodsSparse2017}. The variant of LDL$^\top$ with $2 \times 2$ pivots is often referred to as LBL$^\top$ factorization. If none of these methods succeed, the pivot is perturbed by a small value to allow numerical division \cite{schenkFASTFACTORIZATIONPIVOTING}. This procedure introduces numerical error, which must be corrected through iterative refinements. One of the drawbacks of numerical pivoting is that it requires deviating from the fill-in reducing reordering $P$, leading to additional fill-in and disrupting potential parallelism.

\paragraph{GPU Direct Solvers for Optimization}
As described above, the numerical pivoting procedure is crucial for ensuring the numerical stability of direct sparse linear solvers. However, implementing numerical pivoting has been recognized as one of the most challenging components of direct sparse linear solvers on GPUs, as these strategies are serial in nature \cite{swirydowiczLinearSolversPower2022}. Moreover, since coarse-grained tree-level parallelism must be employed to exploit GPU parallelism, numerical pivoting should be applied in a manner that does not disrupt the parallelism at the elimination tree level, further complicating the implementation. The current version of cuDSS has partial pivoting capabilities, but it does not support the LBL$^\top$ factorization as seen in CPU solvers \cite{nvidiaNVIDIACuDSSPreview}.

Therefore, to fully exploit the benefits of existing GPU direct solvers, it is crucial to ensure that \emph{the \gls*{kkt} system can be solved without numerical pivoting}, which motivates the development of \emph{pivoting-free interior-point methods}. This can be achieved by converting the \gls*{kkt} systems into an \gls*{sqd}, or even \gls*{spd} form, where strong factorizability guarantees the existence of the LDL$^\top$ factorization for any fill-in reducing reordering, allowing the factorization to succeed without relying on pivoting. This can be achieved through regularization or condensation, which we elaborate in \Cref{sec:ipm}. Once the pivoting requirement is eliminated, numerical factorization and triangular solves can be efficiently performed on GPUs \cite{naumovParallelSolutionSparse}. Although algorithms for computing fill-in reducing reorderings (e.g., minimum degree ordering \cite{amestoyApproximateMinimumDegree1996} or nested dissection \cite{karypisMETISSoftwarePackage1997}) are serial (e.g., cuDSS performs this operation on the CPU \cite{nvidiaNVIDIACuDSSPreview}), the reordering needs to be computed only once and can be reused, allowing the overhead to be amortized.




\section{Pivoting-Free Interior-Point Methods}\label{sec:ipm}
We now explain how the \gls*{ipm} can be adapted to avoid numerical pivoting, thereby enabling the use of GPU direct solvers relying only on static pivoting. We first provide a brief overview of the \gls*{ipm} and its KKT system formulation, followed by a discussion about condensed KKT systems.

\paragraph{Interior-Point Methods and KKT Systems.}
The \gls*{ipm} is a class of optimization algorithms designed to solve inequality-constrained optimization problems \cite{nocedalNumericalOptimization2006}. The \gls*{ipm} transforms \cref{eqn:opt} into a sequence of log-barrier subproblems and attempts to solve its \gls*{kkt} conditions:
\begin{align}\label{eqn:kkt}
  \nabla f(x) - \nabla g(x)^\top \lambda = 0, \quad
  S \Lambda e - \mu e = 0, \quad
  g(x) - s = 0,
\end{align}
where $s \in \mathbb{R}^m$ denotes the slack variable used to reformulate the inequality constraints as equality constraints, $\mu > 0$ is the barrier parameter, $\lambda \in \mathbb{R}^m$ are the Lagrange multipliers, $S = \text{diag}(s)$, $\Lambda = \text{diag}(\lambda)$, and $e$ is the vector of ones.

The system~\cref{eqn:kkt} is solved using Newton's method. At each iteration, we obtain the search direction by solving the following (regularized and symmetrized) KKT system:
\begin{align}\label{eqn:kkt_system}
  \begin{bmatrix}
    \nabla^2_{x x} \mathcal{L}(x,s,\lambda) + \delta_p I & & \nabla g(x)^\top
    \\ & S^{-1}\Lambda &  -I
    \\ \nabla g(x) & -I &  - \delta_d I
  \end{bmatrix}
  \begin{bmatrix}
    \phantom{-}d_x \\
    \phantom{-}d_s \\
    -d_\lambda
  \end{bmatrix} =
  -\begin{bmatrix}
    \nabla f(x) - \nabla g(x)^\top \lambda\\
    \Lambda e - \mu S^{-1}e \\
    g(x) - s\\
  \end{bmatrix},
\end{align}
where $\mathcal{L}(x,s,\lambda) := f(x) - \lambda^\top (g(x)-s)$ and $\delta_p, \delta_d $ are the primal-dual regularization parameters. 

\paragraph{Regularization.}
The regularization parameters are used to ensure (i) the well-posedness of \cref{eqn:kkt_system} and/or (ii) the descent property of the Newton step. For convex problems, infinitesimal $\delta_p, \delta_d > 0$ ensures the \gls*{sqd} condition for the matrix in \cref{eqn:kkt_system}, thus ensuring strong factorizability. This idea has led to several robust \gls*{ipm} implementations on CPUs~\cite{friedlanderPrimalDualRegularized2012}. In nonconvex cases, primal-dual regularization provides a mechanism to impose not only the \gls*{sqd} structure but also (for \glspl*{nlp}) to ensure that the Newton step is a descent direction for a merit function \cite{wachterImplementationInteriorpointFilter2006}. \Gls*{ipm} solvers typically utilize a procedure known as \emph{inertia correction}, where the regularization parameters $(\delta_p, \delta_d)$ are increased until the number of positive, negative, and zero eigenvalues (collectively referred to as inertia, and available as a byproduct of the LDL$^\top$ and LBL$^\top$ factorizations) equals $(n+m, m, 0)$. Excessive regularization is undesirable, as it can potentially distort the step direction, leading to slow convergence.


\paragraph{Condensed KKT Systems.}
While \cref{eqn:kkt_system} is directly addressed by some solvers (e.g., Ipopt \cite{wachterImplementationInteriorpointFilter2006}), the system can be further \emph{condensed} into a so-called \emph{condensed \gls*{kkt} system}. In the context of GPU implementation, condensation offers advantages by either (i) reducing the system size and increasing its density—thereby providing more opportunities for parallelism—or (ii) enforcing the \gls*{spd} structure, which enables a pivoting-free implementation. However, depending on the sparsity pattern, the condensed system can become significantly denser, leading to higher memory requirements and computational overhead. Moreover, since the eliminated blocks are often highly ill-conditioned near the solution, the resulting condensed system may also suffer from ill-conditioning. Below, we outline several condensation strategies.
\begin{itemize}[leftmargin=*,itemsep=0pt,parsep=0pt,partopsep=0pt]
\item \textit{Augmented System}:
Since the \(S^{-1}\Lambda\) block in \cref{eqn:kkt_system} is always invertible due to the nature of the \gls*{ipm}, we can eliminate it to obtain the so-called \emph{augmented KKT system} \cite{nocedalNumericalOptimization2006}:
\begin{equation}\label{eqn:augmentedKKT}
  \begin{bmatrix}
    \nabla^2_{xx} \mathcal{L}(x,s,\lambda) + \delta_p I & \nabla g(x)^\top \\
    \nabla g(x) &  - \delta_d I - \Lambda^{-1} S
  \end{bmatrix}
  \begin{bmatrix}
    \phantom{-}d_x\\
    - d_\lambda
  \end{bmatrix} =
  -\begin{bmatrix}
    \nabla f(x) - \nabla g(x)^\top \lambda\\
    g(x) - \mu \Lambda^{-1} e
  \end{bmatrix}.
\end{equation}
This elimination does not incur significant computational overhead, and the number of non-zero entries in the resulting system does not increase.

\item \textit{Primal Condensed System}:
The \(\delta_d I + \Lambda^{-1}S\) block within \cref{eqn:augmentedKKT} is always invertible, and its elimination gives rise to a \emph{primal condensed KKT system}:
\begin{align}\label{eqn:kkt_primal}
  \left(\nabla^2_{xx} \mathcal{L}(x,s,\lambda) + \delta_p I + \nabla g(x)^\top (\delta_d I + \Lambda^{-1} S)^{-1} \nabla g(x) \right) d_x = - r_p \; ,
\end{align}
where \(r_p\) is an appropriate right-hand side derived from \cref{eqn:augmentedKKT}. This condensation has one key advantage for \glspl*{nlp}: the system becomes \gls*{spd} under the application of primal-dual regularization \((\delta_p, \delta_d)\) chosen based on the standard inertia correction procedure \cite{shinAcceleratingOptimalPower2024}, meaning that \emph{the system can be factorized using Cholesky factorization without numerical pivoting or any reordering in a numerically stable manner}. However, since the Jacobian \(\nabla g(x)\) can have dense rows, the condensed system can become arbitrarily dense, necessitating specialized treatment.

\item \textit{Dual Condensed System}:
When the problem is strongly convex or when the regularization parameter \(\delta_p\) is sufficiently large, the \(\nabla^2 \mathcal{L}(x,s,\lambda) + \delta_p I\) block is invertible, and by eliminating it, we obtain the \emph{dual condensed KKT system}:
\begin{align}\label{eqn:kkt_dual}
  \left(\delta_d I + \Lambda^{-1}S + \nabla g(x)\left(\nabla_{xx}^2 \mathcal{L}(x,s,\lambda) + \delta_p I\right)^{-1} \nabla g(x)^\top\right)
  d_\lambda = - r_d \; ,
\end{align}
where \(r_d\) is an appropriate right-hand side. The formulation in \eqref{eqn:kkt_dual} is often used as the default option for \gls*{lp} solvers with \(\delta_p>0\), and this system is often referred to as the \emph{normal equations}. Assuming that the primal Hessian is \gls*{spd}, this system is also \gls*{spd}, meaning that it can be stably factorized using Cholesky factorization without numerical pivoting. However, this system can also become arbitrarily dense when there is a dense column in \(\nabla g(x)\), which requires special treatment.
\end{itemize}



\paragraph{Pivoting-Free IPM.}
We now explain which KKT system formulation among \cref{eqn:kkt_system,eqn:augmentedKKT,eqn:kkt_dual,eqn:kkt_primal} is suitable for pivoting-free \gls*{ipm} implementations. The key requirement is that the KKT system matrix must be at least \gls*{sqd} without aggressive regularization. We detail the conditions below.
\begin{itemize}[leftmargin=*,itemsep=0pt,parsep=0pt,partopsep=0pt]
\item \textit{Convex Case}:
  For convex programs, all four formulations \cref{eqn:kkt_system,eqn:augmentedKKT,eqn:kkt_dual,eqn:kkt_primal} are appropriate, as any of these systems can become \gls*{sqd} for infinitesimal $\delta_p, \delta_d > 0$. However, \cref{eqn:kkt_dual,eqn:kkt_primal} may achieve better numerical stability due to their \gls*{spd} structure. MadIPM, an existing GPU \gls*{ipm} solver, employs \cref{eqn:kkt_system} with fixed primal-dual regularization.
\item \textit{Nonconvex Case}:
  For nonconvex problems, the primal condensed system \cref{eqn:kkt_primal} is the most suitable, as it can be made \gls*{spd} by choosing the primal-dual regularization parameters $(\delta_p, \delta_d)$ based solely on inertia correction. The augmented systems \cref{eqn:kkt_system,eqn:augmentedKKT} are not suitable because they are not guaranteed to be \gls*{sqd} unless aggressive (beyond what is necessary to ensure the descent condition) regularization parameters $(\delta_p, \delta_d)$ are used. The dual condensed system \cref{eqn:kkt_dual} is also unsuitable, as $(\nabla^2_{x x} \mathcal{L}(x,s,\lambda) + \delta_p I)^{-1}$ is difficult to compute due to nonlinear constraints. MadNLP, an existing GPU \gls*{nlp} solver, employs \cref{eqn:kkt_primal} with primal-dual regularization based on inertia correction.
\end{itemize}




\section{Algebraic Modeling Systems and Automatic Differentiation}\label{eqn:ad}
\Gls*{nlp} solvers require external oracles to evaluate $f$, $g$, and their first and second-order derivatives. In most modern optimization software stacks, the derivative evaluation code (either compiled or interpreted) is generated in a fully automated fashion through the so-called \emph{algebraic modeling systems}, which are typically equipped with \gls*{ad} capabilities, such as AMPL \cite{fourerModelingLanguageMathematical1990}, CasADi \cite{anderssonCasADiSoftwareFramework2019}, JuMP \cite{dunningJuMPModelingLanguage2017}, Pyomo \cite{hartPyomoModelingSolving2011}, and Gravity \cite{hijaziGravityMathematicalModeling2018}. As classical instances of mathematical programming problems are typically sparse, these systems have historically been developed independently of machine learning frameworks, which tend to focus more on dense problems.

To enable efficient derivative evaluations and ensure a fully GPU-resident optimization workflow, it is crucial to develop algebraic modeling systems that provide derivative evaluation code in the form of GPU kernels. To achieve this, one can concentrate on the observation that many practical instances of large-scale sparse mathematical programs exhibit highly repetitive structures. For example, $f$ may be a sum of many terms (e.g., $f(x) = \sum_{p\in P} \widetilde{f}(x; p)$), and $g$ may be a collection of numerous constraints generated from a common template (e.g., $g(x) = \left\{\widetilde{g}(x; p)\right\}_{p\in P}$). If such a structure exists, the evaluation and differentiation of $f$ and $g$ become embarrassingly parallel, making it feasible to construct GPU kernels for them. Emerging algebraic modeling systems, such as ExaModels.jl \cite{shinAcceleratingOptimalPower2024} or PyOptInterface \cite{yangPyOptInterfaceDesignImplementation2024}, are designed to capture this; for instance, ExaModels.jl requires users to specify the objective and constraint functions in the form of an iterator, such as
\begin{verbatim}
  objective(c, 100 * (x[i-1]^2 - x[i])^2 + (x[i-1] - 1)^2 for i = 2:N)
\end{verbatim}
which allows the user to inform the modeling system of repeated structures in the model. Then, the reverse-mode \gls*{ad} is applied to the template, and the resulting code is compiled into a GPU kernel. This approach enables efficient evaluation of the objective and constraints on GPUs, as well as the computation of their derivatives \cite{shinAcceleratingOptimalPower2024}.




\section{Numerical Results}\label{eqn:num}
We benchmarked the performance of two GPU implementations (MadIPM for \glspl*{lp} and MadNLP for \glspl*{nlp}) against reference CPU solvers (Gurobi for \glspl*{lp} and Ipopt for \glspl*{nlp}). We conducted the benchmark using MIPLIB 2010 (for \glspl*{lp}) \cite{kochMIPLIB20102011}, PGLIB-OPF (for \glspl*{nlp}) \cite{babaeinejadsarookolaeePowerGridLibrary2021}, and COPS (for \glspl*{nlp}) \cite{dolanBenchmarkingOptimizationSoftware2001}. The results are summarized in \Cref{tab:results}, and more details can be found in \Cref{apx:num}. These results can be reproduced using the source code available at \url{https://anonymous.4open.science/r/neurips2025-mathprog-on-gpu-0333}. \textit{Disclaimer}: The numerical results presented herein aim to demonstrate the current capabilities of GPU solvers by providing a comparison with comparable implementations on CPUs. This benchmark is not intended for a head-to-head performance comparison of the solvers. For example, some performance-critical options for CPU solvers, such as presolve and crossover, have been disabled to allow for a focused comparison of barrier iteration performance. Additionally, the convergence criteria for each solver differ slightly, and performance comparisons are based on user-facing tolerance options.



\begin{table}
  \footnotesize
  \begin{tabular}{|c|c|c|cc|cc|cc|cc|}
  \hline
  &\multirow{ 3}{*}{\bfseries Tol} & \multirow{ 3}{*}{\bfseries Solver} & \multicolumn{2}{c|}{\textbf{Small}}& \multicolumn{2}{c|}{\textbf{Medium}}& \multicolumn{2}{c|}{\textbf{Large}}& \multicolumn{2}{c|}{\multirow{2}{*}{\textbf{Total}}}\\
  &&& \multicolumn{2}{c|}{nnz $<2^{18}$}& \multicolumn{2}{c|}{$2^{18}\leq$ nnz $<2^{20}$}& \multicolumn{2}{c|}{$2^{20}\leq$ nnz}&&\\
  &&&  Solved & Time &  Solved & Time &  Solved & Time &  Solved & Time \\
  \hline\hline
  \multirow{4}{*}{\rotatebox{90}{\bfseries MIPLIB}}&    \multirow{2}{*}{$10^{-4}$} & MadIPM & 87 & 1.3013 & 56 & 5.0480 & 27 & 19.7925 & 170 & 4.5319  \\
  && Gurobi & 88 & 1.5439 & 58 & 10.4671 & 23 & 78.5783 & 169 & 9.3939  \\
  \cline{2-11}
  &\multirow{2}{*}{$10^{-8}$} & MadIPM & 85 & 2.8157 & 48 & 18.2642 & 25 & 33.1676 & 158 & 10.2820  \\
  && Gurobi & 88 & 1.5708 & 58 & 10.6148 & 24 & 76.3206 & 170 & 9.3826  \\
  \hline\hline
  \multirow{4}{*}{\rotatebox{90}{\bfseries OPF}}&\multirow{2}{*}{1e-4} & MadNLP & 31 & 0.4166 & 24 & 2.6380 & 11 & 3.7040 & 66 & 1.6979  \\
  && Ipopt & 31 & 0.3970 & 24 & 5.0697 & 11 & 38.5053 & 66 & 5.3817  \\
  \cline{2-11}
  &\multirow{2}{*}{1e-8} & MadNLP & 30 & 2.5037 & 24 & 4.6016 & 10 & 12.8040 & 64 & 4.6228  \\
  && Ipopt & 31 & 0.5100 & 24 & 5.4292 & 11 & 37.7818 & 66 & 5.5541  \\
  \hline\hline
  \multirow{4}{*}{\rotatebox{90}{\bfseries COPS}}&\multirow{2}{*}{1e-4} & MadNLP & 13 & 0.8665 & 15 & 4.8665 & 16 & 3.8194 & 44 & 3.2314  \\
  && Ipopt & 13 & 5.2315 & 15 & 15.9701 & 15 & 45.8411 & 43 & 19.2243  \\
  \cline{2-11}
  &\multirow{2}{*}{1e-8} & MadNLP & 13 & 0.8575 & 16 & 1.5572 & 16 & 8.3549 & 45 & 3.3797  \\
  && Ipopt & 13 & 5.9413 & 15 & 17.6758 & 15 & 40.8639 & 43 & 19.2999  \\ 
  \hline
\end{tabular}

%%% Local Variables:
%%% mode: LaTeX
%%% TeX-master: "main"
%%% End:

  \caption{Solution times for CPU solvers (Gurobi and Ipopt) and GPU solvers (MadIPM and MadNLP) are represented using SGM10, defined as $(\prod_{i=1}^n (t_i + 10))^{1/n} - 10$, where $t_i$ denotes the solve time for the $i$-th instance (in seconds; unsolved instances are assigned a maximum wall time of 900 seconds) across various datasets: MIPLIB (88 small, 58 medium, and 28 large \glspl*{lp}), PGLIB-OPF (31 small, 24 medium, and 11 large \glspl*{nlp}), and COPS (13 small, 16 medium, and 16 large \glspl*{nlp}). For Gurobi, the Barrier method is used, with both the Presolve and Crossover options disabled. MadNLP is configured with cuDSS, while Ipopt is configured with either Ma27 (for PGLIB-OPF) or Ma57 (for COPS). All \glspl*{nlp} are modeled using ExaModels, which supports \gls*{nlp} function evaluation on both CPU and GPU. The benchmarking was conducted on a workstation equipped with two Intel Xeon Gold 6130 CPUs, two Quadro GV~100 GPUs, and 128 GB of memory.
  }
  \label{tab:results}
\end{table}



\paragraph{MIPLIB.}
We have performed the benchmark against a curated subset of instances within the MIPLIB 2010 library by selecting 174 instances that are sufficiently large and not trivially solved. The results in \Cref{tab:results} indicate that the GPU solver can achieve, on average, approximately 4x speed-up for the 28 largest instances (with more than $2^{20}$ non-zeros) when the problems are solved to medium precision. The speed-up is relatively modest for medium-sized instances, and there is practically no advantage for small instances. This is expected, as the GPU solver is designed to handle large-scale problems, and small-scale problems cannot fully utilize the available parallel cores. In such cases, the overhead related to parallelism, such as task scheduling and thread launching, dominates the computation time rather than providing actual performance gains. For high precision, however, the speed-up is less pronounced, and the GPU solver solved significantly fewer instances.

\paragraph{PGLIB-OPF.}
We have benchmarked the performance of the solver for solving AC OPF problems based on polar power flow formulations \cite{PowerModelsJLOpenSource}. The results in \Cref{tab:results} indicate that the GPU solver can achieve an average speed-up of more than 10x for large instances when the problems are solved to medium precision. The speed-up is relatively modest for medium-sized instances, and there is practically no advantage for small instances. However, for high precision, the GPU solver does not reach the same level of robustness as the CPU solver, as the condensed system utilized by the GPU solvers often encounters worse conditioning; the GPU solver fails on two more instances. Nevertheless, the overall speed-up remains significant (3x on average for large instances).

\paragraph{COPS Benchmark.}
We conducted benchmarks using the COPS benchmark library on curated instances. The COPS benchmark instances are scalable, allowing users to specify the problem size. For each instance type, we formulated the problem in five different sizes, approximately doubling the number of variables and constraints each time. The results are similar to those of the PGLIB-OPF benchmark, but the speed-up is more pronounced in these instances. Again, for large instances, we can achieve more than a 10x speed-up on average.

\section{Conclusions and Future Outlook}\label{sec:conclusion}
We have presented an overview of the current landscape of GPU-accelerated second-order optimization solvers. With two specific existing solvers—MadIPM and MadNLP—and a modeling environment---ExaModels---we have demonstrated that GPU acceleration can achieve more than an order of magnitude speed-up for large instances when solved to medium precision. Solving problems robustly to high precision remains a challenge for both \gls*{lp} and \gls*{nlp} solvers. Some open questions and implementation challenges are summarized below.


\begin{itemize}[leftmargin=*,itemsep=0pt,parsep=0pt,partopsep=0pt]
\item \textit{Numerical Precision of Condensed KKT Systems}: Condensed \gls*{kkt} systems are often preferred in pivoting-free implementations; however, stability can be compromised. Further research is needed to develop strategies to mitigate stability issues, especially for high-precision solves.
\item \textit{Batch Solvers}: GPU solvers can solve many small- and medium-sized problems in parallel, which can be facilitated through the implementation of batch solvers. Implementing second-order algorithms is feasible, as batch solutions (with or without uniform sparsity patterns) have been supported by CUDSS since version 0.6.
\item \textit{Hardware Portability}: Currently, most existing optimization and linear solvers are limited to NVIDIA GPUs. However, there is interest in developing hardware-agnostic solvers that can run on various GPU architectures, including AMD and Intel GPUs. A key requirement for this will be the development of portable sparse LDL$^\top$ factorizations.
\end{itemize}





\pagebreak

\bibliographystyle{plain}
\bibliography{shin}

\appendix
\crefalias{section}{appendix}

\section{More Details on Numerical Results}\label{apx:num}
\subsection{Solver Options}
\subsection{Gurobi}
\begin{verbatim}
FeasibilityTol = 1e-4 or 1e-8
OptimalityTol = 1e-4 or 1e-8
TimeLimit = 900.0
Method = 2
Presolve = 0
Crossover = 0
Threads = 16
\end{verbatim}
\subsection{Ipopt}
\begin{verbatim}
tol = 1e-4 or 1e-8
bound_relax_factor = 1e-4 or 1e-8
max_wall_time = 900.0
linear_solver = "ma27" or "ma57"
ma57_automatic_scaling = "yes"
dual_inf_tol = 10000.0
constr_viol_tol = 10000.0
compl_inf_tol = 10000.0
honor_original_bounds = "no"
print_timing_statistics = "yes"
\end{verbatim}
\subsection{MadIPM} 
\begin{verbatim}
tol = 1e-4 or 1e-8
max_wall_time = 900.0
max_iter = 500
linear_solver = MadNLPGPU.CUDSSSolver
cudss_algorithm = MadNLP.LDL
regularization = MadIPM.FixedRegularization(1e-8, -1e-8)
print_level = MadNLP.INFO
rethrow_error = true
\end{verbatim}
\subsection{MadNLP}
\begin{verbatim}
tol = 1e-4 or 1e-8
max_wall_time = 900.0
\end{verbatim}
\subsection{Full Numerical Results}
\begin{lstlisting}
--------------------------------------------------------------------
              MIPLIB benchmark results (tol = 0.0001)
--------------------------------------------------------------------
           problem | log2(nnz)|      MadIPM      |      Gurobi        
                   |          | solved|     time | solved|     time     
--------------------------------------------------------------------
              n3-3 |    15.15 |   1   |     0.30 |   1   |     0.26
       neos-506422 |    15.26 |   1   |     0.14 |   1   |     0.15
            ramos3 |    15.26 |   1   |     0.30 |   1   |     0.39
      iis-bupa-cov |    15.42 |   1   |     0.22 |   1   |     0.27
       neos-777800 |    15.47 |   1   |     0.22 |   1   |     0.17
            d10200 |    15.54 |   1   |     0.24 |   1   |     0.16
            hanoi5 |    15.69 |   1   |     0.35 |   1   |     0.78
         ns1778858 |    15.69 |   1   |     0.22 |   1   |     0.28
           eil33-2 |    15.70 |   1   |     0.14 |   1   |     0.17
       neos-941262 |    15.76 |   1   |     0.40 |   1   |     0.42
   lectsched-4-obj |    15.77 |   1   |     0.19 |   1   |     0.24
       neos-984165 |    15.79 |   1   |     0.41 |   1   |     0.44
       neos-935769 |    15.80 |   1   |     0.32 |   1   |     0.37
       neos-948126 |    15.85 |   1   |     0.39 |   1   |     0.44
        reblock166 |    15.87 |   1   |     0.51 |   1   |     2.31
           lrsa120 |    15.91 |   1   |     0.21 |   1   |     0.79
       neos-935627 |    15.95 |   1   |     0.40 |   1   |     0.44
  rococoC12-111000 |    15.96 |   1   |     0.83 |   1   |     1.50
      neos-1171737 |    15.99 |   1   |     0.29 |   1   |     0.21
       neos-937511 |    16.08 |   1   |     0.43 |   1   |     0.44
            sp98ir |    16.14 |   1   |     0.25 |   1   |     0.22
         atm20-100 |    16.18 |   0   |     0.13 |   1   |     0.43
    methanosarcina |    16.18 |   1   |     0.21 |   1   |     0.95
       neos-937815 |    16.19 |   1   |     0.54 |   1   |     0.53
       neos-826812 |    16.22 |   1   |     0.57 |   1   |     0.38
    satellites1-25 |    16.26 |   1   |     0.86 |   1   |     1.14
          wachplan |    16.27 |   1   |     0.28 |   1   |     0.27
      iis-pima-cov |    16.30 |   1   |     0.35 |   1   |     0.82
          dano3mip |    16.34 |   1   |     0.85 |   1   |     4.34
           biella1 |    16.35 |   1   |     0.75 |   1   |     0.46
           30n20b8 |    16.39 |   1   |     0.37 |   1   |     0.44
             air04 |    16.47 |   1   |     0.41 |   1   |     0.47
       neos-826694 |    16.52 |   1   |     0.50 |   1   |     0.35
         queens-30 |    16.55 |   1   |     0.14 |   1   |     0.22
      neos-1605075 |    16.64 |   1   |     0.77 |   1   |     1.15
      neos-1605061 |    16.66 |   1   |     0.99 |   1   |     1.27
   ash608gpia-3col |    16.68 |   1   |     0.64 |   1   |     1.31
          blp-ic97 |    16.74 |   1   |     0.32 |   1   |     0.28
            sts405 |    16.75 |   1   |     0.36 |   1   |     0.39
             sct32 |    16.77 |   1   |     0.74 |   1   |     0.86
        opm2-z7-s2 |    16.82 |   1   |     2.11 |   1   |     5.64
      rmatr200-p20 |    16.85 |   1   |     0.50 |   1   |     1.86
       neos-693347 |    16.86 |   1   |     0.47 |   1   |     0.64
       lectsched-2 |    16.87 |   1   |     0.37 |   1   |     0.55
      neos-1109824 |    16.89 |   1   |     0.74 |   1   |    12.48
              sct1 |    16.89 |   1   |     1.70 |   1   |     1.68
             net12 |    16.90 |   1   |     1.00 |   1   |     3.85
         momentum1 |    17.04 |   1   |     0.99 |   1   |    14.44
         shipsched |    17.06 |   1   |     0.77 |   1   |     0.87
              dc1c |    17.07 |   1   |     1.16 |   1   |     1.12
       neos-916792 |    17.07 |   1   |     0.21 |   1   |     0.71
              leo1 |    17.08 |   1   |     0.51 |   1   |     0.90
      rmatr200-p10 |    17.10 |   1   |     0.55 |   1   |     2.35
       neos-738098 |    17.14 |   1   |     0.57 |   1   |     0.68
       rmatr200-p5 |    17.20 |   1   |     0.55 |   1   |     3.45
       neos-952987 |    17.22 |   1   |     0.54 |   1   |     0.81
         ex1010-pi |    17.23 |   1   |     0.99 |   1   |     1.08
            mzzv11 |    17.30 |   1   |     1.15 |   1   |     2.14
       neos-934278 |    17.43 |   1   |     1.13 |   1   |     1.16
        neos808444 |    17.44 |   1   |     0.84 |   1   |     0.82
            d20200 |    17.44 |   1   |     0.41 |   1   |     0.43
       lectsched-3 |    17.48 |   1   |     0.59 |   1   |     0.86
              sct5 |    17.50 |   1   |     1.34 |   1   |     2.19
          germanrr |    17.51 |   1   |     0.44 |   1   |     0.91
 satellites2-60-fs |    17.59 |   1   |     1.61 |   1   |     2.37
              bab5 |    17.63 |   1   |     0.88 |   1   |     0.81
       lectsched-1 |    17.64 |   1   |     0.59 |   1   |     0.96
   lectsched-1-obj |    17.64 |   1   |     0.61 |   1   |     1.13
       neos-824661 |    17.65 |   1   |     1.16 |   1   |     0.70
             t1722 |    17.65 |   1   |     0.66 |   1   |     0.89
      core2536-691 |    17.68 |   1   |     1.03 |   1   |     2.20
            dolom1 |    17.68 |   1   |     1.28 |   1   |     1.11
          ns930473 |    17.68 |   1   |     0.99 |   1   |     1.14
        reblock420 |    17.68 |   1   |     2.31 |   1   |    18.20
             sing2 |    17.70 |   1   |     0.95 |   1   |     1.95
         ns1456591 |    17.71 |   1   |     0.77 |   1   |     0.35
          blp-ar98 |    17.73 |   1   |     0.53 |   1   |     0.49
         stockholm |    17.81 |   1   |     4.06 |   1   |     9.09
              leo2 |    17.82 |   1   |     0.68 |   1   |     0.45
              bab1 |    17.87 |   1   |     0.84 |   1   |     1.23
           rmine10 |    17.90 |   1   |     2.94 |   1   |    28.49
       neos-933966 |    17.92 |   1   |     1.83 |   1   |     1.19
         uc-case11 |    17.92 |   1   |     1.69 |   1   |     2.26
        rocII-4-11 |    17.96 |   1   |     0.76 |   1   |     0.83
       neos-933638 |    17.98 |   1   |     1.83 |   1   |     1.42
       neos-885086 |    18.01 |   1   |     0.67 |   1   |     0.61
            app1-2 |    18.01 |   1   |     0.92 |   1   |     4.54
             neos6 |    18.04 |   1   |     0.59 |   1   |     0.64
     core4872-1529 |    18.05 |   1   |     2.16 |   1   |     4.09
         ns1685374 |    18.07 |   1   |     1.38 |   1   |     2.74
            neos13 |    18.09 |   1   |     0.55 |   1   |     0.47
            sp97ar |    18.29 |   1   |     0.74 |   1   |     0.63
         ns2124243 |    18.30 |   1   |     1.34 |   1   |     1.16
         ns1905797 |    18.32 |   1   |     1.35 |   1   |    51.37
         ns1952667 |    18.36 |   1   |     0.21 |   1   |     0.13
            sp98ic |    18.37 |   1   |     0.59 |   1   |     0.59
       tanglegram1 |    18.39 |   1   |     0.83 |   1   |     1.04
         momentum2 |    18.43 |   1   |     1.14 |   1   |     7.72
        atlanta-ip |    18.44 |   0   |    14.87 |   1   |    17.60
          circ10-3 |    18.44 |   1   |     0.97 |   1   |    14.49
            sts729 |    18.44 |   1   |     2.48 |   1   |     1.16
             map14 |    18.45 |   1   |     4.22 |   1   |    11.68
             map20 |    18.45 |   1   |     4.36 |   1   |     9.01
             map06 |    18.45 |   1   |     4.19 |   1   |    12.88
             map10 |    18.45 |   1   |     4.14 |   1   |    11.30
          uc-case3 |    18.45 |   1   |     1.69 |   1   |     3.44
             map18 |    18.45 |   1   |     4.58 |   1   |     8.98
    satellites2-60 |    18.46 |   1   |     4.30 |   1   |     6.45
       neos-520729 |    18.47 |   1   |     1.25 |   1   |     2.15
       neos-957389 |    18.51 |   1   |     0.77 |   1   |     1.08
             nsr8k |    18.73 |   0   |     5.51 |   1   |     5.43
       neos-885524 |    18.75 |   1   |     1.37 |   1   |     1.19
 satellites3-40-fs |    18.80 |   1   |     4.34 |   1   |    10.44
        rocII-7-11 |    18.83 |   1   |     1.36 |   1   |     1.39
           vpphard |    18.85 |   1   |     2.07 |   1   |     6.26
             t1717 |    18.85 |   1   |     1.87 |   1   |     2.22
               ex9 |    18.93 |   1   |     1.75 |   1   |    22.39
               van |    18.99 |   1   |     1.80 |   1   |     5.37
              dc1l |    18.99 |   1   |     1.63 |   1   |     2.63
       opm2-z10-s2 |    19.05 |   1   |     6.91 |   1   |   387.96
          triptim1 |    19.08 |   1   |     2.80 |   1   |     7.54
          triptim2 |    19.09 |   1   |     3.41 |   1   |     7.50
       neos-506428 |    19.09 |   1   |     1.81 |   1   |     4.44
       neos-932816 |    19.09 |   1   |     2.21 |   1   |     2.81
          triptim3 |    19.10 |   1   |     3.24 |   1   |     6.72
         ns1116954 |    19.11 |   1   |    20.40 |   1   |   468.65
         ns1904248 |    19.18 |   1   |     1.90 |   1   |     9.19
           rail507 |    19.18 |   1   |     2.92 |   1   |     2.45
         ns1111636 |    19.19 |   1   |     1.92 |   1   |     1.82
        rocII-9-11 |    19.20 |   1   |     1.87 |   1   |     1.85
             n15-3 |    19.23 |   1   |     2.83 |   1   |     2.90
            rail01 |    19.26 |   1   |     7.10 |   1   |     9.42
        gmut-75-50 |    19.30 |   1   |     2.28 |   1   |     2.52
   pb-simp-nonunif |    19.41 |   1   |     1.93 |   1   |     4.16
       opm2-z11-s8 |    19.52 |   1   |    10.95 |   1   |   395.59
       neos-941313 |    19.67 |   1   |     3.87 |   1   |     2.76
    satellites3-40 |    19.73 |   1   |    17.34 |   1   |    18.92
       neos-859770 |    19.76 |   1   |     0.83 |   1   |     1.03
      neos-1140050 |    19.76 |   1   |     1.15 |   1   |    62.57
      netdiversion |    19.89 |   1   |     5.16 |   1   |     9.36
           rmine14 |    19.92 |   1   |    17.69 |   1   |   177.85
         momentum3 |    19.98 |   1   |     3.14 |   1   |   136.43
    buildingenergy |    19.99 |   1   |     6.55 |   1   |    21.72
           rvb-sub |    20.00 |   1   |     1.81 |   1   |     3.16
      opm2-z12-s14 |    20.02 |   1   |    13.01 |   0   |   905.66
       opm2-z12-s7 |    20.02 |   1   |    16.17 |   0   |   903.53
          vpphard2 |    20.03 |   1   |     5.81 |   1   |    23.73
             stp3d |    20.05 |   1   |    13.54 |   1   |    22.59
         eilA101-2 |    20.06 |   1   |     2.27 |   1   |     3.56
            npmv07 |    20.07 |   0   |    47.94 |   1   |    12.03
         ns2118727 |    20.10 |   1   |     7.03 |   1   |    31.67
           sing245 |    20.10 |   1   |     8.00 |   1   |   100.44
         ns2137859 |    20.11 |   1   |     7.51 |   1   |     5.41
              ex10 |    20.17 |   1   |     5.01 |   1   |   136.32
         ns1854840 |    20.27 |   1   |     6.48 |   1   |    13.41
            rail02 |    20.31 |   1   |    18.35 |   1   |    32.93
       neos-631710 |    20.35 |   1   |     4.41 |   1   |     6.49
           datt256 |    20.53 |   1   |     8.10 |   0   |   900.34
         ns1758913 |    20.89 |   1   |    11.03 |   1   |   886.11
      neos-1429212 |    20.93 |   1   |     4.19 |   1   |    11.67
  wnq-n100-mw99-14 |    20.94 |   1   |    25.02 |   1   |   809.00
         ns1853823 |    20.94 |   1   |    13.41 |   1   |    82.23
            co-100 |    21.00 |   1   |     3.00 |   1   |     4.28
            rail03 |    21.63 |   1   |    48.47 |   1   |    98.37
           n3seq24 |    21.73 |   1   |     7.47 |   1   |    15.93
       neos-476283 |    21.92 |   1   |    46.87 |   1   |    29.43
              bab3 |    21.97 |   1   |    22.44 |   1   |    24.84
           rmine21 |    22.33 |   1   |   233.02 |   0   |   922.05
         ivu06-big |    24.72 |   1   |   337.38 |   1   |   150.96
            mspp16 |    24.74 |   1   |    57.65 |   1   |   544.68
--------------------------------------------------------------------
\end{lstlisting}
    
\begin{lstlisting}
--------------------------------------------------------------------
              MIPLIB benchmark results (tol = 1.0e-8)
--------------------------------------------------------------------
           problem | log2(nnz)|      MadIPM      |      Gurobi        
                   |          | solved|     time | solved|     time     
--------------------------------------------------------------------
              n3-3 |    15.15 |   1   |     0.31 |   1   |     0.23
       neos-506422 |    15.26 |   1   |     0.20 |   1   |     0.10
            ramos3 |    15.26 |   1   |     0.32 |   1   |     0.42
      iis-bupa-cov |    15.42 |   1   |     0.22 |   1   |     0.28
       neos-777800 |    15.47 |   1   |     0.22 |   1   |     0.15
            d10200 |    15.54 |   1   |     0.28 |   1   |     0.17
         ns1778858 |    15.69 |   1   |     3.59 |   1   |     0.31
            hanoi5 |    15.69 |   1   |     0.33 |   1   |     0.76
           eil33-2 |    15.70 |   1   |     0.15 |   1   |     0.14
       neos-941262 |    15.76 |   1   |     0.44 |   1   |     0.41
   lectsched-4-obj |    15.77 |   1   |     0.21 |   1   |     0.23
       neos-984165 |    15.79 |   1   |     0.45 |   1   |     0.43
       neos-935769 |    15.80 |   1   |     0.37 |   1   |     0.37
       neos-948126 |    15.85 |   1   |     0.38 |   1   |     0.42
        reblock166 |    15.87 |   1   |     0.94 |   1   |     2.40
           lrsa120 |    15.91 |   1   |     0.26 |   1   |     0.36
       neos-935627 |    15.95 |   1   |     0.44 |   1   |     0.45
  rococoC12-111000 |    15.96 |   1   |     0.98 |   1   |     1.56
      neos-1171737 |    15.99 |   1   |     0.30 |   1   |     0.21
       neos-937511 |    16.08 |   1   |     0.42 |   1   |     0.41
            sp98ir |    16.14 |   1   |     0.27 |   1   |     0.20
    methanosarcina |    16.18 |   1   |     0.33 |   1   |     5.63
         atm20-100 |    16.18 |   0   |     0.14 |   1   |     0.41
       neos-937815 |    16.19 |   1   |     0.52 |   1   |     0.53
       neos-826812 |    16.22 |   1   |     0.61 |   1   |     0.39
    satellites1-25 |    16.26 |   1   |     7.92 |   1   |     1.06
          wachplan |    16.27 |   1   |     0.31 |   1   |     0.26
      iis-pima-cov |    16.30 |   1   |     0.37 |   1   |     0.76
          dano3mip |    16.34 |   1   |     1.06 |   1   |     1.48
           biella1 |    16.35 |   1   |     0.79 |   1   |     0.46
           30n20b8 |    16.39 |   1   |     0.41 |   1   |     0.45
             air04 |    16.47 |   1   |     0.50 |   1   |     0.40
       neos-826694 |    16.52 |   1   |     0.51 |   1   |     0.34
         queens-30 |    16.55 |   1   |     0.14 |   1   |     0.23
      neos-1605075 |    16.64 |   1   |     1.00 |   1   |     1.22
      neos-1605061 |    16.66 |   1   |     0.96 |   1   |     1.28
   ash608gpia-3col |    16.68 |   1   |     0.72 |   1   |     2.15
          blp-ic97 |    16.74 |   1   |     0.30 |   1   |     0.29
            sts405 |    16.75 |   1   |     0.38 |   1   |     0.37
             sct32 |    16.77 |   0   |     5.31 |   1   |     0.86
        opm2-z7-s2 |    16.82 |   1   |     2.20 |   1   |     5.70
      rmatr200-p20 |    16.85 |   1   |     0.93 |   1   |     1.97
       neos-693347 |    16.86 |   1   |     0.49 |   1   |     0.59
       lectsched-2 |    16.87 |   1   |     0.41 |   1   |     0.57
              sct1 |    16.89 |   0   |    10.24 |   1   |     1.65
      neos-1109824 |    16.89 |   1   |     0.73 |   1   |    12.76
             net12 |    16.90 |   1   |     1.01 |   1   |     3.76
         momentum1 |    17.04 |   1   |     1.44 |   1   |    14.47
         shipsched |    17.06 |   1   |     0.60 |   1   |     0.88
       neos-916792 |    17.07 |   1   |     0.22 |   1   |     0.27
              dc1c |    17.07 |   1   |     4.77 |   1   |     1.13
              leo1 |    17.08 |   1   |     0.37 |   1   |     0.26
      rmatr200-p10 |    17.10 |   1   |     1.08 |   1   |     2.38
       neos-738098 |    17.14 |   1   |     0.69 |   1   |     0.72
       rmatr200-p5 |    17.20 |   1   |     1.13 |   1   |     3.07
       neos-952987 |    17.22 |   1   |     0.67 |   1   |     0.85
         ex1010-pi |    17.23 |   1   |     1.21 |   1   |     1.14
            mzzv11 |    17.30 |   1   |     1.41 |   1   |     1.89
       neos-934278 |    17.43 |   1   |     1.23 |   1   |     1.11
        neos808444 |    17.44 |   1   |     0.82 |   1   |     0.76
            d20200 |    17.44 |   1   |     0.47 |   1   |     0.46
       lectsched-3 |    17.48 |   1   |     0.53 |   1   |     0.88
              sct5 |    17.50 |   1   |     4.20 |   1   |     2.15
          germanrr |    17.51 |   1   |     0.48 |   1   |     0.93
 satellites2-60-fs |    17.59 |   1   |     4.30 |   1   |     2.60
              bab5 |    17.63 |   1   |     1.10 |   1   |     0.81
       lectsched-1 |    17.64 |   1   |     0.55 |   1   |     0.94
   lectsched-1-obj |    17.64 |   1   |     0.64 |   1   |     1.24
             t1722 |    17.65 |   1   |     0.78 |   1   |     0.89
       neos-824661 |    17.65 |   1   |     1.26 |   1   |     0.75
      core2536-691 |    17.68 |   1   |     1.34 |   1   |     2.27
          ns930473 |    17.68 |   1   |     1.08 |   1   |     1.22
        reblock420 |    17.68 |   1   |     2.17 |   1   |    18.07
            dolom1 |    17.68 |   1   |     1.33 |   1   |     1.13
             sing2 |    17.70 |   1   |     1.24 |   1   |     1.96
         ns1456591 |    17.71 |   1   |     0.94 |   1   |     0.39
          blp-ar98 |    17.73 |   1   |     0.60 |   1   |     0.49
         stockholm |    17.81 |   1   |     3.82 |   1   |     9.31
              leo2 |    17.82 |   1   |     1.05 |   1   |     0.94
              bab1 |    17.87 |   1   |     0.90 |   1   |     1.21
           rmine10 |    17.90 |   1   |     3.33 |   1   |    28.39
         uc-case11 |    17.92 |   1   |     1.89 |   1   |     2.28
       neos-933966 |    17.92 |   1   |     1.96 |   1   |     1.21
        rocII-4-11 |    17.96 |   1   |     0.90 |   1   |     0.86
       neos-933638 |    17.98 |   1   |     1.83 |   1   |     1.42
       neos-885086 |    18.01 |   1   |     0.71 |   1   |     0.61
            app1-2 |    18.01 |   1   |     1.19 |   1   |     4.37
             neos6 |    18.04 |   1   |     0.59 |   1   |     0.61
     core4872-1529 |    18.05 |   1   |     2.95 |   1   |     4.87
         ns1685374 |    18.07 |   1   |     1.44 |   1   |     3.81
            neos13 |    18.09 |   1   |     0.67 |   1   |     0.52
            sp97ar |    18.29 |   1   |     0.93 |   1   |     0.64
         ns2124243 |    18.30 |   1   |     1.49 |   1   |     1.21
         ns1905797 |    18.32 |   1   |     1.44 |   1   |    49.25
         ns1952667 |    18.36 |   1   |     0.52 |   1   |     0.13
            sp98ic |    18.37 |   1   |     0.63 |   1   |     0.68
       tanglegram1 |    18.39 |   1   |     0.91 |   1   |     0.99
         momentum2 |    18.43 |   1   |     1.50 |   1   |     7.80
            sts729 |    18.44 |   1   |     2.53 |   1   |     1.19
          circ10-3 |    18.44 |   1   |     1.12 |   1   |    14.33
        atlanta-ip |    18.44 |   0   |    13.35 |   1   |    17.24
             map20 |    18.45 |   0   |    19.16 |   1   |     8.95
             map14 |    18.45 |   0   |    18.57 |   1   |    12.04
             map18 |    18.45 |   0   |    19.42 |   1   |     9.71
          uc-case3 |    18.45 |   1   |     1.84 |   1   |     3.54
             map10 |    18.45 |   0   |    19.66 |   1   |    11.79
             map06 |    18.45 |   0   |    18.64 |   1   |    11.85
    satellites2-60 |    18.46 |   1   |     8.63 |   1   |     6.61
       neos-520729 |    18.47 |   1   |     1.44 |   1   |     2.21
       neos-957389 |    18.51 |   1   |     0.77 |   1   |     1.08
             nsr8k |    18.73 |   0   |     5.41 |   1   |     5.31
       neos-885524 |    18.75 |   1   |     1.31 |   1   |     1.27
 satellites3-40-fs |    18.80 |   0   |    39.26 |   1   |    11.04
        rocII-7-11 |    18.83 |   1   |     1.37 |   1   |     1.41
           vpphard |    18.85 |   1   |     2.23 |   1   |     6.22
             t1717 |    18.85 |   1   |     2.01 |   1   |     2.49
               ex9 |    18.93 |   1   |     1.84 |   1   |    19.90
               van |    18.99 |   1   |     2.01 |   1   |     5.49
              dc1l |    18.99 |   1   |     2.63 |   1   |     2.61
       opm2-z10-s2 |    19.05 |   1   |    33.31 |   1   |   390.08
          triptim1 |    19.08 |   0   |    29.71 |   1   |     7.69
       neos-506428 |    19.09 |   1   |     1.84 |   1   |     4.35
       neos-932816 |    19.09 |   1   |     2.50 |   1   |     2.75
          triptim2 |    19.09 |   1   |     4.32 |   1   |     7.49
          triptim3 |    19.10 |   1   |     4.14 |   1   |     6.70
         ns1116954 |    19.11 |   1   |    21.46 |   1   |   472.68
           rail507 |    19.18 |   1   |     2.97 |   1   |     2.49
         ns1904248 |    19.18 |   1   |     2.09 |   1   |     9.54
         ns1111636 |    19.19 |   1   |     2.10 |   1   |     1.83
        rocII-9-11 |    19.20 |   1   |     1.94 |   1   |     1.99
             n15-3 |    19.23 |   1   |     2.71 |   1   |     2.94
            rail01 |    19.26 |   1   |     8.13 |   1   |     9.57
        gmut-75-50 |    19.30 |   1   |     2.58 |   1   |     2.45
   pb-simp-nonunif |    19.41 |   1   |     1.91 |   1   |     4.06
       opm2-z11-s8 |    19.52 |   1   |    45.95 |   1   |   393.77
       neos-941313 |    19.67 |   1   |     4.94 |   1   |     2.60
    satellites3-40 |    19.73 |   0   |   221.62 |   1   |    20.07
      neos-1140050 |    19.76 |   1   |     9.59 |   1   |   150.90
       neos-859770 |    19.76 |   1   |     0.88 |   1   |     1.03
      netdiversion |    19.89 |   1   |     5.33 |   1   |     8.88
           rmine14 |    19.92 |   1   |    19.97 |   1   |   179.57
         momentum3 |    19.98 |   1   |     4.30 |   1   |    89.99
    buildingenergy |    19.99 |   1   |     9.61 |   1   |    17.10
           rvb-sub |    20.00 |   1   |     1.87 |   1   |     3.23
      opm2-z12-s14 |    20.02 |   1   |    90.17 |   0   |   904.86
       opm2-z12-s7 |    20.02 |   0   |   168.88 |   0   |   903.73
          vpphard2 |    20.03 |   1   |     5.10 |   1   |    24.21
             stp3d |    20.05 |   1   |    14.57 |   1   |    25.07
         eilA101-2 |    20.06 |   1   |     2.59 |   1   |     3.49
            npmv07 |    20.07 |   0   |    48.03 |   1   |    12.66
           sing245 |    20.10 |   1   |     8.81 |   1   |   106.63
         ns2118727 |    20.10 |   1   |    11.31 |   1   |    32.39
         ns2137859 |    20.11 |   1   |     7.75 |   1   |     5.27
              ex10 |    20.17 |   1   |     4.31 |   1   |    59.36
         ns1854840 |    20.27 |   1   |     6.67 |   1   |    13.37
            rail02 |    20.31 |   1   |    19.56 |   1   |    33.68
       neos-631710 |    20.35 |   1   |     4.26 |   1   |     7.58
           datt256 |    20.53 |   1   |     8.29 |   1   |   709.67
         ns1758913 |    20.89 |   1   |    12.05 |   1   |   872.01
      neos-1429212 |    20.93 |   1   |     4.08 |   1   |     9.32
  wnq-n100-mw99-14 |    20.94 |   1   |    25.63 |   1   |   775.53
         ns1853823 |    20.94 |   1   |    18.23 |   1   |    85.47
            co-100 |    21.00 |   1   |     3.55 |   1   |     4.23
            rail03 |    21.63 |   1   |    51.85 |   1   |   102.11
           n3seq24 |    21.73 |   1   |     7.41 |   1   |    16.68
       neos-476283 |    21.92 |   1   |    93.77 |   1   |    32.56
              bab3 |    21.97 |   1   |    23.74 |   1   |    24.50
           rmine21 |    22.33 |   1   |   265.36 |   0   |   919.34
         ivu06-big |    24.72 |   1   |   353.88 |   1   |   148.94
            mspp16 |    24.74 |   1   |    59.55 |   1   |   559.65
--------------------------------------------------------------------
\end{lstlisting}
    
\begin{verbatim}
--------------------------------------------------------------------
              opf benchmark results (tol = 0.0001)
--------------------------------------------------------------------
           problem | log2(nnz)|      MadNLP      |      Ipopt        
                   |          | solved|     time | solved|     time     
--------------------------------------------------------------------
        case3_lmbd |     7.93 |   1   |     0.12 |   1   |     0.01
         case5_pjm |     8.91 |   1   |     0.18 |   1   |     0.01
       case14_ieee |    10.60 |   1   |     0.09 |   1   |     0.01
   case24_ieee_rts |    11.55 |   1   |     0.15 |   1   |     0.02
         case30_as |    11.63 |   1   |     0.10 |   1   |     0.02
       case30_ieee |    11.63 |   1   |     0.11 |   1   |     0.03
       case39_epri |    11.81 |   1   |     0.25 |   1   |     0.03
       case57_ieee |    12.59 |   1   |     0.14 |   1   |     0.02
          case60_c |    12.74 |   1   |     0.20 |   1   |     0.04
   case73_ieee_rts |    13.21 |   1   |     0.16 |   1   |     0.04
      case118_ieee |    13.81 |   1   |     0.16 |   1   |     0.05
     case89_pegase |    13.96 |   1   |     0.22 |   1   |     0.06
     case200_activ |    14.22 |   1   |     0.12 |   1   |     0.03
       case179_goc |    14.31 |   1   |     0.23 |   1   |     0.10
  case162_ieee_dtc |    14.41 |   1   |     0.36 |   1   |     0.08
      case197_snem |    14.43 |   1   |     0.13 |   1   |     0.04
      case300_ieee |    14.96 |   1   |     0.43 |   1   |     0.11
     case240_pserc |    15.08 |   1   |     1.52 |   1   |     0.63
      case588_sdet |    15.70 |   1   |     0.28 |   1   |     0.15
       case500_goc |    15.78 |   1   |     0.34 |   1   |     0.21
       case793_goc |    16.12 |   1   |     0.31 |   1   |     0.23
   case1354_pegase |    17.23 |   1   |     0.41 |   1   |     0.63
      case1888_rte |    17.58 |   1   |     2.92 |   1   |     1.41
      case1951_rte |    17.62 |   1   |     1.07 |   1   |     1.40
     case1803_snem |    17.71 |   1   |     0.48 |   1   |     1.21
      case2383wp_k |    17.78 |   1   |     0.57 |   1   |     1.28
      case2312_goc |    17.83 |   1   |     0.46 |   1   |     1.02
     case2737sop_k |    17.95 |   1   |     0.46 |   1   |     0.82
      case2736sp_k |    17.95 |   1   |     0.50 |   1   |     1.12
      case2746wp_k |    17.96 |   1   |     0.44 |   1   |     0.99
     case2746wop_k |    17.97 |   1   |     0.40 |   1   |     0.86
      case2000_goc |    18.08 |   1   |     0.44 |   1   |     1.24
      case3012wp_k |    18.08 |   1   |     0.65 |   1   |     1.83
      case3120sp_k |    18.13 |   1   |     0.59 |   1   |     1.67
      case2848_rte |    18.16 |   1   |     1.49 |   1   |     2.45
      case2868_rte |    18.17 |   1   |     2.09 |   1   |     2.81
     case2853_sdet |    18.21 |   1   |     0.60 |   1   |     2.11
      case3022_goc |    18.28 |   1   |     0.53 |   1   |     1.67
      case3375wp_k |    18.30 |   1   |     0.63 |   1   |     2.03
   case2869_pegase |    18.43 |   1   |     0.70 |   1   |     2.04
      case2742_goc |    18.45 |   1   |     2.01 |   1   |     5.59
     case4661_sdet |    18.83 |   1   |     1.01 |   1   |     3.76
      case3970_goc |    18.96 |   1   |     0.70 |   1   |     4.70
      case4917_goc |    18.99 |   1   |     0.69 |   1   |     3.42
      case4020_goc |    19.03 |   1   |     1.08 |   1   |     6.51
      case4601_goc |    19.08 |   1   |     1.01 |   1   |     5.92
      case4837_goc |    19.18 |   1   |     0.90 |   1   |     5.01
      case4619_goc |    19.25 |   1   |     0.91 |   1   |     5.89
      case6468_rte |    19.40 |   1   |    13.65 |   1   |    15.39
      case6495_rte |    19.41 |   1   |    32.78 |   1   |    15.96
      case6470_rte |    19.41 |   1   |    18.65 |   1   |     7.71
      case6515_rte |    19.41 |   1   |     5.59 |   1   |    11.69
 case5658_epigrids |    19.41 |   1   |     0.94 |   1   |     5.22
 case7336_epigrids |    19.75 |   1   |     1.05 |   1   |     6.63
     case10000_goc |    19.96 |   1   |     1.06 |   1   |    11.87
   case8387_pegase |    20.09 |   1   |     1.69 |   1   |    11.61
      case9591_goc |    20.22 |   1   |     1.82 |   1   |    20.10
   case9241_pegase |    20.23 |   1   |     1.64 |   1   |    12.14
case10192_epigrids |    20.31 |   1   |     1.60 |   1   |    15.12
     case10480_goc |    20.44 |   1   |     2.21 |   1   |    21.72
  case13659_pegase |    20.59 |   1   |     2.37 |   1   |    20.39
case20758_epigrids |    21.29 |   1   |     3.18 |   1   |    27.56
     case19402_goc |    21.34 |   1   |     4.02 |   1   |    60.80
     case30000_goc |    21.39 |   1   |     5.54 |   1   |   203.64
     case24464_goc |    21.47 |   1   |     5.09 |   1   |    40.54
case78484_epigrids |    23.20 |   1   |    16.23 |   1   |   339.30
--------------------------------------------------------------------
\end{verbatim}
    

\begin{lstlisting}
--------------------------------------------------------------------
              opf benchmark results (tol = 1.0e-8)
--------------------------------------------------------------------
           problem | log2(nnz)|      MadNLP      |      Ipopt        
                   |          | solved|     time | solved|     time     
--------------------------------------------------------------------
        case3_lmbd |     7.93 |   1   |     0.25 |   1   |     0.01
         case5_pjm |     8.91 |   1   |     0.54 |   1   |     0.02
       case14_ieee |    10.60 |   1   |     0.13 |   1   |     0.02
   case24_ieee_rts |    11.55 |   1   |     0.38 |   1   |     0.02
       case30_ieee |    11.63 |   1   |     0.17 |   1   |     0.02
         case30_as |    11.63 |   1   |     0.12 |   1   |     0.01
       case39_epri |    11.81 |   1   |     0.43 |   1   |     0.02
       case57_ieee |    12.59 |   1   |     0.22 |   1   |     0.02
          case60_c |    12.74 |   1   |     0.30 |   1   |     0.03
   case73_ieee_rts |    13.21 |   1   |     0.39 |   1   |     0.04
      case118_ieee |    13.81 |   1   |     0.30 |   1   |     0.05
     case89_pegase |    13.96 |   1   |     0.24 |   1   |     0.07
     case200_activ |    14.22 |   1   |     0.30 |   1   |     0.05
       case179_goc |    14.31 |   1   |     0.33 |   1   |     0.13
  case162_ieee_dtc |    14.41 |   1   |     0.36 |   1   |     0.08
      case197_snem |    14.43 |   1   |     0.48 |   1   |     0.07
      case300_ieee |    14.96 |   1   |     0.73 |   1   |     0.12
     case240_pserc |    15.08 |   1   |     2.00 |   1   |     0.65
      case588_sdet |    15.70 |   1   |     0.39 |   1   |     0.19
       case500_goc |    15.78 |   1   |     0.60 |   1   |     0.23
       case793_goc |    16.12 |   1   |     0.81 |   1   |     0.28
   case1354_pegase |    17.23 |   1   |     0.83 |   1   |     0.76
      case1888_rte |    17.58 |   0   |    13.39 |   1   |     3.43
      case1951_rte |    17.62 |   1   |    15.49 |   1   |     1.71
     case1803_snem |    17.71 |   1   |     1.14 |   1   |     1.40
      case2383wp_k |    17.78 |   1   |     0.73 |   1   |     1.46
      case2312_goc |    17.83 |   1   |     0.83 |   1   |     1.22
      case2736sp_k |    17.95 |   1   |     0.57 |   1   |     1.17
     case2737sop_k |    17.95 |   1   |     0.52 |   1   |     1.04
      case2746wp_k |    17.96 |   1   |     0.60 |   1   |     1.19
     case2746wop_k |    17.97 |   1   |     0.67 |   1   |     1.05
      case3012wp_k |    18.08 |   1   |     0.91 |   1   |     1.97
      case2000_goc |    18.08 |   1   |     0.81 |   1   |     1.35
      case3120sp_k |    18.13 |   1   |     0.85 |   1   |     1.92
      case2848_rte |    18.16 |   1   |     7.06 |   1   |     2.76
      case2868_rte |    18.17 |   1   |    29.18 |   1   |     2.99
     case2853_sdet |    18.21 |   1   |     0.91 |   1   |     1.82
      case3022_goc |    18.28 |   1   |     1.15 |   1   |     2.03
      case3375wp_k |    18.30 |   1   |     1.41 |   1   |     2.37
   case2869_pegase |    18.43 |   1   |     0.94 |   1   |     2.50
      case2742_goc |    18.45 |   1   |     4.43 |   1   |     6.01
     case4661_sdet |    18.83 |   1   |     2.13 |   1   |     4.16
      case3970_goc |    18.96 |   1   |     1.47 |   1   |     5.18
      case4917_goc |    18.99 |   1   |     1.48 |   1   |     4.24
      case4020_goc |    19.03 |   1   |     1.88 |   1   |     7.04
      case4601_goc |    19.08 |   1   |     2.13 |   1   |     6.23
      case4837_goc |    19.18 |   1   |     1.46 |   1   |     5.23
      case4619_goc |    19.25 |   1   |     1.50 |   1   |     6.15
      case6468_rte |    19.40 |   1   |    11.27 |   1   |    13.55
      case6470_rte |    19.41 |   1   |    20.15 |   1   |     8.39
 case5658_epigrids |    19.41 |   1   |     1.65 |   1   |     5.90
      case6495_rte |    19.41 |   1   |    50.00 |   1   |    16.25
      case6515_rte |    19.41 |   1   |    13.51 |   1   |    12.47
 case7336_epigrids |    19.75 |   1   |     1.89 |   1   |     7.64
     case10000_goc |    19.96 |   1   |     2.39 |   1   |    13.54
   case8387_pegase |    20.09 |   1   |     3.28 |   1   |    13.16
      case9591_goc |    20.22 |   1   |     3.21 |   1   |    22.01
   case9241_pegase |    20.23 |   0   |   114.25 |   1   |    13.93
case10192_epigrids |    20.31 |   1   |     2.84 |   1   |    17.14
     case10480_goc |    20.44 |   1   |     3.24 |   1   |    23.52
  case13659_pegase |    20.59 |   1   |     3.41 |   1   |    17.56
case20758_epigrids |    21.29 |   1   |     7.98 |   1   |    31.55
     case19402_goc |    21.34 |   1   |     5.27 |   1   |    62.56
     case30000_goc |    21.39 |   1   |     9.59 |   1   |    98.95
     case24464_goc |    21.47 |   1   |     4.96 |   1   |    43.97
case78484_epigrids |    23.20 |   1   |    19.62 |   1   |   365.90
--------------------------------------------------------------------
\end{lstlisting}
    
\begin{lstlisting}
--------------------------------------------------------------------
              cops benchmark results (tol = 0.0001)
--------------------------------------------------------------------
           problem | log2(nnz)|      MadNLP      |      Ipopt        
                   |          | solved|     time | solved|     time     
--------------------------------------------------------------------
     camshape-1600 |    14.10 |   1   |     0.19 |   1   |     0.66
     camshape-3200 |    15.10 |   1   |     0.23 |   1   |     3.56
         robot-400 |    15.24 |   1   |     0.37 |   1   |     8.34
     camshape-6400 |    16.10 |   1   |     0.37 |   1   |    16.63
        marine-400 |    16.10 |   1   |     0.26 |   1   |     0.26
         robot-800 |    16.24 |   1   |     3.91 |   1   |     9.54
          elec-100 |    16.67 |   1   |     0.90 |   1   |     1.49
     steering-3200 |    17.10 |   1   |     0.29 |   1   |     0.45
    camshape-12800 |    17.10 |   1   |     0.27 |   1   |    66.00
        marine-800 |    17.10 |   1   |     0.29 |   1   |     0.61
        gasoil-800 |    17.22 |   1   |     0.23 |   1   |     0.46
        robot-1600 |    17.24 |   1   |     4.61 |   1   |     3.59
       rocket-3200 |    17.93 |   1   |     0.38 |   1   |     2.19
        pinene-800 |    18.01 |   1   |     0.48 |   1   |     0.97
       marine-1600 |    18.10 |   1   |     0.46 |   1   |     2.88
     steering-6400 |    18.10 |   1   |     0.62 |   1   |     1.06
    camshape-25600 |    18.10 |   1   |     0.30 |   1   |   301.15
       gasoil-1600 |    18.22 |   1   |     0.47 |   1   |     1.22
        robot-3200 |    18.24 |   0   |    19.21 |   1   |   103.29
   bearing-200,200 |    18.63 |   1   |     0.34 |   1   |     0.78
          elec-200 |    18.68 |   1   |     0.74 |   1   |     4.52
       rocket-6400 |    18.93 |   1   |     0.91 |   1   |    11.88
       pinene-1600 |    19.01 |   1   |     0.80 |   1   |     2.52
       marine-3200 |    19.10 |   1   |     1.05 |   1   |    12.61
    steering-12800 |    19.10 |   1   |     0.97 |   1   |     3.66
       gasoil-3200 |    19.22 |   1   |     0.99 |   1   |     6.14
        robot-6400 |    19.24 |   1   |    11.48 |   0   |   177.21
   bearing-300,300 |    19.79 |   1   |     1.31 |   1   |     1.95
      rocket-12800 |    19.93 |   1   |     1.76 |   1   |    19.82
       pinene-3200 |    20.01 |   1   |     2.17 |   1   |     4.87
       marine-6400 |    20.10 |   1   |     2.07 |   1   |    59.12
    steering-25600 |    20.10 |   1   |     1.78 |   1   |     7.89
       gasoil-6400 |    20.21 |   1   |     1.63 |   1   |    18.30
   bearing-400,400 |    20.62 |   1   |     0.99 |   1   |     3.73
          elec-400 |    20.68 |   1   |     1.90 |   1   |    28.20
      rocket-25600 |    20.93 |   1   |     3.64 |   1   |   113.01
       pinene-6400 |    21.01 |   1   |     2.95 |   1   |    15.86
    steering-51200 |    21.10 |   1   |     3.52 |   1   |    20.09
      gasoil-12800 |    21.21 |   1   |     3.33 |   1   |    24.19
   bearing-600,600 |    21.79 |   1   |     2.30 |   1   |     9.47
      rocket-51200 |    21.93 |   1   |     6.19 |   1   |   856.66
      pinene-12800 |    22.01 |   1   |     5.79 |   1   |    71.92
   bearing-800,800 |    22.61 |   1   |     4.17 |   1   |    19.09
          elec-800 |    22.68 |   1   |     9.27 |   1   |   263.42
         elec-1600 |    24.68 |   1   |    14.56 |   0   |   909.28
--------------------------------------------------------------------
\end{lstlisting}
    
\begin{verbatim}
--------------------------------------------------------------------
              cops benchmark results (tol = 1.0e-8)
--------------------------------------------------------------------
           problem | log2(nnz)|      MadNLP      |      Ipopt        
                   |          | solved|     time | solved|     time     
--------------------------------------------------------------------
     camshape-1600 |    14.10 |   1   |     0.36 |   1   |     1.13
     camshape-3200 |    15.10 |   1   |     0.45 |   1   |     3.89
         robot-400 |    15.24 |   1   |     1.05 |   1   |     1.00
        marine-400 |    16.10 |   1   |     0.30 |   1   |     0.28
     camshape-6400 |    16.10 |   1   |     1.07 |   1   |    17.16
         robot-800 |    16.24 |   1   |     0.74 |   1   |     2.41
          elec-100 |    16.67 |   1   |     0.53 |   1   |     1.50
     steering-3200 |    17.10 |   1   |     0.53 |   1   |     0.52
    camshape-12800 |    17.10 |   1   |     1.20 |   1   |    95.98
        marine-800 |    17.10 |   1   |     0.36 |   1   |     0.81
        gasoil-800 |    17.22 |   1   |     0.55 |   1   |     0.53
        robot-1600 |    17.24 |   1   |     1.73 |   1   |    33.31
       rocket-3200 |    17.93 |   1   |     2.48 |   1   |     1.53
        pinene-800 |    18.01 |   1   |     0.50 |   1   |     1.09
       marine-1600 |    18.10 |   1   |     0.58 |   1   |     3.37
     steering-6400 |    18.10 |   1   |     0.99 |   1   |     1.32
    camshape-25600 |    18.10 |   1   |     1.02 |   1   |   451.05
       gasoil-1600 |    18.22 |   1   |     0.83 |   1   |     1.26
        robot-3200 |    18.24 |   1   |     2.22 |   1   |   147.12
   bearing-200,200 |    18.63 |   1   |     0.39 |   1   |     1.23
          elec-200 |    18.68 |   1   |     2.14 |   1   |     4.58
       rocket-6400 |    18.93 |   1   |     1.50 |   1   |     3.63
       pinene-1600 |    19.01 |   1   |     0.67 |   1   |     2.66
       marine-3200 |    19.10 |   1   |     1.26 |   1   |    23.98
    steering-12800 |    19.10 |   1   |     3.06 |   1   |     4.04
       gasoil-3200 |    19.22 |   1   |     1.66 |   1   |     5.64
        robot-6400 |    19.24 |   1   |     5.72 |   0   |   900.32
   bearing-300,300 |    19.79 |   1   |     0.63 |   1   |     3.17
      rocket-12800 |    19.93 |   1   |     2.81 |   1   |    24.24
       pinene-3200 |    20.01 |   1   |     1.47 |   1   |     5.18
       marine-6400 |    20.10 |   1   |     3.39 |   1   |   123.19
    steering-25600 |    20.10 |   1   |     2.99 |   1   |     8.88
       gasoil-6400 |    20.21 |   1   |     7.75 |   1   |    11.23
   bearing-400,400 |    20.62 |   1   |     1.12 |   1   |     5.82
          elec-400 |    20.68 |   1   |     2.99 |   1   |    28.05
      rocket-25600 |    20.93 |   1   |    19.02 |   1   |    33.88
       pinene-6400 |    21.01 |   1   |     3.64 |   1   |    19.28
    steering-51200 |    21.10 |   1   |     6.69 |   1   |    23.00
      gasoil-12800 |    21.21 |   1   |    24.91 |   1   |    28.71
   bearing-600,600 |    21.79 |   1   |     2.40 |   1   |    14.62
      rocket-51200 |    21.93 |   1   |    27.43 |   1   |    81.64
      pinene-12800 |    22.01 |   1   |     6.35 |   1   |    81.15
   bearing-800,800 |    22.61 |   1   |     4.30 |   1   |    29.51
          elec-800 |    22.68 |   1   |    10.56 |   1   |   329.08
         elec-1600 |    24.68 |   1   |    53.04 |   0   |   909.32
--------------------------------------------------------------------
\end{verbatim}
    


\newpage
\section*{NeurIPS Paper Checklist}

% %%% BEGIN INSTRUCTIONS %%%
% The checklist is designed to encourage best practices for responsible machine learning research, addressing issues of reproducibility, transparency, research ethics, and societal impact. Do not remove the checklist: {\bf The papers not including the checklist will be desk rejected.} The checklist should follow the references and follow the (optional) supplemental material.  The checklist does NOT count towards the page
% limit.

% Please read the checklist guidelines carefully for information on how to answer these questions. For each question in the checklist:
% \begin{itemize}
%     \item You should answer \answerYes{}, \answerNo{}, or \answerNA{}.
%     \item \answerNA{} means either that the question is Not Applicable for that particular paper or the relevant information is Not Available.
%     \item Please provide a short (1–2 sentence) justification right after your answer (even for NA).
%    % \item {\bf The papers not including the checklist will be desk rejected.}
% \end{itemize}

% {\bf The checklist answers are an integral part of your paper submission.} They are visible to the reviewers, area chairs, senior area chairs, and ethics reviewers. You will be asked to also include it (after eventual revisions) with the final version of your paper, and its final version will be published with the paper.

% The reviewers of your paper will be asked to use the checklist as one of the factors in their evaluation. While "\answerYes{}" is generally preferable to "\answerNo{}", it is perfectly acceptable to answer "\answerNo{}" provided a proper justification is given (e.g., "error bars are not reported because it would be too computationally expensive" or "we were unable to find the license for the dataset we used"). In general, answering "\answerNo{}" or "\answerNA{}" is not grounds for rejection. While the questions are phrased in a binary way, we acknowledge that the true answer is often more nuanced, so please just use your best judgment and write a justification to elaborate. All supporting evidence can appear either in the main paper or the supplemental material, provided in appendix. If you answer \answerYes{} to a question, in the justification please point to the section(s) where related material for the question can be found.

% IMPORTANT, please:
% \begin{itemize}
%     \item {\bf Delete this instruction block, but keep the section heading ``NeurIPS Paper Checklist"},
%     \item  {\bf Keep the checklist subsection headings, questions/answers and guidelines below.}
%     \item {\bf Do not modify the questions and only use the provided macros for your answers}.
% \end{itemize}


% %%% END INSTRUCTIONS %%%


\begin{enumerate}

\item {\bf Claims}
    \item[] Question: Do the main claims made in the abstract and introduction accurately reflect the paper's contributions and scope?
    \item[] Answer: \answerYes{} % Replace by \answerYes{}, \answerNo{}, or \answerNA{}.
    \item[] Justification: See \Cref{tab:results} and \Cref{apx:num} for numerical results. 10x speed-up on average can be confirmed, in particular, from OPF and COPS benchmarks.

\item {\bf Limitations}
    \item[] Question: Does the paper discuss the limitations of the work performed by the authors?
    \item[] Answer: \answerYes{} % Replace by \answerYes{}, \answerNo{}, or \answerNA{}.
    \item[] Justification: In \Cref{sec:conclusion}, we mention that ``Solving problems robustly to high precision remains a challenge for both \gls*{lp} and \gls*{nlp} solvers.''

\item {\bf Theory assumptions and proofs}
    \item[] Question: For each theoretical result, does the paper provide the full set of assumptions and a complete (and correct) proof?
    \item[] Answer: \answerNA{} % Replace by \answerYes{}, \answerNo{}, or \answerNA{}.
    \item[] Justification: We do not provide any theorems in this paper.

    \item {\bf Experimental result reproducibility}
    \item[] Question: Does the paper fully disclose all the information needed to reproduce the main experimental results of the paper to the extent that it affects the main claims and/or conclusions of the paper (regardless of whether the code and data are provided or not)?
    \item[] Answer: \answerYes{} % Replace by \answerYes{}, \answerNo{}, or \answerNA{}.
    \item[] Justification: We provide the solver options and settings used in \Cref{tab:results} and \Cref{apx:num}. The experiments can be exactly reproduced with the code provided in the repository link.

\item {\bf Open access to data and code}
    \item[] Question: Does the paper provide open access to the data and code, with sufficient instructions to faithfully reproduce the main experimental results, as described in supplemental material?
    \item[] Answer: \answerYes{} % Replace by \answerYes{}, \answerNo{}, or \answerNA{}.
    \item[] Justification: Anonymized repository link is provided.

\item {\bf Experimental setting/details}
    \item[] Question: Does the paper specify all the training and test details (e.g., data splits, hyperparameters, how they were chosen, type of optimizer, etc.) necessary to understand the results?
    \item[] Answer: \answerYes{} % Replace by \answerYes{}, \answerNo{}, or \answerNA{}.
    \item[] Justification: The full details of the experimental settings are provided in \Cref{apx:num} and the repository link.

\item {\bf Experiment statistical significance}
    \item[] Question: Does the paper report error bars suitably and correctly defined or other appropriate information about the statistical significance of the experiments?
    \item[] Answer: \answerNA{} % Replace by \answerYes{}, \answerNo{}, or \answerNA{}.
    \item[] Justification: We do not perform any statistical experiments in this paper.

\item {\bf Experiments compute resources}
    \item[] Question: For each experiment, does the paper provide sufficient information on the computer resources (type of compute workers, memory, time of execution) needed to reproduce the experiments?
    \item[] Answer: \answerYes{} % Replace by \answerYes{}, \answerNo{}, or \answerNA{}.
    \item[] Justification: We provide this in \Cref{tab:results}.

\item {\bf Code of ethics}
    \item[] Question: Does the research conducted in the paper conform, in every respect, with the NeurIPS Code of Ethics \url{https://neurips.cc/public/EthicsGuidelines}?
    \item[] Answer: \answerYes{} % Replace by \answerYes{}, \answerNo{}, or \answerNA{}.
    \item[] Justification: We believe that our research conforms to the NeurIPS Code of Ethics.


\item {\bf Broader impacts}
    \item[] Question: Does the paper discuss both potential positive societal impacts and negative societal impacts of the work performed?
    \item[] Answer: \answerNA{} % Replace by \answerYes{}, \answerNo{}, or \answerNA{}.
      
\item {\bf Licenses for existing assets}
    \item[] Question: Are the creators or original owners of assets (e.g., code, data, models), used in the paper, properly credited and are the license and terms of use explicitly mentioned and properly respected?
    \item[] Answer: \answerNA{} % Replace by \answerYes{}, \answerNo{}, or \answerNA{}.

\item {\bf New assets}
    \item[] Question: Are new assets introduced in the paper well documented and is the documentation provided alongside the assets?
    \item[] Answer: \answerNA{} % Replace by \answerYes{}, \answerNo{}, or \answerNA{}.

\item {\bf Crowdsourcing and research with human subjects}
    \item[] Question: For crowdsourcing experiments and research with human subjects, does the paper include the full text of instructions given to participants and screenshots, if applicable, as well as details about compensation (if any)?
    \item[] Answer: \answerNA{} % Replace by \answerYes{}, \answerNo{}, or \answerNA{}.

\item {\bf Institutional review board (IRB) approvals or equivalent for research with human subjects}
    \item[] Question: Does the paper describe potential risks incurred by study participants, whether such risks were disclosed to the subjects, and whether Institutional Review Board (IRB) approvals (or an equivalent approval/review based on the requirements of your country or institution) were obtained?
    \item[] Answer: \answerNA{} % Replace by \answerYes{}, \answerNo{}, or \answerNA{}.

\item {\bf Declaration of LLM usage}
    \item[] Question: Does the paper describe the usage of LLMs if it is an important, original, or non-standard component of the core methods in this research? Note that if the LLM is used only for writing, editing, or formatting purposes and does not impact the core methodology, scientific rigorousness, or originality of the research, declaration is not required.
    %this research?
    \item[] Answer: \answerNA{} % Replace by \answerYes{}, \answerNo{}, or \answerNA{}.
\end{enumerate}


\end{document}

%%% Local Variables:
%%% mode: LaTeX
%%% TeX-master: t
%%% End:

